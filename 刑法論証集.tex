\documentclass[11pt]{jsarticle}


\usepackage{okumacro}
\usepackage{fancybox}
\usepackage{ascmac}
\usepackage[noalphabet]{pxchfon}  
\setminchofont{A-OTF-RyuminPro-Light.otf}
\setgothicfont{A-OTF-FutoGoB101Pr6N-Bold.otf}



\title{\vspace{-30mm}{\textgt{\Large{刑法論証集}}}}
\date{\vspace{-15mm}}


\begin{document}

\maketitle

\part{総論}

	\section{因果関係}
		行為と結果との間に事実的因果関係は認められるが、同時に介在事情も存在するため、法的因果関係を認めてもよいかが問題となる。
		
		法的因果関係は、偶然的な帰責範囲を防ぐために要求される。
		これは、客観的に存在する全ての事情を判断資料とし、実行行為の危険が結果へと現実化したか否かによって判断される。
		
	\section{不作為犯}
		\subsection{作為義務}
			\subsubsection{ロング}
				ここで刑法199条は「人を殺」す行為について規定しているが、
				不作為義務に反して作為をすることと、作為義務に反して作為をしないことは同価値であるから、
				そうした不作為によって構成要件的結果を惹起したのであれば、当該不作為は構成要件で予定される態様として考えることができるので、不作為を処罰することも可能である。
		
				作為義務の有無は、作為との同価値性の観点から判断される。
				作為による法益侵害は、結果に至る因果の流れを設定し、これを実質的に支配することに特徴がある。
				したがって、作為義務を認めるためには、不作為者による結果原因の支配が認められることが必要である。
				加えて、偶然的に支配を獲得した場合を除くため、保護の引受け、先行行為や客体との社会的関係性等を考慮することが必要である。
			
			\subsubsection{ショート}
				作為義務の有無は、作為との同価値性の観点から結果原因の支配を基礎として、
				偶然的な支配の獲得の場合を除くため、保護の引受け、先行行為や客体との社会的関係性等を考慮して判断する。
	
	\section{正当防衛}
		\subsection{急迫不正の侵害(正当防衛状況)}
			\subsubsection{急迫不正の侵害}
				「急迫不正の侵害」とは、不正な法益侵害が現に存在するか間近に押し迫っていることをいう。
				
				そして、正当防衛は、公的機関による保護が期待できない緊急状況において例外的に認められるものであるから、
				行為者が侵害を予期した上で対抗行為に出た場合、
				対抗行為に先行する事情を含めた行為全般の状況から「緊急状況」と認められない場合は、急迫性が否定される\footnote{最決平成29年4月26日形集71巻4号275頁}。
			
			\subsubsection{自招侵害(侵害を予期していない場合)}
				正当防衛は、公的機関による保護が期待できない緊急状況において例外的に認められるものであるから、
				\MARU{1}不正の行為により自らそれと一体の侵害を招いたときは、
				\MARU{2}招致行為と侵害が緩やかな均衡を保つ限り、緊急状況性が否定され、正当防衛は成立しない\footnote{最決平成20年5月20日形集62巻6号1786頁}。
				
		\subsection{防衛の意思}
			刑法36条の「防衛するため」という文言から、正当防衛の意思が必要である。
			防衛の意思とは、急迫不正の侵害を認識しつつ、これを避けようとする単純な心理状態をいう。
			
			(攻撃の意思が併存する場合)防衛の意思は攻撃の意思が併存していることを以て否定されるものではないが、
			専ら攻撃の意思で反撃を加える行為は、防衛の意思が認められず、正当防衛が否定される。
			
		\subsection{やむを得ずにした行為(相当性)}
			やむを得ずにした行為とは、反撃行為が防衛手段として必要最小限度のものであること、
			すなわち、反撃行為が防衛手段として相当性を有するものであることをいう。
			必要最小限度か否かは、可能な防衛手段の選択肢及び態様を考慮した上で具体的に判断する。
	\section{緊急避難}
		\subsection{やむを得ずにした行為(補充性)}
			やむを得ずにした行為とは、行ったひな行為よりも侵害性の小さな行為では危難を回避できないこと、すなわち、補充性が認めら得る行為をいう。
		
	
		
	\section{過剰防衛・誤想防衛・誤想過剰防衛}
		\subsection{故意犯の成否}
			故意とは犯罪事実の認識・認容をいう。
			構成要件該当事実の認識によって故意が基礎付けられるのは、構成要件が違法性を基礎づける事実の類型であるからである。
			したがって、違法性阻却事由を基礎づける事実を認識している場合、行為者には違法性を基礎づける事実の認識に欠ける以上、故意は否定される。
			
			そこで、行為者の認識を基礎として正当防衛が成立するかを検討する。
			
		\subsection{刑法36条2項準用・適用の可否}
			(故意の誤想過剰防衛の場合)誤想した侵害に対する反撃行為を行った行為者の主観面は、
			現実に存在する侵害に対する過剰防衛の場合と同じであり、
			したがって、責任の限度もそれと同じだから、課すことができる刑は行為者の責任を限度とする以上、
			刑法36条2項を適用することができる。
			
			急迫不正の侵害を誤想したことについて過失がある場合は、
			生じた結果について過失犯が成立しているから、36条2項によって刑の免除は認められない。
			
	\section{事実の錯誤}
		\subsection{具体的事実の錯誤}
			故意とは、構成要件該当事実の認識及び認容をいうから、故意の認識対象として重要な事実は、
			特定の構成要件に該当するという事実であって、当該事実について認識・認容していれば故意が認められ、
			当該事実の具体手製についての錯誤は故意を阻却しないというべきである(抽象的法定符合説)。
		
	
		
		
	

\part{各論}
	\section{住居侵入罪(刑法130条)}
		\subsection{「侵入」の意義}
			「侵入」とは管理権者の意思に反する立ち入りをいう。
		
	\section{窃盗罪(刑法235条)}
		\subsection{他人の財物}
			「他人の財物」とは、「窃取」の客体であることから、
			他人の占有する他人の所有物を意味する。
			
		\subsection{窃取した}
			「窃取した」とは、他人が占有している財物を、
			占有者の意思に反して占有を侵害し、
			自己又は第三者の占有に移転させることをいう。
			
			\subsubsection{占有}
				占有とは財物に対する事実的支配を意味し、その存否の判断は、
				客観的な支配の事実と支配の意思を考慮して、社会通念に従って行う。
			
			\subsubsection{死者の占有}
				死者には占有の事実も占有の意思も観念できない以上、占有が認められないことから問題となる。
			
				占有	が認められなければ、当該財物は窃取の客体となりえないから、窃盗罪は成立しない。
				しかし、被害者を殺害し占有を失わせる行為が、窃盗罪で処罰されるべき後行行為の罪責を軽くする方向で検討させることは妥当でない。
				そこで殺害行為(占有離脱行為)と取得行為の一体性が認められる場合には、
				当該行為者との関係においては、なお生前の占有が存続しているものとして罪責評価を行うべきである。
		\subsection{不法領得の意思}
			不法領得の意思とは、
			\MARU{1}権利者を排除して、他人の物を自己の所有物と同様に(権利者排除意思)、
			\MARU{2}その経済的用法に従いこれを利用し処分する意思(利用処分意思)をいう。
			
			権利者排除意思は、軽微な一時使用と窃盗罪を区別しながら、
			犯罪の成否を占有取得時に確定させ、既遂時期の明確性を担保するために要求される。
			
			利用処分意思は、窃盗罪等の領得罪を毀棄罪から区別するために要求される。
			財物の効用を取得する意思は、犯行の強力な動機になることに加え、そうした動機で行われる行為については非難可能性が高まることから、
			毀棄罪よりも重い処罰を基礎づけることができる。
			もっとも、厳密な経済的用法に従う必要はなく、当該財物から直接的に何らかの効用を享受する意思があれば足りる。
			
	\section{詐欺罪(刑法246条)}
		\subsection{欺罔行為}
			欺罔行為とは、\MARU{1}交付行為に向けて、
			\MARU{2}交付の判断の基礎となる重要な事項についての錯誤を惹起する行為をいう。
			
			詐欺罪において財産は交換手段・目的達成手段として保護されているから、
			重要な事項とは、交付によって達成される目的にとって重要な事項をいう
		
			
	
	\section{放火罪}
		\subsection{「焼損」の意義}
			「焼損」とは、火が媒介物を離れて目的物が独立に燃焼を継続しうる状態になったことをいう。
			
		\subsection{刑法110条における「公共の危険」}
			「公共の危険」とは、不特定多数の人の生命、身体又は財産に対する危険をいう。
			「公共の危険」について、刑法108条、刑法109条1項に規定する建造物等に延焼する危険に限定すべきとの見解もあるが、
			公共の危険の発生は建造物等への延焼以外からも生ずる以上、
			公共の危険を限定する合理的根拠に欠ける。
			
		\subsection{刑法110条における「公共の危険」の認識}
			刑法110条1項は、「よって」という文言を使っているから、結果的加重犯であると解される。
			したがって、行為者が加重結果である「公共の危険」の発生について認識していなくても、
			他の犯罪事実について認識・認容が認められれば、故意を認めることができる。
		
		
	\section{文書偽造罪}
		\subsection{「偽造」の意義}
			『偽造」とは、名義人と作成者との人格の同一性を偽ることをいう。
			名義人とは、文書に表示された意思・観念の帰属主体として文書の形式・内容から認識される者をいい、
			作成者とは文書に表示されて意思・観念の帰属主体をいう。
			
	
			
	\section{公務執行妨害罪(刑法95条)}
		\subsection{職務の適法性}
			「職務」とは、公務員の行う事務の全てをいう。
			そして公務執行妨害罪の保護法益は公務員による職務の円滑な執行であって、
			違法な職務は保護に値しないと考えられるから、
			刑法95条にいう「職務」とは、適法な職務をいうと解される。
			
			職務の適法性は、\MARU{1}当該職務が当該公務員の抽象的権限の範囲に属すること、
			\MARU{2}当該公務員が当該職務を行う具体的権限を有していること、
			\MARU{3}法律上の重要な方式を履践していること、
			という要件の下、行為時を基準に客観的に判断される。
		
	
	
		
		
		
	






	
\end{document}








