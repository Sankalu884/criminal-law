\documentclass[fontsize=11pt,
jlreq_notes
]{jlreq}

% 1. フォント・日本語処理
\usepackage{pxchfon} 
\usepackage{otf}

% 2. フォント設定
\setminchofont{A-OTF-RyuminPro-Light.otf}
\setgothicfont{A-OTF-FutoGoB101Pr6N-Bold.otf}

% 3. 汎用パッケージ
\usepackage{fancybox}

% 4. 自作パッケージ
\usepackage[sect]{kian}



\title{\vspace{-30mm}{\textgt{\Large{\fbox{25} 報復と仲間割れ}}}}
\date{\vspace{-15mm}}


\begin{document}

\maketitle

\sectionA{甲乙がE公園でDに暴行を加え、傷害を負わせた行為(第1暴行)の罪責}
	甲、乙の上記行為に傷害罪の共同正犯(刑法204条、60条)が成立するか。
	
	甲、乙、はワゴン車の中で、Dに暴力を振るって制裁を加えた上、慰謝料を支払わせる旨の共謀を行っている。
	その後、甲、乙は、こもごもDに暴行を加え、少なくとも\UTF{2461}の傷害をおわせているが、これはDに制裁を加え、慰謝料を払わせるための暴行と評価できるから、
	甲、乙は、傷害罪の構成要件に該当する行為を共同して行ったといえる。

したがって、傷害罪の共同正犯が成立する。
	
\sectionA{第2暴行にかかる甲、乙の罪責}
	\sectionB{甲の罪責}
		第2暴行について、暴行罪の共同正犯(刑法208条、60条)が成立するか。
		\UTF{2460}の傷害がどのように生じたかが明らかでないから、第2暴行のみにより生じたものとして検討を行う。
		A男が顔面を殴打するなどした行為は、Dの身体に対する有形力の行使として、暴行罪(刑法208条)の構成要件に該当する。
		そして、第2暴行についても、Dに制裁を加えるという共謀の危険が現実化したものといえる。
		しかし、甲が第2暴行の前に「俺帰る」と一言述べて犯行現場から離脱しているため、
		甲の離脱により、共同正犯関係が解消され、第2暴行について甲は罪責を負わないのではないかが問題となる。
		
		\sectionC{}
			共同正犯が成立し、一部実行全部責任という法的効果が生じるためには、構成要件該当事実を共同して惹起することが必要である。
			したがって、一部の者に離脱により共同正犯関係の解消が認められるのは、離脱者が作出した因果的影響力が遮断されたと評価できる場合である。
			
			具体的には、離脱者において自己の作出した心理的・物理的因果性の両者を切断させるに足る防止措置を講じたこと、
			残余者側において、離脱者との共謀によって作出された心理的・物理的因果性を利用して犯行を継続する危険が消滅したことが必要である。
			
		\sectionC{}
			本問では、共謀に基づいて第1暴行が実行されており、Dに対する暴行の犯意は強化されている。
			また、客観的にもDは暴行によって抵抗が困難になっているから、甲の意思とは無関係に既遂結果に対する影響力を持ちうる。
			
			したがって、説得によってその後の実行の継続を中止させたり、警察への通報によって実行を阻止するなどの措置をとらずに、
			単に「俺帰る」と一言述べ、犯行現場から離脱するだけでは、共同正犯関係の解消を認めることはできない。
			
			以上より、甲は離脱後の第2暴行についても責任を負うから、\UTF{2460}の傷害がどのように生じたとしても、一連の暴行につき、傷害罪の共同正犯(刑法204条、60条)が成立する。
		
		\sectionB{乙の罪責}
			第2暴行について、暴行罪の共同正犯(刑法208条、60条)が成立するか。
			乙は第2暴行の前にA男に顔面を殴打されて失神しているから、共同正犯関係が解消されるのではないか。甲と同様に検討する。
			
			確かに乙は、DをA男から離して、大丈夫かという趣旨の問いかけをしており、一時的に犯行の継続を中止させているといえる。
			しかし、その後A男によって失神させられており、犯行を阻止できておらず、自らの及ぼした因果性を完全に遮断したとは評価できない。
			
			したがって、共同正犯関係の解消は認められず、一連の暴行につき、傷害罪の共同正犯(刑法204条、60条)が成立する。
			
\sectionA{DにC子に対する慰謝料として10万円の支払いをさせた行為の罪責}
	\sectionB{}
		上記行為に恐喝罪の共同正犯(刑法249条1項、60条)が成立するか。
		恐喝罪は、暴行・脅迫を手段として相手方の意思に基づいて財物を交付させる犯罪であるから、暴行・脅迫は、財物の交付に向けられた、相手方の犯行を抑圧する程度に至らないものである必要がある。
		
		\sectionC{}本問において、Aは1人で暴行を行い、DはAの申出を承諾して支払を行っているから、Aの暴行は相手方の犯行を抑圧する程度に至らないものといえる。
		
		\sectionC{}
		では、財物の交付に向けられたものといえるか。
		ここで、CはDに対して慰謝料を請求する権利を有していると考えられるから、Dに財産的損害は認められず、恐喝罪は否定されるとも思える。
		
		しかし、自力救済が否定され、金銭債権の実現も法的手段によるべきであるのが原則とされる以上、そのことの反射的利益として、法的手段によらなければ交付しない利益を肯定することができる。
		
		したがって、甲の暴行は財物の交付に向けられたものといえるから、恐喝罪における暴行が認められる。DはAの恐喝に畏怖し、この畏怖に基づいて10万円を支払っている。A男には故意と不法領得の意思も認められるので、甲、乙、A男は、恐喝罪(刑法249条1項)の構成要件に該当する行為を共同して実行したといえる。
		
	\sectionB{}
		もっとも、A男の行為は債権の行使であるから、違法性が阻却されないか。
		
		権利行使として恐喝罪の違法性が阻却されるのは、\UTF{2460}権利の範囲内で、かつ、\UTF{2461}その方法が社会通念上忍容すべきものと認められる場合である。
		
		本問において、A男が請求した30万円は、裁判等によって確定した債権額ではないから、権利の範囲内か否かを判断できない。仮に、慰謝料請求権の範囲内であるとしても、慰謝料を請求するのに暴行を用いる必要性は認められないし、顔面を手拳で殴打するという比較的強度の高い暴行を加えることに相当性も認められない。
		
		したがって、違法性阻却は認められないから、恐喝罪の共同正犯(刑法249条1項、60条)が成立する。
		
\sectionA{罪数}
	以上により、甲、乙には\UTF{2460}第1暴行につき傷害罪の共同正犯(刑法204条、60条)、\UTF{2461}第2暴行につき傷害罪の共同正犯、\UTF{2462}恐喝罪の共同正犯(刑法249条1項、60条)が成立する。
	
	\UTF{2460}と\UTF{2461}は同一の被害者に対する、時間的場所的近接性の認められる行為であるから、包括一罪となる。これと\UTF{2462}は併合罪(刑法45条前段)となる。
		
		
		
	






\raggedleft{以上}
	
\end{document}








