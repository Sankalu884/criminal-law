\documentclass[fontsize=11pt,
jlreq_notes,
sidenote_length=10zw
]{jlreq}

% 1. フォント・日本語処理
\usepackage{pxchfon} 
\usepackage{otf}

% 2. フォント設定
\setminchofont{A-OTF-RyuminPro-Light.otf}
\setgothicfont{A-OTF-FutoGoB101Pr6N-Bold.otf}

% 3. 汎用パッケージ
\usepackage{fancybox}

% 4. 自作パッケージ
\usepackage[sect]{kian}


\title{\vspace{-30mm}{\textgt{\Large{\fbox{31} 招かれざる客}}}}
\date{\vspace{-15mm}}


\begin{document}

\maketitle

\sectionA{甲の罪責}
	\sectionB{昏睡強盗未遂罪の共同正犯の成否}
		甲とAが、昏睡強盗を行うために、Bに一気飲みをさせてBを酔わせるように仕向けたり、ビールグラスに睡眠薬を入れたりした行為につき、昏睡強盗未遂罪(刑法243条、239条)の共同正犯(刑法60条)の成否を検討する。
	
		共同正犯の成立要件は、\ajMaru{1}共謀と、\ajMaru{2}共謀に基づく実行である。甲とAは、Bに睡眠薬を飲ませて眠らせ、金品を盗取する旨の共謀を行い、当該共謀のもとでBに酒や睡眠薬を飲ませて、Bの意識を朦朧とさせているから、共謀に基づく昏睡強盗罪(刑法239条)の実行の着手が認められる。もっとも、この状態を利用して財物を盗取していないので既遂犯は成立せず、昏睡強盗未遂罪の共同正犯が成立するにとどまる。
		
	\sectionB{強盗致傷罪の共同正犯の成否}
		甲が、AがBに暴行を加えて気絶させた後、金品を奪った行為につき、強盗致傷罪(刑法240条)の共同正犯(刑法60条)の成否を検討する。
		
		Aは、Bの顔面を手拳で数回殴打し、更に1回足蹴にして気絶させた上でBのバッグの中から財物である現金およびネックレスなどを奪っており、故意と不法領得の意思も認められるから、強盗罪(刑法236条1項)が成立する。また、これによりBは頭部顔面外傷の傷害を負っているから、Aには、強盗致傷罪(刑法240条)が成立する。
		
		\sectionC{}
			もっとも、甲とAの当初の共謀内容は昏睡強盗であったため、甲が強盗致傷罪の共同正犯の罪責を負うか、すなわち、Aの強盗致傷が、当初の共謀に基づいて行われたといえるか(共謀の射程は及ぶか)が問題となる。
		
			共犯の処罰根拠は因果性に求められらるから、共謀に基づく実行と評価するためには、共謀と結果との間に因果性が認められること、すなわち、共謀の危険の現実化として結果が発生したといえる関係が要求される。具体的には、共謀の内容と実際に行なわれた行為を比較し、危険の現実化を否定するような重大な不一致があるかどうかを検討すべきである。
			
		ここで、\ajMaru{1}昏睡強盗と強盗は、被害者の犯行を困難にして財物を奪取する点で共通すること、\ajMaru{2}被害者が昏睡しないことから、昏睡強盗か強盗に変化させて財物を奪取することは、十分考えられる事態であること、\ajMaru{3}犯行の日時、場所、被害者、動機は全く同一であることから、Aの行為と当初の共謀内容の間に危険の現実化を否定するような重大な不一致は存在せず、共謀の射程は及ぶとも思える。
		
			しかし、\ajMaru{4}昏睡強盗と強盗は暴行・脅迫の有無という点で罪質を異にすること、\ajMaru{5}甲とAの当初の話し合いでは、「怪我させたりはしないでおこう」と、暴行を加えることを明示的に共謀の範囲から除外していた。したがってAの行為と当初の共謀内容との間には危険の現実化を否定するような重大な不一致が存在し、Aの行為はもっぱらAの個人的な判断に基づくものであって、甲との共謀に基づく意思決定とはいえない\sidenote{
				「呆然と事態を見守っていた」ことから、甲はAが暴行に及ぶことを全く認識してなかったと考えられる、という事情をどこで使えばよいか。
				
				共謀の射程の問題が、共謀が結果に及ぼした影響(因果性)の問題であると考えると、Aの行為時の甲の認識は問題とならないように思った。
			}から、共謀の射程は及ばないと解するべきである。
			
		\sectionC{}
			そうすると、強盗罪はAの単独犯として開始されたことになるから、甲は、Aが単独でBの犯行を抑圧した後に甲と共謀して財物を奪取したとして、強盗の承継的共同正犯の成否が問題となる。
			
			上述の通り、共同正犯の処罰根拠は結果に対する因果性にあるので、承継的共同正犯は結果に対する因果性が認められる限度で肯定される。
			
			では、因果性を及ぼすべき「結果」とは何か。強盗罪は財産犯であるから、第一次的な保護法益は財産権であって、財物の占有侵害という結果に因果性を有していれば承継を認めることができるとの見解もある。しかし、強盗罪の法益侵害性は財産権侵害に尽きるわけではなく、暴行・強迫による身体の安全や意思決定の自由への侵害が含まれている。強盗罪においては、暴行・強迫による法益侵害性が窃盗罪と比較して法定刑を加重する関係にあり、これらの法益侵害も強盗罪の結果と解すべきであるから、強盗罪の承継的共同正犯を認めるためには、財物奪取だけではなく、暴行・脅迫にも因果性を及ぼす必要がある。
			
			本問において、甲の財物奪取は、Aの暴行によってBが負傷・気絶したのちに行われており、暴行・脅迫に因果性を有していない以上、強盗罪の承継的共同正犯は認められない。したがって、財物奪取行為につき、窃盗罪の限度で承継的共同正犯となりうる。
			
			では、Aとの関係においてどの範囲で共同正犯が成立するか。共同正犯の本質は、特定の犯罪を共同して実行することに求められるから、異なる構成要件間での共同正犯は成立しない。もっとも、構成要件が同質的で実質的に重なり合う場合には、その範囲で共同正犯を認めることができる(部分的犯罪共同説)。窃盗罪と強盗罪は共に財産犯であり、財物を不法に奪取する点で構成要件の実質的な重なり合いを認めることができる。甲には故意と不法領得の意思も認められるから\sidenote{
				構成要件の実質的重なり合いを認めた後に、別途、新たな共謀の認定を行い、主観的要件の検討を共謀の認定に解消させたほうがよいか。
			}、窃盗罪の範囲で共同正犯が成立する。
			
	\sectionB{}
		以上より、甲には、\ajMaru{1}昏睡強盗未遂罪(刑法243条、239条)の共同正犯(刑法60条)、\ajMaru{2}窃盗罪の共同正犯(刑法60条、刑法235条)が成立する。\ajMaru{1}と\ajMaru{2}は、Bの財産という同一の法益侵害に向けられ、かつ、時間的・場所的にも近接しているといえるので、\ajMaru{2}は重い\ajMaru{1}に吸収される(吸収一罪)。
		
\sectionA{乙の罪責}
	乙が状況を理解しながら、現金数千円を奪った行為につき、強盗罪(刑法236条)の共同正犯(刑法60条)の成否を検討する。共同正犯が成立するためには、結果に対する因果性を及ぼす必要があるが、甲と同様、Aの暴行によってBが負傷・気絶した後に犯行に加わった乙に強盗罪の結果に因果性を及ぼすことはできないから、強盗罪の共同正犯は成立しない。乙には、構成要件の実質的な重なり合いが認められる窃盗罪につき共同正犯が成立すにとどまる\sidenote{
		共謀と主観的要件につき、上と同じ。
	}。







\raggedleft{以上}
	
\end{document}








