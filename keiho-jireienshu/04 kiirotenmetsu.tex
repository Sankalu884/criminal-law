\documentclass[11pt]{jsarticle}

\usepackage[sect]{kian}
\usepackage{okumacro}
\usepackage{fancybox}
\usepackage[noalphabet]{pxchfon}  
\setminchofont{A-OTF-RyuminPro-Light.otf}
\setgothicfont{A-OTF-FutoGoB101Pr6N-Bold.otf}



\title{\vspace{-30mm}{\textgt{\Large{\fbox{4} 黄色点滅信号}}}}
\date{\vspace{-15mm}}

\begin{document}
\maketitle
	\sectionB{}
		見通しの悪い交差点に徐行せず、
		時速30ないし40キロメートルで進入して自動車を衝突させ、Bを死亡させた行為につき、
		過失運転致死罪(自動車運転死傷行為処罰法5条)の成否を検討する。

		\sectionC{}
			見通しの悪い交差点に徐行せず漫然と侵入する行為は、交差点に侵入する他の自動車と衝突し、
			死傷結果を発生させる実質的危険性のある行為であり、同罪の実行行為にあたる。
		
		\sectionC{}
			そして、Bは死亡している。
			
		\sectionC{}
			問題は、Bが甲の行為に「よって死傷」したといえるかである。
			
			本問において、甲が交差点にするにあたって、時速10キロメートルないし15キロメートルまで減速し、
			通常事故が発生しない程度にまで結果発生の現実的危険性を減少させたとしても、
			死傷事故が発生した疑いが否定できない。
			
			したがって、甲の行為と結果との間に因果関係が認められない。
	\sectionB{}
		以上より、甲の行為は過失運転致死罪の客観的構成要件に該当せず、不可罰である。
	\begin{flushright}
  		以上	
	\end{flushright}
\end{document}