\documentclass[fontsize=11pt,
jlreq_notes
]{jlreq}

% 1. フォント・日本語処理
\usepackage{pxchfon} 
\usepackage{otf}

% 2. フォント設定
\setminchofont{A-OTF-RyuminPro-Light.otf}
\setgothicfont{A-OTF-FutoGoB101Pr6N-Bold.otf}

% 3. 汎用パッケージ
\usepackage{fancybox}

% 4. 自作パッケージ
\usepackage[sect]{kian}


\title{\vspace{-30mm}{\textgt{\Large{\fbox{30} 暗転した同窓会}}}}
\date{\vspace{-15mm}}


\begin{document}

\maketitle

\sectionA{第1暴行について}
	\sectionB{甲、乙がそれぞれBに暴行を加え、Bを死亡させた行為につき、傷害致死罪(刑法205条)の共同正犯(刑法60条)の成否を検討する。}
		\sectionC{}
			共同正犯の成立要件は、\ajMaru{1}共謀と\ajMaru{2}共謀に基づく実行である。BがAの髪を掴んで乱暴を始めた時点で、Bの暴行を止めてAを守るのために共同で暴行を行う旨の共謀が成立しているといえる。その後、甲がBの顔面を殴打したことにより、Bは頭蓋骨骨折の傷害を負い、これによって死亡しているが、これはBの暴行を止めてAを守るための暴行であったといえるから、甲、乙は、共謀に基づいて暴行罪の結果的加重犯として傷害致死罪の構成要件に該当する行為を共同して行ったといえる。
		\sectionC{}
			では、甲、乙の行為に正当防衛(刑法36条1項)が成立するか。正当防衛が成立するためには、\ajMaru{1}急迫不正の侵害に対する、\ajMaru{2}防衛するための、\ajMaru{3}やむを得ずにした行為と認められることが必要である。
			\sectionD{}
				BのAに対する暴行は、不正の侵害であり、その後も継続することが予想される状況であったから、急迫不正の侵害であると認められる。
			\sectionD{}
				では、「自己または他人の権利を防衛するため」の行為(防衛行為)であったといえるか。
				
				防衛行為であるといえるためには、「防衛するため」という文言から、防衛の意思が必要である。防衛の意思とは、急迫不正の侵害を認識しつつこれを避けようとする単純な心理状態をいう。防衛の意思は攻撃の意思が併存していることを以て否定されるものではないが、専ら攻撃の意思で反撃を加える行為は、防衛の意思が認められず、正当防衛が否定される。
				
				防衛の意思は主観的な要件であるから、その有無は甲と乙それぞれで判断される必要がある。
				
				甲は、憤激の意思を有しながら反撃行為を行っているが、反撃行為は「何とかBの手を離させようとして」行われたものであり、もっぱら攻撃の意思で行われたとは評価できない。
				
				乙は、BがAの髪を掴んだ状態であることを認識して反撃行為に及んでいるから、防衛の意思に基づく反撃行為であることに問題はない。
				
				したがって、甲、乙の行為は防衛行為であったといえる。
			
			\sectionD{}
				次に、甲、乙の行為は、「やむを得ずにした行為」といえるか。
				
				やむを得ずにした行為とは、反撃行為が防衛手段として必要最小限度のものであること、すなわち、反撃行為が防衛手段として相当性を有するものであることをいう。必要最小限度か否かは、可能な防衛手段の選択肢及び態様を考慮した上で具体的に判断する。
				
				甲と乙は、2対1でBに対して暴行を加えており、その態様も人体の枢要部である顔面や腹部を殴るというものであって、死亡結果発生の危険性も認められる。また、殴るような侵害性の高い反撃行為に出なくとも、2人でBの身体を押さえつけてAから引き剥がすといった、より侵害性の低い他の手段も考えらえれるから、相当性は認められない。
				
	\sectionB{}
		以上により、正当防衛は成立せず、傷害致死罪(刑法205条)の共同正犯(刑法60条)が成立する。なお甲らの行為は、防衛の程度を超えた行為なので過剰防衛として任意的に刑の減免がされる(刑法36条2項)。
				
				
\sectionA{第2暴行について}
	\sectionB{}
		乙がBが倒れて動かなくなった後も続けて暴行を加えた行為につき、暴行罪(刑法208条)が成立するか。乙の行為は暴行罪の構成要件に該当するが、正当防衛として違法性が阻却されるか。
		
		\sectionC{}
			第1暴行の後、Bは意識を失ったように動かなくなって仰向けに倒れており、Bによる不正の侵害は終了していたから、正当防衛は成立しない。では、第1暴行の時点で存在していた急迫不正の侵害に対して反撃を行っているうちに、その反撃が量的に過剰になったとして過剰防衛(刑法36条2項)が成立するかが問題となる。
			
			第1暴行により、Bは動かなくなっており、甲は現場から立ち去ろうとしている。一方、乙はこれを十分に認識していたにもかかわらず、激怒して新たに攻撃を加えており、乙の行為を、反撃が継続するうちに過剰になったと評価することはできない。
			
			したがって、過剰防衛は成立しない。
			
			以上より、乙には、暴行罪が成立する。
			
		\sectionC{}
			なお、甲と乙の間には、乱暴されたAを守るという、正当防衛の限度で共謀が成立したに過ぎず、乙の第二暴行は、共謀に基づくものとはいえない(共謀の射程外)から、第2暴行につき新たな共謀が成立していたとはいえない以上、第2暴行につき、甲は罪責を負わない。
			
\sectionA{罪数}
	以上より、甲には、\ajMaru{1}傷害致死罪(刑法205条)の共同正犯(刑法60条)、乙には、\ajMaru{2}傷害致死罪(刑法205条)の共同正犯(刑法60条)および\ajMaru{3}暴行罪(刑法208条)が成立する。\ajMaru{2}と\ajMaru{3}は併合罪(刑法45条前段)となる。
	





\raggedleft{以上}
	
\end{document}








