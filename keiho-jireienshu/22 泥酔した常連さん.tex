\documentclass[11pt]{jsarticle}

\usepackage[sect]{kian}
\usepackage{otf}
\usepackage{fancybox}
\usepackage{ascmac}
\usepackage[noalphabet]{pxchfon}  
\setminchofont{A-OTF-RyuminPro-Light.otf}
\setgothicfont{A-OTF-FutoGoB101Pr6N-Bold.otf}



\title{\vspace{-30mm}{\textgt{\Large{\fbox{22} 泥酔した常連さん}}}}
\date{\vspace{-15mm}}


\begin{document}

\maketitle

\sectionB{甲が酒に酔った状態で自動車を運転して、Dを死亡させた行為につき、危険運転致死罪(自動車運転死傷行為処罰法2条1号)の成否を検討する。}

	\sectionC{}
		甲は、酒に酔ってハンドル操作や前方注視が困難な状態に陥っていたから、「アルコールの影響により正常な運転か困難な状態で自動車を走行させて」、事故を起こし、「よって」Dを死亡させたといえる。
		
		では甲に故意は認められるか。故意とは犯罪事実の認識・認容をいう。甲は、酒の影響でハンドル操作や前方注視が困難な状況に陥っていることを認識していた。正常な運転が困難な状態で自動車を走行させれば、事故が発生する蓋然性は高いといえるから、甲は事故が発生する蓋然性を認識し、それにもかかわらず、早く家に帰りたいなどと考えて運転を継続することにより、これを認容していたといえる。
		
		したがって、甲には、危険運転致死罪(自動車運転死傷行為処罰法2条1号)が成立する。
		ただし、行為時において行動制御能力が著しく減弱していた可能性があるため、刑法39条2項により、刑の必要的減軽を受ける。
		
	\sectionC{}
		もっとも、飲酒行為(原因行為)の時点では完全な責任能力があったにもかかわらず、運転行為(結果行為)に
		39条2項を一律に適用して減軽を行うことは不当であるとも思われる。そこで、原因行為時には責任能力があったことに着目して、行為者の可罰性を肯定できないか。
		いわゆる原因において自由な行為の理論の適用の可否が問題となる。
		
		\sectionD{}ここで、責任とは犯罪行為に対する非難可能性である以上、責任能力は実行行為時に存在することが必要である(実行行為と責任能力の同時存在の原則)。
		しかし、原因行為に実行行為性を肯定できる場合には、責任能力が問題となる時点を原因行為時に求めることができるため、当該時点で責任能力が備わっていれば、完全な責任を問うことが可能である。

		行為者は、原因行為から結果行為を経て発生した結果について帰責されるから、原因行為には心神耗弱状態における結果行為を誘発して結果に至る危険性が認められなければならない。具体的には、\UTF{2460}原因行為時の意思決定によって結果行為が支配されている場合か、\UTF{2461}原因行為に結果行為を誘発する顕著な傾向がある場合には、そのような危険性を認めることができる。
		
	\sectionD{}本問において、甲は酒を飲んだ状態で自動車を運転して帰宅することを前提にスナックに出かけており、原因行為時と結果行為時の意思の連続性が認められる。
		したがって、飲酒時の意思決定に支配されて運転を行ったといえるから、飲酒行為に実行行為性を認めることができる。

	\sectionD{}では、故意は認められるか。故意が認められるためには、結果発生の認識に加えて、実行行為性の認識も必要である。甲は、一定の酒量を超えると酩酊の度合いが急に深まって運転に支障を来すことは認識しており、Aからも指摘を受けていた。したがって、飲酒によって正常な運転が困難な状態に陥り、事故を発生させる危険性を認識していたといえる。それにもかかわらず、飲酒を続けることにより、これを認容していたといえる。
	
\sectionB{}
	以上により、甲には危険運転致死罪が成立し、原因において自由な行為の理論により完全な責任を問うことができるから、刑法39条2項による刑の必要的減軽を受けることはない。
\begin{flushright}
	以上
\end{flushright}
	
\end{document}








