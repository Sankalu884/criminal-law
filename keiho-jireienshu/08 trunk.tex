\documentclass[11pt]{jsarticle}
\usepackage[sect]{kian}
\usepackage{okumacro}
\usepackage{fancybox}
\usepackage[noalphabet]{pxchfon}  
\setminchofont{A-OTF-RyuminPro-Light.otf}
\setgothicfont{A-OTF-FutoGoB101Pr6N-Bold.otf}
\title{\vspace{-30mm}{\textgt{\Large{\fbox{8} トランク監禁の悲劇}}}}
\date{\vspace{-15mm}}

\begin{document}
\maketitle

\sectionA{甲がAの顔面を手拳で数回殴打した行為について}	
	甲がAの顔面を殴打する行為は、Aの身体に対する有形力の行使として暴行罪(刑法208条)が成立する。
	\sectionA{甲と乙が一緒にAを捕まえてトランク内に押し込み、事故により死亡させた行為について監禁致死罪の共同正犯(刑法60条、刑法221条)の成否を検討する。}	\sectionB{監禁致死罪の成否}
			\sectionC{}
				甲と乙は、Aをトランクに押し込むことで脱出を困難にしており、これは「監禁」にあたる。
			\sectionC{}
				Aは、脳損傷により脳死状態に陥っており、3月11日に脳死が確定した。ここで、脳死の状態に至れば、脳機能は不可逆的に停止し、蘇生は不可能となるから、脳死を刑法上の人の終期と捉えることができる。
				
				したがって、Aは遅くとも、3月11日午後7時35分に死亡したといえる。
			\sectionC{}
				では、Aは監禁に「よって」死亡したといえるか。監禁行為と、Aの死亡との間に条件関係は認められるものの、当該行為と結果との間に、丙の運転する自動車が衝突するという介在事情が存在するため、法的因果関係の存否が問題となる。
				
				法的因果関係は、偶然的結果を排除し適正な帰責範囲を画するために要求される。これは、客観的に存在する全ての事情を判断資料とし、実行行為の危険が結果へと現実化したか否かによって判断される。
				
				本問において、Aの直接の死因を形成したのは、丙の過失による自動車の衝突であって、介在事情の結果への寄与度は大きい。また、甲乙の行為が丙の過失行為を誘発した等の事情も認められない。
				
				しかし、自動車のトランク内は通常人が入ることが想定されていないため、衝突に対して脆弱であって、ひとたび自動車事故が起これば、死亡する危険が非常に高い場所であるといえるから、人をトランク内に監禁して停車する行為は、事故が発生すれば生命侵害の危険を生じさせる特別の危険を有する行為であるといえる。そして、自動車事故はしばしば起こりうることを考えると、停車している自動車に前方不注意により衝突して事故が発生することは異常性が高いとまではいえない。
				
				したがって、人をトランク内に監禁した自動車を、片側1車線の道路に停車させるという行為は、死亡結果が発生する危険のある状況を設定するものであって、Aの死亡結果はこのような危険が介在事情を通じて実現化したものといえるので、甲、乙の監禁行為とAの死亡結果との間には法的因果関係が認められる。
				
				以上より、監禁致死罪(刑法221条)が成立する。なお、後述の通り、甲と乙は共同正犯となる(刑法60条)。
		\sectionB{共同正犯の成否}
			甲と乙は、共同正犯(刑法60条)とならないか。共同正犯の成立要件は\MARU{1}共謀と、\MARU{2}共謀に基づく実行である。
			
			本問において、遅くとも乙がAを捕まえた時点で、監禁につき現場共謀が成立しているといえる。そして、この共謀に基づいて甲と乙は共同してAをトランク内に押し込んでいるから、共同正犯が成立する。
			
			なお、結果的加重犯にも共同正犯が成立するかは問題となりうるが、基本犯につき共同正犯が成立し、加重結果を共同して惹起したと評価できる以上、結果に対する因果性は肯定できるから、共同正犯の成立を認めてよい。
			
			以上より、甲と乙には監禁致死罪の共同正犯(刑法60条、刑法221条)が成立する。
\sectionA{丙の罪責}
	丙が前方不注意により乙の自動車に衝突し、Aを死亡させた行為につき、過失運転致死罪(自動車運転死傷行為処罰法5 条)の成否を検討する。
	\sectionB{}
		前方不注意の状態で自動車を運転する行為は、停車している他の自動車と衝突し、死傷結果を発生させる実質的危険性のある行為であり、過失運転致死罪の実行行為に当たる。
	\sectionB{}
		前述の通り、Aは死亡している。
	\sectionB{}
		では、Aは丙の行為に「よって死傷」したといえるか。
		Aは丙の車が衝突したことにより発生した頭部挫傷の傷害を直接の原因として死亡している。事故が発生したのはほぼ直線の見通しのよい道路であり、前方を注意して運転していれば、停車している乙車を発見し迂回するなどして衝突を避ける措置をとり、結果を回避することが十分可能であったといえる。
		
		したがって、前方不注意の運転とAの死亡結果には因果関係が認められる。
	\sectionB{}
		また、道路に他の車が停車していること、前方不注意により当該車に衝突すれば死傷結果が発生しうることは、十分予見可能であるといえるから、丙にはA死亡の結果を帰責できるだけの過失が認められる。
		
		以上より、丙には過失運転致死罪が成立する。
\sectionA{罪数}
	甲には\MARU{1}暴行罪(刑法208条)、\MARU{2}監禁致死罪の共同正犯(刑法60条、刑法221条)が成立する。
	
	乙には、\MARU{3}監禁致死罪の共同正犯(刑法60条、刑法221条)が成立する。
	
	丙には、\MARU{4}過失運転致死罪(自動車運転死傷行為処罰法5 条)が成立する。
	
	\MARU{1}と\MARU{2}は併合罪(刑法45条前段)となる。


\begin{flushright}	
	以上
\end{flushright}	

\end{document}