\documentclass[11pt]{jsarticle}

\usepackage[sect]{kian}
\usepackage{okumacro}
\usepackage{fancybox}
\usepackage{ascmac}
\usepackage[noalphabet]{pxchfon}  
\setminchofont{A-OTF-RyuminPro-Light.otf}
\setgothicfont{A-OTF-FutoGoB101Pr6N-Bold.otf}



\title{\vspace{-30mm}{\textgt{\Large{\fbox{13} 一線を越えた男友達}}}}
\date{\vspace{-15mm}}


\begin{document}

\maketitle

\sectionA{クレジットカードを用いた決済につき詐欺罪の成否}
	甲がA名義のクレジットカードを用いて買い物を行った行為につき詐欺罪(刑法246条1項)の成否を検討する。
	
	詐欺罪が成立するためには、「人を欺いて」錯誤を生じさせ、その錯誤に基づいて「財物」を「交付させ」ることが必要である。
	
	では、Aの行為は「人を欺」く行為、すなわち欺罔行為といえるか。
	欺罔行為とは、\MARU{1}交付行為に向けて、\MARU{2}交付の判断の基礎となる重要な事項についての錯誤を惹起する行為をいう。
		
	甲は商品の交付に向けて、カードの名義人と同一性を偽っている。
	また詐欺罪において財産は交換手段・目的達成手段として保護されているから、
	重要な事項とは、交付によって達成される目的にとって重要な事項をいう。
		
	ここで、クレジットカードを用いた決済のシステムは名義人に対する個別的信用を基礎としているから、
	加盟店が確認を義務付けられる利用者と名義人の同一性についての偽りは、代金決済についての偽りがなくとも、
	重要な事項についての偽りといえる。
		
	そして甲が買い物を行ったクレジットカードの加盟店は、甲の欺罔行為によって錯誤に陥り、
	商品を交付し、甲はこれを受領している。
		
	甲には故意と不法領得の意思も認められるから、詐欺罪(刑法246条1項)が成立する。
	
	\sectionA{有印私文書偽造罪・偽造有印私文書行使罪の成否}
	甲がクレジットカードの利用に際して、売上票用紙にAの名前を記入し提出した行為につき、
	有印私文書偽造罪(刑法159条1項)と同行使罪(刑法161条1項)の成否を検討する。
	
	\sectionB{有印私文書偽造罪の成否}
	
		\sectionC{}
			クレジットカード売上票用紙は、「権利義務……に関する文書」である。
	
		\sectionC{}
			「偽造」とは、名義人と作成者との人格の同一性を偽ることをいう。
			名義人とは、文書に表示された意思・観念の帰属主体として文書の形式・内容から認識される者をいい、
			作成者とは文書に表示された意思・観念の帰属主体をいう。本問において、名義人はAである。
		
			名義人の承諾を得て文書を作成した場合、当該文書は名義人の意思に基づくと解されるから、
			名義人と作成者の人格の不一致は生じず、原則として文書偽造罪は成立しない。
		
			しかし、クレジットカードシステムはカードの名義人に対する個別的信用を基礎としており、
			クレジットカード売上票用紙は、そこに記載された者がカード決済を行える信用を有することを確認し、証明する文書である。
			したがって、その文書の性質上、そのような信用を有しない者には当該文書の意思・観念を帰属させることはできない。
		
			以上より、甲がクレジットカード売上票用紙にAの署名をする行為は、
			Aの承諾を得ていても、作成したクレジットカード売上票用紙にAの意思・観念を帰属させることはできないから、
			「偽造」にあたる。
		
		\sectionC{}
			甲は名義人であるAの「署名を使用し」た。
	
		\sectionC{}
			甲には、偽造したクレジットカード売上票用紙を真正な文書として使用するという「行使の目的」が認められる。
			また、これらの犯罪事実の認識・認容があるから、故意も認められる。
		
			以上より、甲には有印私文書偽造罪(刑法159条1項)が成立する。
	\sectionB{偽造私文書行使罪の成否}
		甲は、クレジットカード売上票用紙にAの署名をしてこれを加盟店に提出したので、偽造有印私文書行使罪(刑法161条1項)が成立する。


\sectionA{クレジットカードを用いたキャッシングにつき窃盗罪の成否}
	甲がA名義のクレジットカードを用いてキャッシングを行った行為につき窃盗罪(刑法235条)の成否を検討する。
	
	窃盗罪は、「他人の財物」を「窃取した」ときに成立する。
	
	「他人の財物」とは、「窃取」の客体であることから、他人の占有する他人の所有物を意味する。
	キャッシングにより引き出された現金はキャッシング機の管理者の占有が及ぶから、「他人の財物」である。
	
	クレジットカードによるキャッシングは決済と同様、カードの名義人に対する個別的信用を基礎としているから、
	名義人でない甲が現金を引き出す行為は、管理者の意思に反する占有の移転と評価できる。したがって「窃取した」といえる。
	
	甲には故意と不法領得の意思も認められるから、窃盗罪(刑法235条)が成立する。
	
\sectionA{クレジットカードの利用につきAに対する財産犯の成否}
	甲は、Aから承諾を得ていた10万円の限度を超えてクレジットカードを利用している。これらの行為につきAに対する犯罪は成立しないか。
	
	犯罪の客体は「10万円の限度でA名義のクレジットカードを利用する地位」という財産上の利益であるから、背任罪(刑法247条)の成否を検討する。
	
	クレジットカードの利用は、名義人であるAに固有の事務であるから、Aから承諾を得た甲は、「他人のためにその事務を処理する者」にあたる。
	
	甲は月に10万円というAの承諾の限度を超えてカードを使用しているから、「任務に背く行為」をしたといえる。
	
	背任罪で要求される図利加害目的は、本人図利目的が存在しないことを背任罪の成立要件とし、それを裏側から規定したものであると解される。
	甲には本人であるAの利益を図る目的はないから、図利加害目的が認められる。
	
	甲には故意も認められるから、背任罪(刑法247条)が成立する。
	
				
\sectionA{強盗殺人罪の成否}
	甲がAの腹部を刺して失血死させた行為について強盗殺人罪(刑法240条後段、刑法236条2項)の成否を検討する。

	\sectionB{2項強盗罪の成否}
		強盗殺人罪は「強盗(犯人)」を主体とする犯罪であるから、強盗殺人罪の成否を論じる前提として、強盗罪(刑法236条2項)の成否を検討する。
		
		2項強盗罪は、暴行・脅迫を用いて「財産上不法の利益」を得ることによって成立する。
		
		\sectionC{}
			強盗罪における暴行・強迫は、財物・財産上の利益奪取の手段として行われるものであるから、
			財物・財産上の利益奪取に向けられた、被害者の犯行を抑圧するに足りる程度のものでなければならない。
			
			甲は、刃渡り20cmという殺傷能力の極めて高い武器を使用してAを殺害しており、
			甲の暴行は反抗を抑圧するに足りる程度のものであったといえる。
			
			後述の通り、債務の免脱も「財産上不法の利益」にあたるから、これに向けられた甲の暴行は2項強盗にいう「暴行」にあたる。
		
		\sectionC{}
			2項強盗罪の客体である「財産上不法の利益」を安易に認めると、
			処罰範囲が不当に拡大することとなる。
			そこで、2項強盗罪の成立を認めるためには、財産上の利益の移転が現実的かつ具体的なものである必要があると解される。
			
			本問において、甲はAのクレジットカードを使用して買い物やキャッシングをしており、外形的にはA本人の使用とされているから、
			債権の存在の記録がない。したがって、Aの相続人が甲に対して債権を行使することは事実上極めて困難であり、
			Aの殺害によって、事実上、債務の支払を免脱した状態になっている。
			
			したがって、財産上の利益の現実的かつ具体的な移転が認められる。
			
		\sectionC{}
			甲には故意と不法領得の意思も認められるから、強盗罪(刑法236条2項)が成立する。
			
	\sectionB{強盗殺人罪の成否}
		\sectionC{}
			甲には後述の通り殺意が認められるが、刑法240条は結果的加重犯としての規定形式を有するため、
			強盗犯人が殺意を持って被害者を殺害した場合(強盗殺人)であっても、刑法240条は適用されるかが問題となる。
			
			ここで、仮に殺人罪と強盗致死罪の観念的競合を認めると、死亡結果を二重に評価していることになる上、
			強盗罪と殺人罪の観念的競合とした場合には、殺人の故意のある方が、
			故意がない場合(強盗致死罪)よりも刑の下限が軽くなるという不均衡が生じるから、
			強盗殺人にも同条は適用されると解すべきである。
		
		\sectionC{}
			では、甲に殺人の故意は認められるか。故意とは、犯罪事実の認識・認容をいう。
			
			甲は刃渡り20cmという殺傷能力の高い武器を使用して、人体の枢要部である腹部を突き刺しているが、
			このような行為は死亡結果を発生させる危険性の極めて高い行為であり、現にAは死亡している。
			甲はこれらの事実を認識しているから、殺人の認識があったといえる。
			
			次に、甲はAを刺した後、救急車を要請するなどの救命措置をとることなく、現場を立ち去ろうとしているから、
			Aの死亡を認容しているといえる。
			
			したがって、甲には殺人の故意が認められる。
			
			以上より、甲には強盗殺人罪(刑法240条後段、刑法236条2項)が成立する。
			
\sectionA{Aの死後、キャッシュカードを持ち去った行為につき窃盗罪の成否}
	甲がAを殺害した後、Aのハンドバッグを見つけてキャッシュカードを取得する意思を生じ、
	これを持ち去った行為につき、窃盗罪(刑法235条)の成否を検討する。
	
	\sectionB{}
		Aのキャッシュカードは「他人の財物」といえるか。
		キャッシュカードはそれにより預金を引き出すことができるという財産的価値があるから、財物性は肯定できる。
		
		問題は、窃盗罪の客体である「他人の財物」は、「窃取」の客体であるから、
		他人の占有する他人の所有物を意味するが、
		既に死亡しているAには占有の意思も占有の事実も観念できないため占有が認められない点にある。
		
		しかし、被害者を殺害して占有を喪失させる行為が、窃盗罪で処罰されるべき後行行為の罪責を軽くする方向で考慮されることは妥当でない。
		そこで、殺害行為と財物取得行為の一体性が肯定できる場合には、行為者との関係では、
		なお被害者の生前の占有が存続しているとして犯罪の成否を検討すべきである。
		
		本問において、甲の殺害行為とキャッシュカードの領得行為は時間的・場所的に近接した行なわれている。
		また、Aが死亡している状態を利用する意思でキャッシュカードを持ち去っており、主観的な連続性も認められる。
		したがって、両行為には一体性が認められるから、甲の罪責評価との関係では、Aの占有が存続しているものとして扱われる。
		
		以上より、キャッシュカード「他人の財物」にあたる。
		
	\sectionB{}
		甲はAの意思に反してキャッシュカードの占有を自身の下に移しているから、「窃取した」にあたる。
		
	\sectionB{}
		故意と不法領得の意思も認められるから、甲には窃盗罪(刑法235条)が成立する。

\sectionA{(電子計算機使用詐欺罪・窃盗罪)の未遂犯の成否}
	甲がATMでAの口座から自身の借金返済のため、振込送金を試みた行為および現金を引き出そうとした行為につき
	それぞれ電子計算機使用詐欺罪の未遂犯(刑法250条、刑法246条の2)、窃盗罪の未遂犯(刑法243条、刑法235条)の成否を検討する。
	
	\sectionB{}
		一般にキャッシュカードと暗証番号を取得すれば、振り込みや現金の引き出しを行うことができるから、
		甲の行為は電子計算機使用詐欺罪および窃盗罪の実行行為に該当する。
		問題は、行為時にはAが暗証番号を変更しており、甲が目的を達成することは不可能であったことから、
		不能犯とならないかである。
		
	\sectionB{}
		未遂犯の処罰根拠は、結果発生の具体的危険に求められる。
		そこで、未遂犯と不能犯の区別に当たっては、当該危険性の有無を判断することになる。
		具体的には、結果発生に必要な事実(仮定的事実)の存在可能性の有無を、一般人の事後的な危険感を実質的な基準として判断する。
		
		本問においてAが暗証番号を変更していなければ、結果を発生させることが可能であったから、
		仮定的事実は、Aが暗証番号を変更していなかったという事実である。
		
		そして、Aが暗証番号を変更する前に甲がAを殺害して、振込送金・現金の引き出しを行う可能性は十分認められる。
		
	\sectionB{}
		したがって、結果発生の具体的危険性が認められるから、甲の行為は同罪の実行行為といえる。
		故意と不法領得の意思も認められるから、甲には電子計算機使用詐欺罪の未遂犯(刑法250条、刑法246条の2)、
		窃盗罪の未遂犯(刑法243条、刑法235条)が成立する。
		
\sectionA{建造物侵入罪の成否}
	甲が犯罪目的でATMコーナーに立ち入った行為につき、建造物侵入罪(刑法130条前段)の成否を検討する。
	
	ATMコーナーはその設置者による管理・支配が認められるので、人の看守する建造物に当たる。
	
	「侵入」とは、管理権者の意思に反する立ち入りをいう。電子計算機使用詐欺罪という犯罪目的でのATMコーナーへの立ち入りは、
	管理者である銀行の支店長の意思に反するものであることは明らかであるから、「侵入」にあたる。
	
	甲には故意が認められるから、建造物侵入罪(刑法130条前段)が成立する。
	
\sectionA{強盗致傷罪の成否}
	甲が逮捕を免れる目的でGに暴行を加え、傷害を負わせた行為につき、強盗致傷罪(刑法240条前段、刑法238条)	の成否を検討する。
	
	\sectionB{}
		強盗致傷罪は「強盗(犯人)」を主体とする犯罪であるから、強盗致傷罪の成否を論じる前提として、
		事後強盗罪(刑法238条)の成否を検討する。
		
		\sectionC{}
			先に検討した通り、甲には窃盗罪の未遂犯が成立するから、甲は「窃盗」にあたる。
	
		\sectionC{}	
			事後強盗罪は、強盗罪と同様に処罰されるから、事後強盗罪における「暴行または脅迫」は
			反抗を抑圧するに足りる程度のものであり、かつ、窃盗の現場ないし窃盗の機会の継続中に行なわれる必要がある。
		
			甲はGの胸部を力強く突いているが、これは人体の枢要部である胸部に対する強度な暴行であり、現にGは全治10日間を要する傷害を負っている。
			したがって、甲の暴行は反抗を抑圧するに足りる程度のものであるといえる。
		
			そして、甲の暴行は窃盗未遂が行なわれた現場でなされたものであるから、甲の暴行は事後強盗罪における「暴行」といえる。
		
		\sectionC{}
			甲は逮捕免脱目的でGに対して暴行を加えている。
			ここで、Gは甲を逮捕するために近づいたのではないが、事後強盗罪で要求される目的は、
			主観的な要件であるから、甲が逮捕免脱目的で行為に及んでいれば、客観的に逮捕が行われたか否かは無関係である。
			
			以上より、甲には事後強盗罪(刑法238条)が成立する。
			
	\sectionB{}
		「強盗」犯人である甲は、暴行の結果Gを「負傷」させているから、強盗致傷罪(刑法240条前段、刑法238条)が成立する。

\sectionA{罪数}
	\sectionB{クレジットカードの使用について}
		\sectionC{}
		甲にはクレジットカードによる1回の決済毎に、\MARU{1}詐欺罪(刑法246条)、
			\MARU{2}有印私文書偽造罪(刑法159条1項)、
			\MARU{3}偽造有印私文書行使罪(刑法161条1項)が成立する。
			\MARU{2}は\MARU{3}の手段として行なわれており、
			\MARU{3}は\MARU{1}の手段として行なわれているから、これらの犯罪は牽連犯(刑法54条1項後段)となる。
		\sectionC{}
			また、キャッシングによる1回の決済毎に
			\MARU{4}窃盗罪(刑法235条)が成立する。
		\sectionC{}
			承諾の限度を超えたカードの利用についてAに対する\MARU{5}背任罪(刑法247条)が成立する。
			(1)、(2)のうち、承諾の限度を超えた利用については\MARU{5}と観念的競合(刑法54条1項)となる。
		
	\sectionB{Aの宅での犯行について}
		また、甲には\MARU{6}強盗殺人罪(刑法240条後段、刑法236条2項)、
		キャッシュカードの持ち去りにつき\MARU{7}窃盗罪(刑法235条)が成立する。
		これらは併合罪(刑法45条前段)となる。
		
	\sectionB{ATMコーナーでの犯行について}
		甲には、\MARU{8}電子計算機使用詐欺罪の未遂犯(刑法250条、刑法246条の2)、
		現金を引き出そうとした行為につき\MARU{9}窃盗罪の未遂犯(刑法243条、刑法235条)、
		\MARU{10}建造物侵入罪(刑法130条前段)、
		\MARU{11}強盗致傷罪(刑法刑法240条前段、刑法238条)が成立する。
		
		\MARU{10}と\MARU{8}、\MARU{10}と\MARU{9}はそれぞれ手段目的の関係にあるから、牽連犯(刑法54条1項後段)となる。
		\MARU{9}は\MARU{11}に吸収されて一罪となる。
		ここで\MARU{8}と\MARU{11}は併合罪の関係にあるが、
		\MARU{10}が「かすがい」となって、全体が科刑上一罪となる(かすがい現象)。
					
\begin{flushright}
	以上
\end{flushright}
	
\end{document}








