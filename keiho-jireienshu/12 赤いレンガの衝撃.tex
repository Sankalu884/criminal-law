\documentclass[11pt]{jsarticle}

\usepackage[sect]{kian}
\usepackage{okumacro}
\usepackage{fancybox}
\usepackage[noalphabet]{pxchfon}  
\setminchofont{A-OTF-RyuminPro-Light.otf}
\setgothicfont{A-OTF-FutoGoB101Pr6N-Bold.otf}



\title{\vspace{-30mm}{\textgt{\Large{\fbox{12} 赤いレンガの衝撃}}}}
\date{\vspace{-15mm}}


\begin{document}

\maketitle

\sectionA{}
	甲がAの顔面を突くことでAに対側損傷をおわせて死亡させた行為につき傷害致死罪(刑法205条)の成否を検討する。
	
	\sectionB{}
		甲は少なくとも暴行の故意でAの顔面を突くという暴行を加え、Aは対側損傷という傷害を負っている。また、Aは翌日「死亡」している。
		
		では「よって」死亡したといえるか、すなわち法的因果関係の有無が問題となる。
		法的因果関係は、客観的に存在する全事情を基礎に、実行行為の危険が結果へと現実化したか否かによって判断される。
		
		Aの死亡結果は、甲の暴行によって生じた傷害を原因とするものであり、甲の行為の結果への寄与度は高い。
		また、酒に酔った人の顔面を突けば後ろ向きに倒れることは十分考えられる。本問におけるスナックの出入り口の状況を考慮すれば、
		甲の行為は、後ろ向きに倒れた人が頭部を強打して死亡結果を発生させ得る程度の傷害を負う危険性を有する行為であるといえる。
		
		したがって、Aの死亡は、甲の行為の有する危険が結果へと直接実現したものであるといえるから、甲の行為とAの死亡結果との間に法的因果関係が認められる。
		
		以上より、甲の行為には、傷害致死罪の構成要件該当性が認められる。
	\sectionB{}
		もっとも、甲がAに対して暴行を加えたのは、Aが殴りかかってきたからであるから、正当防衛(刑法36条1項)が成立しないか。
		
		\sectionC{}
			ここで、Aが殴りかかってきたのは、甲がAに対して椅子を蹴りつけたことを契機とするから、正当防衛の成立が否定されるのではないかが問題となる。
			
			正当防衛は、公的機関による法的保護が期待できない緊急状況において例外的に認められるものであるから、\MARU{1}不正の侵害により自らそれと一体の侵害を招いたときは、
			\MARU{2}招致行為と侵害が緩やかな均衡を保つ限り緊急状況性が否定され、正当防衛は成立しない。
			
			本問において、確かに甲はAに対して椅子を蹴りつけるという暴行を行っているが、Aから反撃や反論もなかった上、甲は口論の後、退店しようとしており、Aの暴行は甲の暴行と一体のものとは評価できない。
			また、甲の暴行は対物暴行であるのに対して、Aの暴行は、甲の身体に直接向けられた暴行であって、両者の均衡も保たれていない。したがって、正当防衛の前提となる緊急状況性は否定されない。
		\sectionC{}
			そこで、正当防衛の成立要件を順に検討する。
			
			Aは甲に突然殴りかかっているから、不正な法益侵害が間近に押し迫っているといえ、急迫不正の侵害が認められる。
			
			次に、正当防衛が成立するためには、刑法36条の「防衛するため」という文言から、防衛の意思が必要である。防衛の意思とは、急迫不正の侵害を認識しつつ、これを避けようとする単純な心理状態をいう。
			なお、防衛の意思は攻撃の意思が併存していることを以て直ちに否定されるものではないが、専ら攻撃の意思で反撃を加える場合は、防衛の意思が認められず、正当防衛が否定される。
			
			本問において、甲は憤激しており攻撃の意思を認めうるものの、「殴られてたまるか」という、侵害を避けようとする意思も認められるから、防衛の意思が肯定できる。
			
			では、「やむを得ずにした行為」といえるか。やむを得ずにした行為とは、反撃行為が防衛手段として必要最小限度のものであること、すなわち、反撃手段が防衛手段として相当性を有するものであることをいう。
			必要最小限度か否かは、可能な防衛手段の選択肢及び態様を考慮した上で具体的に判断する。
			
			本問において、Aは甲と比べて体格に劣り、加えて当時は酒に酔っていたから、Aの暴行は危険性の低いものと評価できる。
			一方、甲は体格に勝る上、武術の心得があり、行為自体も前述の通り死亡結果を発生させる危険性の高い行為であった。
			また、甲はBの助けを借りてAをなだめる、殴りかかってこられないように身体を押さえるなど、より攻撃性の低い手段をとることも容易であった。
			したがって、甲の行為は、防衛手段として必要用最小限度ものとはいえず、相当性は認められない。
			
			以上より、正当防衛は成立せず違法性は阻却されないから、甲には傷害致死罪(刑法205条)が成立する。
			なお、「防衛の程度を超えた」行為(過剰防衛)として任意的に刑の減免がなされる(刑法36条2項)。
			
			
\sectionA{}
	甲がAに暴行を加えてAが転倒したところ、これをよけようとしたBが額を打ち付けて打撲傷を負った行為につき傷害罪(刑法204条)の成否を検討する。
	
	\sectionB{}
	Bは甲の防衛行為(暴行)から傷害を負っているが、甲の行為とBの傷害結果との間に法的因果関係は認められるか。
	スナックの狭い出入り口に複数人いる状況において、一人を殴って転倒させれば、それを避けるために額を打ち付けて傷害を負うことは十分考えうることであり、
	甲の防衛行為にはそのような危険が含まれているといえる。
	したがって、甲の行為とBの傷害結果との間に法的因果関係を認めることができる。
	
	\sectionB{}
	では、違法性が阻却される余地はないか。
	
	Bは不正の侵害者ではないから正当防衛(刑法36条1項)は成立しないので、緊急避難(刑法37条1項)の成否を検討する。
	
	緊急避難が成立するためには、
	\MARU{1}「現在の危難」、\MARU{2}避難の意思、\MARU{3}「やむを得ずにした行為」(補充性)、\MARU{4}「生じた害が避けようとした害の程度を超えなかった」こと(害の均衡)が必要である。
	
	Aは甲に殴りかかっているから、法益侵害が間近に押し迫っているといえ、現在の危難が認められる。「避けるため」という文言から避難意思が必要であるが、防衛の意思と同様、認められる。
	緊急避難における「やむを得ずにした行為」とは、行った避難行為よりも侵害性の小さな行為では危難を回避できないこと、すなわち、補充性が認められる行為をいう。
	甲が現在の危難を避けるためには、先に検討した通り、より侵害性の小さな手段が存在し、かつそのような行為を取ることも可能であったから、補充性が認められない。
	
	したがって、緊急避難は成立しない。
	
	\sectionB{}
	もっとも、甲はAに対して暴行を加えているものの、Bに対して暴行を加える認識に欠ける。
	そこで、故意が認められないのではないか、いわゆる具体的事実の錯誤のうち、方法の錯誤(打撃の錯誤)が問題となる。
	
	故意とは、構成要件該当事実の認識及び認容をいうから、故意の認識対象として重要な事実は、特定の構成要件に該当するという事実であって、
	当該事実について認識・認容していれば故意が認められ、当該事実の具体性についての錯誤は故意を阻却しないというべきである(抽象的法定符合説)。
	
	本問において、甲は暴行罪・傷害罪の客体としての「人」について認識が認められる。
	当該「人」がAであるか、Bであるかという事実は、故意の認識対象として重要な事実ではないから、故意は阻却されない。
	
	以上より、甲には傷害罪(刑法204条)が成立する。
	
	
	
	
	
\sectionA{罪数}
	甲には、\MARU{1}Aに対する傷害致死罪(刑法205条)、\MARU{2}Bに対する傷害罪(刑法204条)が成立する。これらは1個の行為によるので、観念的競合(刑法54条1項前段)となる。





\begin{flushright}
	以上
\end{flushright}
	
\end{document}








