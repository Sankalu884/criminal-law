\documentclass[11pt]{jsarticle}

\usepackage[sect]{kian}
\usepackage{otf}
\usepackage{fancybox}
\usepackage{ascmac}
\usepackage[noalphabet]{pxchfon}  
\setminchofont{A-OTF-RyuminPro-Light.otf}
\setgothicfont{A-OTF-FutoGoB101Pr6N-Bold.otf}



\title{\vspace{-30mm}{\textgt{\Large{\fbox{19} 週刊だけど「毎朝」}}}}
\date{\vspace{-15mm}}


\begin{document}

\maketitle
\sectionA{\UTF{2474}の記事について}
	\sectionB{}
		名誉毀損罪(刑法230条1項)の成否を検討する。名誉毀損罪は、「公然と事実を摘示し」、「人の名誉を毀損した」ときに成立する。
		
		\sectionC{}
			まず、公然とは、摘示された事実を不特定又は多数の人が認識しうる状態をいう。摘示される事実は、それ自体として人の社会的評価を低下させるような事実であり、かつ、真実性の証明の対象となりうる程度に具体的でなければならない。
			
			本問において、週刊誌の記事は、不特定多数の人の目に触れるので、「公然と」にあたる。記事では、AがB殺害の真犯人である旨書かれているが、B殺害の犯人であるという事実は、Aの社会的評価を低下させるような事実である。真犯人とされている人物もAと特定されているから、摘示された事実は具体的なものであるといえる。
			
			\sectionC{}
				次に、社会的評価が実際に低下したかを立証することは困難であるから、「人の名誉を毀損した」といえるためには、名誉侵害の抽象的危険が発生したことで足り、事実の適示によって名誉が現実かつ具体的に侵害されたことまでは要しない。
				
				本問において、週刊誌の記事により、Aの名誉が侵害されるおそれが生じたといえる。
				
			\sectionC{}
				甲は、記事の発表によりAの名誉を侵害するおそれがあることを認識していた。
				そうであるにもかかわらず、記事を発表しているから、その認容も認められ、故意が肯定できる。
		\sectionB{}
			もっとも、名誉毀損罪は、摘示した事実が真実である場合にも成立しうるため、表現の自由との関係で問題が生じる。
			そこで、刑法230条の2は、名誉毀損行為が\UTF{2460}「公共の利害に関する事実に係り」(事実の公共性)、\UTF{2461}「その目的が専ら公益を図ることにあ」り(目的の公益性)、\UTF{2462}摘示した事実が「真実であることの証明があったとき」(真実性の証明)は、これを罰しないとしている。
			\sectionC{事実の公共性}
				事実の公共性は、公訴提起前の犯罪行為に関する事実であれば、肯定される(同条2項)。
				本問において、B殺害の犯人に関する事実は、公訴提起前の犯罪行為に関する事実である。
			\sectionC{目的の公益性}
				公共性を備えた事実の摘示を処罰しないという同条の趣旨から、目的の公益性は、客観的に公共性を備えた事実の摘示であるとの認識があれば足りると解する。
				本問において、甲には、公訴提起前の犯罪行為に関する事実であるとの認識があるから、目的の公益性が認められる。
			\sectionC{真実性の証明}
				真実性の証明の対象は摘示された事実である。
				本問において、AがB殺害の真犯人であるという事実がAの社会的評価を低下させるような事実であるといえるから、
				Aが犯人であるという犯罪事実自体が証明の対象となる。
				しかし、甲が当該証明を行うのは困難である。
				
				もっとも、真実性の証明に失敗しても、行為時に、行為者がその事実を真実であると誤信し、その誤信したことについて、確実な資料、証拠に照らし相当の理由があるときは、故意が阻却され、犯罪は成立しない。
				
				本問において、甲は、Aが真犯人であるという事実を真実であると誤信しているため、その誤信の相当性が問題となる。
				報道において、甲は一応自ら取材活動を行い、また、Aが真犯人であるとの論調が一般的になっていたとはいえ、それらの報道が誤っているおそれは否定できず、Aに容疑がかかっているという事実を越えて、Aが真犯人であるという事実を真実であると誤信をすることの相当性は認められない。
	
	\sectionB{}
		以上により、甲には名誉毀損罪(刑法230条)が成立する。

\sectionA{(2)の記事について}
		\sectionB{}
			甲は、Cが横領を行ったという事実を特集する記事を掲載することにより、公然と事実を摘示し、Cの名誉を毀損したといえ、故意も認められるから、名誉毀損罪の構成要件に該当する。なお、刑事事件の審判結果は公知の事実であるが、その事実を知らない人に届く可能性があるから、名誉侵害のおそれのある、摘示の許されない事実と認められる。
		
		\sectionB{}
			230条の2に当たらないか。摘示された事実は、公訴提起前の犯罪行為に係る事実ではないが、一般に、刑事事件の結果は国民の大きな関心事といえるから、事実の公共性が認められる。甲はこのことを認識していたといえるから、目的の公益性も認められる。
			
			本問において、摘示された事実は、控訴審で破棄されており、甲が真実性の証明を行うことは困難である。そこで、誤信の相当性が問題となる。
			
			確かに、裁判が確定するまでの間、事実認定は変更される可能性があるから、相当性は認められないとも思える。しかし、刑事裁判では、厳格な証明により合理的疑いを越える程度の証明がなされているから、原則として、裁判所の認定事実を真実であると信頼したことにつき相当性が認められると解すべきである。
			本問では、他に相当性を否定するような事情は存在しないから、誤信の相当性が認められ、名誉毀損罪は成立しない。
			
	\sectionA{\UTF{2476}の記事について}
		\sectionB{甲の罪責}
			甲はEが犯人であるという内容の記事を掲載することにより、公然と事実を摘示し、Cの名誉を毀損したといえ、故意も認められるから、名誉毀損罪の構成要件に該当する。
			
			摘示された事実の公共性、目的の公益性は認めうるが、裁判でFの有罪が確定している以上、真実性の証明は困難であり、甲は、自身で十分な調査を行わず、乙の発現を鵜呑みにして記事を書いているから、誤信の相当性は認められない。
			
			以上により、名誉毀損罪(刑法230条)が成立する。
		\sectionB{乙の罪責}
			\sectionC{}	乙が、甲にEが犯人であるという、Eの社会的評価を下げるような事実を伝えた行為に名誉毀損罪は成立しないか。
			
			甲一人に名誉を侵害する事実を伝えた場合、甲は特定少数人であるから、「公然と事実を摘示」したとはいえないが、当該特定少数の者から、不特定または多数人に事実が広がる可能性があれば、公然性を認めうるという見解がある(伝播性の理論)。
			
			この見解によれば、本問において、乙は、甲が自身の発言をそのまま用いて記事を作成する可能性を認識しつつ、甲に対して事実を摘示しているから、名誉毀損罪が成立しうる。
			
			しかし、そのような見解は、事実の摘示を「公然と」行うことの求める文理に反する上、抽象的危険犯と解される名誉毀損罪の危険性がさらに抽象化されてしまうため、妥当でない。
			
			したがって、乙に名誉毀損罪は成立しない。
			
			\sectionC{}
				乙は、甲の取材に応えたに過ぎず、伝播性の理論を否定する以上名誉毀損罪の故意はなく、自身の発言が記事に利用されうることを認識していたとしても共謀に当たらないから、共同正犯は成立しない。
				
				そこで幇助犯(刑法62条)の成否を検討する。幇助犯が成立するためには、幇助行為をそれと認識・認容しつつ行い、正犯行為・結果惹起を促進することが必要である。
				
				乙の行為は、甲の記事作成を容易にしたといえるから、幇助犯が成立するとも思える。
				しかし、取材に応えることにより、その内容が公表されて名誉侵害が行なわれる一般的可能性が肯定できるとしても、本問において、取材内容をどのように編集するか、そもそも編集したものを発表するか否かの判断は取材者である甲に委ねられており、また、刑事事件の取材において、付近住民から話を聞くことがまれではないことも考慮すると、一般的可能性を越える具体的な侵害利用状況の存在と、それに対する乙の認識・認容は認められない。
				
				したがって、幇助の故意が認められないから、乙には幇助犯は成立しない。
				


\begin{flushright}
	以上
\end{flushright}
	
\end{document}








