\documentclass[11pt]{jlreq}

\usepackage[sect]{kian}
\usepackage{otf}
\usepackage{fancybox}
\usepackage{ascmac}
\usepackage{pxchfon}  
\setminchofont{A-OTF-RyuminPro-Light.otf}
\setgothicfont{A-OTF-FutoGoB101Pr6N-Bold.otf}



\title{\vspace{-30mm}{\textgt{\Large{\fbox{24} 警部補のおねだり}}}}
\date{\vspace{-15mm}}


\begin{document}

\maketitle

\sectionA{甲が丙に対して「動くのには少し金がかかる」等と申し向けて100万円の小切手を受け取った行為について}
	\sectionB{}
		受託収賄罪(刑法197条1項後段)の成否を検討する。甲は警察庁の警部補であるから、「公務員」である。
		甲は、丙から事件の捜査を進めるように求められ、これに対して働きかけを行うことを了承して、賄賂として100万円の小切手を受け取っている。
		
		\sectionC{}
		問題は、B警察署で扱っている事件につき、A警察署に勤務する甲の受け取った賄賂が、「その職務に関」するものといえるかである。
		
		賄賂罪の規定は、職務行為と賄賂とが対価関係に立つことによって、職務が賄賂の影響を受け、その公正さが害されることを防ぐことを目的とする。したがって、職務関連性の認められる職務としては、当該公務員がその行為について具体的職務権限を有することまでは要求されず、当該事項がその公務員の一般的職務権限の範囲内に属すれば足りる。なぜなら、その範囲内の事項について担当しうる以上、賄賂によって職務の公正さが害されるおそれがあるからである。
	
		本問において、甲は警視庁の警部補であり、東京都の管轄区域内であれば犯罪捜査につき一般的職務権限を有する。したがって、職務関連性が認められ、受託収賄罪の客観的構成要件を充足する。
		
			\sectionC{}
				甲はこれらの犯罪事実を認識・認容しているから、故意が認められる。
				
				もっとも、賄賂を受け取った公務員に職務行為を行う意思がない場合、職務が賄賂の影響を受けてその公正さが害されるおそれは存在しないから、収賄罪は成立しないと解される。本問において、甲は「丙のために動くつもりは全くなかった」というのであるから、職務行為を行う意思は認められない。
				
				以上より、甲に受託収賄罪(刑法197条1項後段)は成立しない。
	\sectionB{}
		もっとも甲は、欺罔的な手段を用いて小切手を受け取っているから、詐欺罪(刑法246条1項)が成立しないか。
		
			小切手は当然「財物」にあたる。甲は「丙のために動くつもりは全くなかった」にもかかわらず、「少し動いてみますよ」等と述べて金品の交付を促している。この行為は、事件の捜査を進めてもらえるとの誤信を生じさせるものであり、丙の小切手交付の目的にとって重要な事項に関する錯誤を惹起する行為であるから、欺罔行為に当たる。丙は甲の欺罔行為により錯誤に陥り、小切手を交付し、甲はこれを受領している。
				
			甲には故意も不法領得の意思も認められるから、詐欺罪(刑法246条1項)が成立する。
			
\sectionA{乙がC市教育委員会職員に転籍した後、丙から警察官時代の職務に関し、10万円分の供応を受けた行為につき、単純収賄罪(刑法197条1項前段)の成否を検討する。}
	
	\sectionB{}
		甲は公務員であり、丙から賄賂として10万円分の供応を受け取っているが、これは転籍する前に担当していた職務についてのものであるから、「その職務に関し」て賄賂を受け取ったといえるかが問題となる。
		
	\sectionB{}
		賄賂罪の規定は、前述の通り、職務行為と賄賂の対価関係によって職務の公正さが害されることを防止することを目的とする。したがって、過去の職務と賄賂とが対価関係に立つのであれば、職務関連性を認めることができ、現に公務員である以上、収賄罪の成立が肯定される。
		
		なお、事後収賄罪の成立を認める見解もあるが、現に公務員である者を「公務員であった者」(刑法197条の3第3項)と解することはできない。
		
	\sectionB{}
		甲には故意も認められるから、単純収賄罪(刑法197条1項前段)が成立する。

\sectionA{甲が丙を脅して100万円を送付させた行為について}
	\sectionB{}
		甲の行為は、単純収賄罪(刑法197条1項前段)の構成要件に該当する。
			もっとも甲は、乙に対する賄賂の供与について丙を逮捕するという職務行為を行う意思がないから、同罪は成立しない。
			
	\sectionB{}
		甲は、100万円を得るために、逮捕をちらつかせるという、相手方の犯行を抑圧するに至らない程度の脅迫を加えて、丙に100万円の送付させており、恐喝罪(249条1項)の構成要件に該当する。
		
		故意と不法領得の意思も認められるから、甲には恐喝罪が成立する。

\sectionA{丙の罪責}
	\sectionB{甲に対する金品の提供について}
		甲に収賄罪が成立しない以上、収賄罪と必要的共犯の関係に立つ贈賄罪も成立しない。
		
	\sectionB{丙に対する花代・飲食料金の供与について}
		丙は乙に賄賂を提供し、乙はこれを収受している。丙には故意が認められるから、贈賄罪(刑法198条)が成立する。
		
\sectionA{罪数}
	以上より、甲には、\UTF{2460}詐欺罪(刑法246条1項)、\UTF{2461}恐喝罪(刑法249条)が成立し、これらは併合罪(刑法45条前段)となる。。乙には、\UTF{2462}単純収賄罪(刑法197条1項後段)、\UTF{2463}丙には贈賄罪(刑法198条)が成立する。
	
		
			
			
			
			
			
			
\raggedleft{以上}
	
\end{document}








