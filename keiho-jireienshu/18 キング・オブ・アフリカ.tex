\documentclass[11pt]{jsarticle}

\usepackage[sect]{kian}
\usepackage{otf}
\usepackage{fancybox}
\usepackage{ascmac}
\usepackage[noalphabet]{pxchfon}  
\setminchofont{A-OTF-RyuminPro-Light.otf}
\setgothicfont{A-OTF-FutoGoB101Pr6N-Bold.otf}



\title{\vspace{-30mm}{\textgt{\Large{\fbox{18} キング・オブ・アフリカ}}}}
\date{\vspace{-15mm}}


\begin{document}

\maketitle

\sectionA{}
	詐欺罪の成否
	
	\sectionB{}
		乙が、代金を支払うつもりがないのに、「品物を受け取るまでは金は渡せない」とAを騙して宝石を受け取った行為につき、詐欺罪(刑法246条)の成否を検討する。
		
		詐欺罪が成立するためには、「人を欺いて」錯誤を生じさせ、その錯誤に基づいて「財物」を「交付させ」ることが必要である。
		宝石は当然に「財物」にあたる。
		
		\sectionC{}
		では、欺罔行為は認められるか。欺罔行為とは、\UTF{2460}交付行為に向けて、\UTF{2461}交付の判断の基礎となる重要な事項についての錯誤を惹起する行為をいう。

	では、Aが「品物を受け取るまでは金は渡せない」と言われて、宝石を渡した行為は交付行為といえるか。
	交付行為が認められるためには、占有移転行為が被害者の意思に基づいていることが必要である。
	ホテルの部屋は個別に宿泊者の管理権が及んでいるといえるが、部屋の外にまで宿泊者の管理権が及んでいるとはいえない。したがって、本問において、Aが乙に宝石を預けて乙が部屋を出た時点で、宝石の占有はAから乙に移転している。
	Aは、乙が宝石を持って部屋の外に出ることにより、自身の占有が失われることを認識していたといえるから、交付行為が認められる。

	このように、交付行為が認められる以上、乙の行為は、交付行為に向けられた行為であるといえる。

	また、代金の支払に関する事項は、財産的損害に関わる事項であって、交付の判断の基礎となる重要な事項といえるから、
	乙の行為は、欺罔行為に当たる。

		\sectionC{}
			乙はこの欺罔行為によって錯誤に陥ったAから宝石を受け取っている。
		\sectionC{}
			乙にはこれらの犯罪事実の認識・認容があり、故意が認められる。
			当初から宝石を持ち逃げするつもりであったから、不法領得の意思も認められる。
	\sectionB{}
		以上により、乙には詐欺罪(刑法246条)が成立する。
		なお、後述する通り、乙と丙との間で共同正犯となる。
		
\sectionA{強盗殺人罪(刑法240条)の成否}
	\sectionB{}
		甲が、宝石の代金支払を免れるため、Aを殺害した行為につき、強盗殺人罪(刑法240条)の成否を検討する。
		強盗殺人罪は、「強盗」が、被害者を「死亡させた」ときに成立する。
	\sectionB{}
		強盗殺人罪は「強盗(犯人)」を主体とする犯罪であるから、強盗殺人罪を検討する前提として、強盗罪の成否を検討する。強盗罪は、暴行・脅迫を手段として、相手方の意思に基づかず財物を奪取する犯罪であるから、暴行・脅迫は、財物奪取に向けられている必要があるが、甲が乙を殺害したのは、乙らがホテルから離れたのを見届けた後であり、財物の占有は既に確保されているといえる。したがって、Aの殺害は、財物奪取に向けられた暴行とは評価できず、1項強盗罪(刑法236条1項)は成立しない。
		
		そこで、2項強盗罪の成否を検討する。
		\sectionC{}
			強盗罪における暴行・脅迫は、相手方の意思を抑圧する程度のものである必要があるが、
			被害者を殺害する行為は、その最たるものといえるから、強盗の手段としての暴行が認められる。
		\sectionC{}
			2項強盗罪における「財産上不法の利益」を安易に認めると、処罰範囲が不当に拡大することとなる。
			そこで、2項強盗罪の成立を認めるためには、財産上不法の利益の移転が現実的かつ具体的なものである必要があると解すべきである。
		
			Aは無店舗型個人営業の宝石商であり、Aを殺害してしまえば、宝石が奪われたことを知る者はいなくなり、
			甲らは宝石を確定的に取得できる一方で、Aが代金や宝石の返還を請求することはできくなるから、
			財産上の不法な具体的な利益が現実に移転したといえる。
		
			故意と不法領得の意思も認められるので、甲には2項強盗罪が成立する。
	\sectionB{}
		甲はAを殺害して宝石を奪うという計画の下、Aに拳銃を発射しているから、甲には、殺意が認められる。
		そしてAは「死亡」している。
		
		問題は、刑法240 条は結果的加重犯としての規定形式を有するため、強盗犯人が殺意を持って被害者を殺
害した場合であっても、刑法240 条が適用されるかである。

	ここで、仮に殺人罪と強盗致死罪の観念的競合を認めると、死亡結果を二重に評価していることになる上、強盗罪と殺人罪の観念的競合とした場合には、殺人の故意のある方が、故意がない場合(強盗致死罪)よりも刑の下限が軽くなるという不均衡が生じるから、強盗殺人にも同条は適用されると解すべきである。
	
	以上により、甲には強盗殺人罪(刑法240条)が成立する。なお、後述する通り、乙との間で共同正犯となる。
		
\sectionA{甲と乙の共犯関係}
	\sectionB{}
		以上検討した、詐欺罪と強盗殺人罪は(共謀)共同正犯(刑法60条)とならないか。
		共同正犯の成立要件は、\UTF{2460}共謀と、\UTF{2461}共謀に基づく実行であるが、
		詐欺罪と強盗殺人罪において実行行為を行ったのはそれぞれ乙と甲であり、互いに一方の犯罪の実行行為を担当していないことから問題となる。
	
		ここで、刑法60条はその文言上、実行行為の有無によって共同正犯を区別していないし、共同正犯における「一部実行全部責任」という法的効果が導かれるのは、構成要件該当事実を共同で惹起するからであるので、構成要件該当事実惹起に重要な事実的寄与を果たすことにより構成要件該当事実全体の惹起が認められれば、実行行為の分担は必ずしも必要でないと考えることができる。
	
	\sectionB{}
		本問において、甲が乙に対して報酬の用意があることを付言しつつ犯罪を持ちかけたのに対して、乙は具体的な犯行場所、
		犯行手順を示しており、Aを殺害して宝石を奪うという、詐欺罪および強盗殺人罪の共謀が成立していたといえる。
		
		また、甲は犯罪の実行を乙に持ちかけ、主体的に行動しており、一方、乙はAをホテルに呼び出すという本件犯行に必要不可欠な要素を担っていた。したがって、結果発生に対して重要な事実的寄与が認められる。	
		
		甲と乙は、計画した犯行手順に則って、上記共謀に基づいて犯罪を実行したといえるから、共同正犯が成立する。	

\sectionA{乙の300万円の債務免脱について}
	\sectionB{}
		乙は、Aに対して300万円の債務があることを甲に秘して、Aが殺害されることにより300万円の債務を免脱している。
		
		乙はAに借用証書を差し入れておらず、他に借金のことを知る者はいなかったから、Aの殺害により、
		Aは300万円の請求が行えなくなる一方、乙は確定的に債務を免脱できる。
		
		したがって、具体的な利益が現実に移転したといえるから、乙には2項強盗殺人(刑法240条、236条2項)が成立する。
	\sectionB{}
		甲らは、乙の債務を免脱する意図を知らないから、この部分について共謀は成立しておらず、帰責されることはない。
		したがって、乙にのみ、2項強盗殺人罪の共同正犯が成立する。
		
\sectionA{丙の罪責}
	詐欺罪について、丙に共同正犯は成立するか。
	
	丙は、運転や見張りを行っているが、丙が本件犯行に参加したのは、乙から娘に危害を加える旨を告知されて脅されたからであり、甲乙と違って、主体的に本件犯行に参加したとはいえない。
	
	しかし、丙は乙の申出にしぶしぶながらも了承しており、車の中での甲と乙の打ち合わせの際にも異論を挟まなかったというのであるから、少なくとも黙示的な共謀が認められる。
	
	また、乙から宝石を受け取って309号室に行くことにより占有を確保しているから、実行行為の一部を分担している。
	実行行為の一部の分担は、結果発生に対する重要な事実的寄与と考えられるから、共同正犯が成立する。
	
\sectionA{罪数}
	以上より、甲には、\UTF{2460}詐欺罪の共同正犯(刑法60条、刑法246条)、\UTF{2461}宝石の返還を免れることによる2項強盗殺人罪の共同正犯(刑法60条、刑法240条)が成立し、これらは、宝石という同一の財産に向けられたものであるから、2項強盗殺人罪により包括一罪となる。
	
	乙には、\UTF{2462}詐欺罪の共同正犯(刑法60条、刑法246条)、\UTF{2463}宝石の返還を免れることによる2項強盗殺人罪の共同正犯、\UTF{2464}300万円の債務を免れることによる2項強盗殺人罪の共同正犯(刑法60条、刑法240条)が成立する。\UTF{2462}と\UTF{2463}は包括一罪となり、これと\UTF{2464}は観念的競合(刑法54条)となる。
	
	丙には、\UTF{2465}詐欺罪の共同正犯(刑法60条、刑法246条)が成立する。

	
		
	


\begin{flushright}
	以上
\end{flushright}
	
\end{document}








