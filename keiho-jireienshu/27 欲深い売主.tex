\documentclass[fontsize=11pt,
jlreq_notes
]{jlreq}

% 1. フォント・日本語処理
\usepackage{pxchfon} 
\usepackage{otf}

% 2. フォント設定
\setminchofont{A-OTF-RyuminPro-Light.otf}
\setgothicfont{A-OTF-FutoGoB101Pr6N-Bold.otf}

% 3. 汎用パッケージ
\usepackage{fancybox}

% 4. 自作パッケージ
\usepackage[sect]{kian}


\title{\vspace{-30mm}{\textgt{\Large{\fbox{27} 欲深い売主}}}}
\date{\vspace{-15mm}}


\begin{document}

\maketitle

\sectionA{甲の罪責}
	\sectionB{本件不動産をAに売却した後、登記移転が完了していないことを奇貨としてBと売買契約を締結した行為について委託物横領罪(刑法252条1項)の成否を検討する。}

			同罪の成立要件は、\UTF{2460}他人の物、\UTF{2461}自己の占有、\UTF{2462}横領である。そして、占有離脱物横領罪(刑法254条)を越える不法内容を基礎づけるために、自己の占有は、委託信任関係に基づくものであることが必要である。
		\sectionC{}
			所有権は売買契約の成立によって売主から買主に移転する(民法176条)。横領罪で保護すべき実質を欠くことを理由として、代金の大部分が支払われて初めて横領罪が成立しうるという見解もあるが、どちらにせよ本問において、Aは売買契約に基づいて代金の8割を支払っているから本件不動産は他人の物である。
		\sectionC{}
			では、本件不動産の登記名義を有していることによって「自己の占有」といえるか。横領罪における「占有」には、事実上の支配のみならず、法律上の支配も含まれると解される。なぜなら、横領行為は、事実行為による場合のほか、法律的な処分行為によっても行なわれうるため、後者の処分行為を類型的に容易に行いうる地位(濫用のおそれのある支配力)を有していることが重要だからである。
			
			登記名義人は不動産を自由に処分できる地位にあるから、このことから「自己の占有」を基礎づけることができる。そして、売買契約に基づいて、売主には、買主に対して不動産を引渡して登記を移転させる義務を負うから(民法560条参照)、この占有は委託信任関係に基づくといえる。
		\sectionC{}
			次に、横領行為とは、不法領得の意思を発現する一切の行為をいう。そして、横領罪における不法領得の意思とは、委託の趣旨に反した、物の利用意思をいう。判例における不法領得の意思の定義では、利用意思が要求されていないが、横領罪の利欲犯としての性格から器物損壊罪よりも重く処罰されているのだから、利用意思を要求すべきである。
			
			甲がAに売却した土地をさらにBに売却する行為は、Aの甲に対する委託の趣旨に反し、Aに対する売主の権利の範囲を逸脱した行為であって、本件土地を売却することによってその代金を得ようとする行為である。甲はこのような処分行為をする意思に基づいてBに本件不動産を売却しており、不法領得の意思の発現行為が認められる。したがって、甲の行為は「横領行為」にあたる。
			
			しかし、不動産売買においては登記の移転が対抗要件とされているから、登記移転までは確定的に所有権侵害が生じたとはいえない。したがって、Bが登記の移転を完了した時点で「横領した」といえる。
			
			本問において、Bは甲との売買契約を解除して代金の返還を受けており、登記を移転していない。したがって、Aの所有権を侵害したとはいえず、「横領した」とは評価できないから、横領罪は成立しない。
	\sectionB{同行為につき、詐欺罪(刑法246条1項)の成否を検討する。}
		本件不動産の代金は当然、「財物」にあたる。次に、詐欺罪が成立するためには、「人を欺」く行為(欺罔行為)が認められることが必要である。
		\sectionC{}
			欺罔行為とは、\UTF{2460}交付行為に向けて、\UTF{2461}交付の判断の基礎となる重要な事項についての錯誤を惹起する行為をいう。
		
			甲の行為は、代金の交付に向けられたものであるが、\UTF{2461}にあたるといえるか。詐欺罪において財産は交換手段・目的達成手段として保護されているから、重要な事項とは、交付によって達成される目的にとって重要な事項をいう。
		
			本問において、Bの取引目的は、本件不動産を取得するだけにとどまらない。仮に取引先のAと紛争が生じれば、Aとの信頼関係が損なわれ、ひいては勤務先の地位にも影響しかねないという、重大なリスクが発生しうる。このようなリスクが生じないことは、取引目的にとって重要な事項にあたる。
		
			したがって、甲の行為は欺罔行為に当たる。
		\sectionC{}
			Bは甲の欺罔行為により錯誤に陥り、この錯誤に基づいて代金を交付し、甲はこれを受領している。甲には故意も不法領得の意思も認められるから、詐欺罪(刑法246条1項)が成立する。
	\sectionB{甲が本件不動産をAに売却した後に抵当権を設定して登記を備えた行為につき、委託物横領罪(刑法252条1項)の成否を検討する。}
		\sectionC{}
			1で検討した通り、本件不動産は、他人の物であり、委託信任関係に基づく自己の占有も認められる。
		\sectionC{}
			したがって、問題は、抵当権を設定し、登記をする行為が「横領した」といえるかである。
			抵当権を設定する行為は、不法領得の意思を発現する行為といえる。そして、登記された抵当権が実行されれば、本件土地は競売にかけられ、所有権は失われてしまうから、「横領した」といえる。
			
		故意と不法領得の意思も認められるから、甲には委託物横領罪(刑法252条1項)が成立する。
			
	\sectionB{甲が本件不動産をAに売却した後、登記移転が完了していないことを奇貨として乙に売却し、登記を備えさせた行為について委託物横領罪(刑法252条1項)の成否を検討する。}
	本件不動産に抵当権を設定したことにより、委託信任関係は消滅し、かつ、横領は完結しているから、その後の売却は不可罰的事後行為として、横領行為とは評価できないとも思える。
	
	しかし、A都の関係では、甲は依然として登記を備えさせる義務を負担しており、委託信任関係は消滅していない。また、抵当権設定行為は本件不動産の交換価値の一部を侵害するに過ぎず、売却によって所有者に残された交換価値を一層侵害することが可能である以上、横領行為と評価することは可能である。甲は本件不動産を乙に売却した後、登記を移転させているから、横領罪は既遂に達している。
	
	故意と不法領得の意思も認められるから、甲には委託物横領罪(刑法252条1項)が成立する。

\sectionA{乙の罪責}
	乙が、事情を知りながら本件不動産を買い受けた行為について、委託物横領罪の共犯は成立しないか。
	
	ここで、民法177条は、悪意の譲受人であっても、先に登記を備えれば適法に所有権を取得しうることを定めており、民法上許容された行為を処罰することはできないから、背信的悪意者でない限り、委託物横領罪の共犯は成立しないと解すべきである。
	
	そこで、乙が背信的悪意者にあたるか否かを検討する。乙は、単に二重譲渡であることを知っていたに過ぎず、登記の欠缺を主張することが信義に反するとまではいえないため、背信的悪意者には当たらない。
	
	したがって、委託物横領罪の共犯は成立しない。
	
\sectionA{罪数}
	以上により、甲には、\UTF{2460}詐欺罪(刑法246条1項)、\UTF{2461}抵当権設定につき委託物横領罪(刑法252条1項)、\UTF{2462}乙への売却につき委託物横領罪が成立する。\UTF{2461}と\UTF{2462}は包括一罪となる。これと\UTF{2460}は併合罪(45条前段)となる。







\raggedleft{以上}
	
\end{document}








