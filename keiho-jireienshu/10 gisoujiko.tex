\documentclass[11pt]{jsarticle}

\usepackage[sect]{kian}
\usepackage{okumacro}
\usepackage{fancybox}
\usepackage[noalphabet]{pxchfon}  
\setminchofont{A-OTF-RyuminPro-Light.otf}
\setgothicfont{A-OTF-FutoGoB101Pr6N-Bold.otf}



\title{\vspace{-30mm}{\textgt{\Large{\fbox{10} 偽装事故の悲劇}}}}
\date{\vspace{-15mm}}


\begin{document}

\maketitle

\sectionA{}
	甲がA車に自動車事故を装って衝突しAを死亡させた行為について、傷害致死罪(刑法205条)の成否を検討する。
	\sectionB{}
		甲はA車に衝突することでAに頸椎捻挫等の「傷害」を負わせ、Aは「死亡」している。
		問題は、甲の行為に「よって」Aが死亡したといえるかである。
		
		本問では、行為と結果との間に事実的なつながりは存在するが、
		同時に\MARU{1}乙車の衝突、\MARU{2}A自身が安静に勤めなかったという介在事情も存在するため、法的因果関係を認めてもよいかが問題となる。
		
		法的因果関係は、偶然的結果を排除し適正な帰責範囲を確定するために要求される。
		これは客観的に存在する全事情を判断資料とし、実行行為の危険が結果へと現実化したか否かによって判断される。
		\sectionC{乙車の衝突について}
			Aの直接の死因は乙車の衝突時に負った後頸部血管損傷等の傷害であって、介在事情の結果への寄与度は大きい。
			しかし、車両が追突のショックによって交差点の中に押し出されれば、二次的な交通事故を起こす可能性は高いといえる。
			したがって、乙車の衝突は、甲の行為によって誘発されたものであり、実行行為の危険が介在事情を通じて結果へと実現したと評価できるから、法的因果関係は肯定される。
		\sectionC{Aの抜管行為について}
			Aの行為は、死亡に至る右後頸部の傷害に基づく頭部循環障害による脳機能障害の危険を新たに作り出したのではなく、
			既に生じていた当該危険を減退させる措置をとらなかったという不作為であって、結果への寄与度は小さい。
			
			以上より、Aの死亡結果は、甲の行為が持つ「他の自動車事故がからむという介在事情を介して死亡結果を引き起こす危険」が現実化したものと認められるから、法的因果関係が認められる。
	\sectionB{}
		もっとも、Aは軽傷を負うことにつき承諾しているから、違法性が阻却されないか。
		
		ここで、被害者の承諾によって違法性が阻却される根拠は、身体等の個人的法益についてその法益主体が法益を放棄している場合には、当該法益の要保護性が失われることに求められる。
		
		本問において、Aは偽装事故によって軽傷を負うことについて同意をしているから、その限度で法益の要保護性が失われているといいうる。
		しかし、重傷や死亡の結果は同意の範囲外であって、同意によって法益は放棄されていない。
		
		したがって、甲の行為の違法性は阻却されない。
	\sectionB{}
		では、故意は認められるか。
		甲は軽傷を負わせるつもりで行動しており、軽傷を負うという限度でAの同意により違法性が阻却されるのであれば、
		甲には犯罪事実の認識が認められず、責任故意が否定される。
		
		そこで、Aの承諾によって違法性が阻却されるかが問題となる。
		偽装事故はAの依頼によって実行されたものであって、軽傷を負うことにつきAの真摯な同意があったといえる。
		しかし、形式的には法益を侵害しながら、なおその行為を正当化するためには、当該行為が社会的相当性を有することが必要である。
		
		本問において、Aの承諾は、保険金詐欺のためになされており、承諾の目的が正当であるとは言えない。
		また、傷害を発生させる方法も、甲の運転する自動車をA車に追突させるという危険なものであり、手段の相当性も認められない。
		
		以上より、Aの承諾には社会的相当性が認められないから、甲の認識を前提としても違法性が阻却されることはなく、故意が認められる。
\sectionA{}
	乙が前方不注視によりA車両に衝突し、Aを死亡させた行為につき過失運転致死罪(自動車運転死傷行為処罰法5条)の成否を検討する。
	\sectionB{}
		前方不注視で交差点に侵入する行為は、他の自動車等と衝突し、死傷結果を発生させる実質的危険性のある行為であり、同罪の実行行為にあたる。
		Aは死亡している。
	\sectionB{}
		では、Aが乙の行為に「よって死傷」したといえるか。Aの死亡結果と乙の行為との間に因果関係が認められるかが問題である。
		
		Aの直接の死因は乙車の衝突時に負った後頸部血管損傷等の傷害であるから、事実的因果関係は認められる。
		結果との間にはA自身のが安静に勤めなかったという介在事情が存在するが、先に検討した通り、これに因果関係は否定されない。
		
		また、前方を注視して運転していれば、Aを避けて運転する、一時停止するなどの措置をとることによって衝突を避けることが可能であったといえるから、結果回避可能性も認められる。
		
		したがって、乙の前方不注視による運転とAの死亡結果との間には因果関係が認められる。
	\sectionB{}
		また、前方を注視して運転することは、自動車を運転する者にとって最も基本的な義務であり、これを怠って運転すれば死傷結果を生ずることは十分予見可能であるといえるから、
		乙には、Aの死亡結果を帰責できるだけの過失が認められる。
		
		以上より、乙には過失運転致死罪が成立する。


\begin{flushright}
	以上
\end{flushright}
	
\end{document}








