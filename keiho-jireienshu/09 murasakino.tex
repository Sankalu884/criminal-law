\documentclass[11pt]{jsarticle}

\usepackage[sect]{kian}
\usepackage{okumacro}
\usepackage{fancybox}
\usepackage[noalphabet]{pxchfon}  
\setminchofont{A-OTF-RyuminPro-Light.otf}
\setgothicfont{A-OTF-FutoGoB101Pr6N-Bold.otf}



\title{\vspace{-30mm}{\textgt{\Large{\fbox{9} 紫の炎}}}}
\date{\vspace{-15mm}}


\begin{document}

\maketitle

\sectionA{}
	甲が、Aの居住する集合住宅の共用部および居住者用の屋外駐車場に立ち入った行為につき、邸宅侵入罪(刑法130条前段)の成否を検討する。
	
	\sectionB{}
		「邸宅」とは、居住用の建造物で住居以外のものをいう。集合住宅の共用部は「邸宅」にあたる。
		そして、隣接する屋外駐車場も「邸宅」の囲繞地として刑法130条の保護範囲に含まれる。
		集合住宅には管理人がおり、また、駐車場に立ち入るにはカードキーが必要であるから、両者は「人の看守する邸宅」である。
	\sectionB{}
		次に、「侵入」とは管理権者の意思に反する立ち入りをいう。
		
		集合住宅には管理人が常駐するとともに「居住者以外立ち入り禁止」という立札がたっており、居住者以外の立ち入りを禁じる管理者の意思が示されているといえる。
		また、駐車場は施錠されていることから、カードキーを有する者以外の立ち入りを禁じる管理者の意思が示されているといえる。
		
		甲は、これらの意思に反して共用部および駐車場内に立ち入っており、「侵入」にあたる。
	\sectionB{}
		甲がこれらの場所に立ち入ったのは、放火目的であり「正当な理由」は認められない。甲はこれらの犯罪事実を認識・認容しているから、故意が認められる。
		
		以上より、甲には共用部への侵入、駐車場への侵入それぞれにつき邸宅侵入罪が成立する。
		これらは管理権の侵害という同一の法益侵害に向けられ、かつ、意思の連続も認められるので、包括一罪となる。
\sectionA{}
	甲がAの原動機付自転車に放火し、エレベーターホールの天井を焼失させた行為につき、現住建造物等放火罪(108条)の成否を検討する。
		\sectionB{}
			甲は、Aの居室付近に火をつけることを目的にAの原動機付自転車に放火し、エレベーターホールの天井を焼失させている。
			そこで、エレベーターホールが現住建造物にあたるかが問題となる。
			
			Aは集合住宅に現に居住しているから、現住性が認められる。
			エレベーターホール自体に人が住んでいるわけではないが、エレベーターホールは集合住宅の居室部分と不可分的に接続しており、
			また、エレベーターホールから居室部分へと延焼する可能性が認められる。
			したがって、エレベーターホールは、Aの居室をはじめとする集合住宅の居室部分と一体であるといえるから、現住建造物に当たる。
		\sectionB{}
			甲は、Aの原動機付自転車に点火することで「放火」している。
		\sectionB{}
			「焼損」とは、火が媒介物を離れて目的物が独立に燃焼を継続しうる状態になったことをいう。
			エレベーターホールの天井の木材は全て焼失しているから、「焼損」したといえる。
		\sectionB{}
			甲にはこれらの事実の認識・認容が認められるから、現住建造物等放火罪(刑法108条)が成立する。
\sectionA{}
	甲がDの車に放火した行為につき、建造物等以外放火罪(刑法110条1項)の成否を検討する。
	\sectionB{}
		甲は、建造物ではないDの車にライターで点火することで「放火」し、「焼損」させている。
	\sectionB{}
		ではこれによって、「公共の危険」が生じたといえるか。
		
		「公共の危険」とは、不特定又は多数の人の生命、身体又は財産に対する危険をいう。
		「公共の危険」について、刑法108条、刑法109条1項に規定する建造物等に延焼する危険に限定すべきとの見解もあるが、
		公共の危険の発生は建造物等への延焼以外からも生ずる以上、公共の危険を限定する合理的根拠に欠ける。
		
		本問において、D車の近くにはE車が停められており、E車に延焼する危険が認められる。
		また、車には燃料としてガソリンが積まれているから、ひとたび火がつけば炎が大規模なものとなり、
		集合住宅に延焼したり、火災の煙や熱によって直接、周囲の住民や通行人の生命・身体に危険を及ぼす可能性も十分認められる。
		
		したがって、「公共の危険」が生じたといえる。
	\sectionB{}
		甲に故意は認められるか。
		
		ここで、刑法110条1項は「よって」という文言を使っているから、結果的加重犯であると解される。
		したがって、行為者が仮に加重結果である「公共の危険」の発生について認識していなかったとしても、他の犯罪事実について認識・認容が認められれば故意犯が成立する。
		
		甲が「公共の危険」の発生を認識していたかは定かではないが、
		D車を燃やすことについて認識・認容しているといえるから、故意が認められる。したがって、建造物等以外放火罪(刑法110条1項)が成立する。
\sectionA{罪数}
	以上より、甲には\MARU{1}邸宅侵入罪(刑法130条前段)、\MARU{2}現住建造物等放火罪(刑法108条)、\MARU{3}建造物等以外放火罪(刑法110条1項)が成立する。
	
	このうち、\MARU{1}と\MARU{2}は目的・手段の関係にあるので牽連犯(刑法54条1項後段)となる。また、\MARU{1}と\MARU{3}も目的・手段の関係にあるので牽連犯となる。
	ここで、\MARU{2}と\MARU{3}は併合罪の関係にあるが、\MARU{1}が「かすがい」となって、3罪全てが科刑上一罪となる(かすがい現象)。
		


\begin{flushright}
	以上
\end{flushright}
	
\end{document}








