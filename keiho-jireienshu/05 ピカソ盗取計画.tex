\documentclass[11pt]{jsarticle}

\usepackage[sect]{kian}
\usepackage{okumacro}
\usepackage{fancybox}
\usepackage[noalphabet]{pxchfon}  
\setminchofont{A-OTF-RyuminPro-Light.otf}
\setgothicfont{A-OTF-FutoGoB101Pr6N-Bold.otf}



\title{\vspace{-30mm}{\textgt{\Large{\fbox{5} ピカソ盗取計画}}}}
\date{\vspace{-15mm}}


\begin{document}

\maketitle

\sectionA{甲の罪責}
	\sectionB{}
		倉庫の敷地内に侵入した行為につき、建造物侵入罪(刑法130条)の成否を検討する。
		倉庫は、「建造物」にあたる。
		そして、建造物の敷地であっても、建造物と一体的に使用されており、
		塀などで外部と隔離されている場合は、不法な侵入を排除する必要性が高く、
		「建造物」にあたると解する。
		
		本問において、倉庫の敷地には塀が設置されており、倉庫と一体的に使用されているといえるから、
		倉庫の敷地も「建造物」にあたる。
		
		甲は、管理権者の意思に反して倉庫の敷地に侵入しており、故意もあるから、
		建造物侵入罪(刑法130条)が成立する。
		
		なお、後述の通り、乙と共同正犯となる。
	
	\sectionB{甲が、倉庫の建物内に侵入するため鍵を壊そうとした上、
	追いかけてきたCに向かって拳銃を発射し、Cに全治7日の傷害を負わせた行為につき、強盗致傷罪(刑法240条)の成否を検討する。}
	
		強盗致傷罪にいう「強盗」には、事後強盗罪(刑法238条)も含まれるから、甲に事後強盗(未遂)罪が成立するかを検討する。
			\sectionC{}
				事後強盗罪の主体たる「窃盗」は窃盗罪の犯人を意味する。
				そこで、甲に窃盗未遂罪(刑法243条、刑法235条)が成立するか、すなわち、倉庫の鍵を壊そうとした行為に
				窃盗罪の実行の着手が認められるかが問題となる。
				
				実行の着手が認められるためには、刑法43条の文言から、構成要件に密接な行為がなされ、
				かつ、
				未遂の処罰根拠から、結果発生の現実的危険性が発生することが必要である。
				
				本問において、鍵のかかった倉庫から財物を窃取するためには、鍵を破壊するなどして、開錠することが必要不可欠であり、
				また、窃取行為と錠の破壊行為には時間的場所的接着性も認められる。したがって、鍵を破壊する行為は、窃取行為に密接な行為といえる。
				
				そして、鍵が破壊されて倉庫内に侵入することができれば、絵画を窃取することに特段の障害は存在しないし、時間的場所的接着性も認められるから、
				結果発生の現実的危険性が肯定できる。
				
				したがって、甲の行為には窃盗罪の実行の着手が認められる。故意と不法領得の意思も認められるから、甲には窃盗未遂罪が成立する。
			\sectionC{}
				次に、甲は、「逮捕を免れ」るために、拳銃を発射している。
			\sectionC{}
				事後強盗罪は「強盗として論」じられるから、事後強盗罪にいう「暴行又は脅迫」とは強盗罪のそれと同程度のもの、すなわち、
				犯行を抑圧する程度の暴行脅迫である必要がある。
				
				本問において甲は、威嚇射撃をして脅迫を行っているが、拳銃という殺傷能力の極めて高い武器を用いた脅迫は、犯行を抑圧するに至る脅迫にあたる。
			\sectionC{}
				また、事後強盗罪における脅迫は、窃盗の機会に行われる必要がある。
				甲が拳銃を発射したのは、Cからの追跡が継続していており、Cから逃げて逮捕を免れるためであるから、甲の脅迫は窃盗の機会に行われたといえる。
				
				もっとも、甲は絵画を窃取できていないから、事後強盗未遂罪(刑法243条、刑法238条)が成立する。
	\sectionB{}
		Cは全治7日間の傷害を負っている。脅迫とCの傷害との間の因果関係も認められる。
		
		強盗致傷罪における傷害結果は、強盗の機会に発生すればよいから、事後強盗の状況下での脅迫によって傷害が発生した場合も含まれる。
		
		そして、強盗致傷罪は被害者の人身保護を重視するものであるから、基本犯が未遂であっても、傷害結果が生じれば既遂となる。
		
		したがって、甲には強盗致傷罪(刑法240条)が成立する。
		
		なお、後述の通り、乙と共同正犯となる。
		
\sectionA{乙の罪責}
	\sectionB{}
		倉庫の敷地内に侵入した行為につき、建造物侵入罪の共同正犯(刑法60条、刑法130条)の成否を検討する。
		
		共同正犯の成立要件は、\MARU{1}共謀と\MARU{2}共謀に基づく実行である。
		
		甲と乙は、倉庫に侵入することにつき共謀を行い、これに基づいて「建造物」に「侵入」しているから、建造物侵入罪の共同正犯が成立する。
	\sectionB{}
		では、甲に成立する強盗致傷罪(刑法240条)につき、(共謀)共同正犯が成立しないか。
		
		\sectionC{}
			乙は見張りを担当していたに過ぎないことから、(共謀)共同正犯が成立しうるかが問題となる。
		
			ここで、刑法60条はその文言上、実行行為の有無によって共同正犯を区別していないし、共同正犯における「一部実行全部責任」という法的効果が導かれるのは、構成要件該当事実を共同で惹起するからであるから、構成要件該当事実惹起に重要な事実的寄与を果たすことにより構成要件該当事実全体の惹起が認められれば、実行行為の分担は必ずしも必要でないと考えることができる。
		\sectionC{}
			上述の通り、共同正犯の成立要件は、\MARU{1}共謀と\MARU{2}共謀に基づく実行である。
			\sectionD{}
				乙は、甲の説得に応じて犯行に加わることを了承しており、絵画の売却代金から分前を分配される約束になっていた。
				また、鍵を破壊して倉庫内に侵入するには、相当の時間を有し、かつ、音も発生するから、
				犯罪を実行するには見張りが必要不可欠であるといえる。
				さらに、犯行が発覚した際には、逮捕を免れるために甲とともに暴行を加えることについても仕方がないと思っていた。
		
				このようなことからすると、乙は、窃盗及び事後強盗を自己の犯罪として行う意思、及び故意を有していたと認められ、
				かつ、見張をすることにより、行為段階での結果発生の現実的危険性を高めたといえるから、重大な事実的寄与も認められる。
				したがって、甲乙間には共謀が認められる。
				なお、乙は、逮捕を免れるための暴行につき甲が拳銃を用意していることまで知らなかったことから、
				事後強盗罪の共謀が認められないとも思える。
				しかし、乙は、逮捕を免れるために素手で暴行を行うという、同一の構成要件に該当する事実を認識・認容している以上、共謀は否定されない。
			\sectionD{}
				また、甲は以上のような共謀に基づいて実行行為を行なっている。
		\sectionC{}
			ここで、強盗致傷罪は事後強盗罪の結果的加重犯であり、結果的加重犯であっても共同正犯が成立するか否かは問題となりうるが、
			基本犯について共同正犯が成立し、加重結果を共同で惹起したと評価できる以上、
			結果に対する因果性は肯定できるから、結果的加重犯についても共同正犯の成立を認めても良いと解される。
		
			以上より、乙には強盗致傷罪の共同正犯(刑法60条、刑法240条)が成立する。
		
		
		
		
\sectionA{罪数}
	甲には、\MARU{1}建造物侵入罪の共同正犯(刑法60条、刑法130条)、\MARU{2}強盗致傷罪の共同正犯(刑法60条、刑法240条)が成立する。
	乙には、\MARU{3}建造物侵入罪の共同正犯(刑法60条、刑法130条)、\MARU{4}強盗致傷罪の共同正犯(刑法60条、刑法240条)が成立する。
	
	\MARU{1}と\MARU{2}、\MARU{3}と\MARU{4}は、通例、手段と結果の関係にあるので牽連犯(刑法54条1項後段)となる。

\begin{flushright}
	以上
\end{flushright}
		
		
\end{document}
