\documentclass[11pt]{jsarticle}

\usepackage[sect]{kian}
\usepackage{otf}
\usepackage{fancybox}
\usepackage{ascmac}
\usepackage[noalphabet]{pxchfon}  
\setminchofont{A-OTF-RyuminPro-Light.otf}
\setgothicfont{A-OTF-FutoGoB101Pr6N-Bold.otf}



\title{\vspace{-30mm}{\textgt{\Large{\fbox{17} 逆恨み}}}}
\date{\vspace{-15mm}}


\begin{document}

\maketitle

\sectionA{甲が虚偽の通報により、30回にわたって、緊急出動を行わせて行為につき、偽計業務妨害罪(刑法233条)の成否を検討する。}
	
	\sectionB{}
	偽計業務妨害罪が成立するには、「業務」を「偽計」を用いて「妨害した」といえることが必要である。
	
	業務妨害罪にいう「業務」とは、職業その他社会生活上の地位に基づき継続して行う事務又は事業をいう。
	甲の虚偽通報がなければ行なわれたであろう、警察の通常の「業務」は、上記定義に当てはまるが、同時に「公務」でもあるから、
	公務執行妨害罪が暴行・脅迫という限られた手段に対してのみ保護されていることとの関係上、公務が本罪の「業務」に含まれるかが問題となる。
	
	ここで、強制力を行使する権力的公務は、強制力により妨害行為を排除できるため、業務妨害罪の規定による保護を必要としない。
	したがって、強制力を行使する権力的公務については公務執行妨害罪の適用があるが、それ以外の公務については、公務執行妨害罪および業務妨害罪の適用があると解される。
	
	本問の、甲の虚偽通報がなければ行なわれたであろう、警察の通常の「業務」は、非権力的公務であるから、本罪の「業務」にあたる。
	
	\sectionB{}
	「偽計」とは、人を欺罔し、あるいは人の錯誤又は不知を利用することをいう。
	
	虚偽の通報を行って、緊急出動を行わせる行為は、「偽計」にあたる。
	
	\sectionB{}
	「妨害した」とは、妨害の結果発生までは不要であり、業務を妨害するに足る行為が行われればよいとする見解もある。

	しかし、条文の文言は「業務を妨害した」となっており、現実に業務が妨害されたことを要求する解釈が文理に忠実である。
	加えて、妨害手段による処罰範囲の限定が困難であることから、「妨害した」の要件で限定するほかなく、
	妨害の危険だけで本罪の成立を認めると、処罰範囲が大幅に広がり妥当でない。
	したがって、「妨害した」といえるためには現実に業務が妨害されたことが必要であると解する。
	
	本問では、虚偽の通報による緊急出動によって、警察の通常の業務が現実に妨害されているといえる。
	
	\sectionB{}
	甲に故意は認められるか。故意とは、犯罪事実の認識及び認容をいうが、
	虚偽の通報をすることにより、緊急出動が行なわれ、虚偽通報がなければ行なわれたはずの業務が妨害されることは、当然に認識可能であるから、
	それにも関わらず、行為に出ることによってこれを認容していたといえる。
	
	以上により、甲には偽計業務妨害罪(刑法233条)が成立する。なお、30回の虚偽通報に対応する緊急出動それぞれに対して、偽計業務妨害罪が成立する。

\sectionA{甲がインターネット上の掲示板の書き込みにより、警察に人員を増やした警戒を強めたパトロールを行わせた行為につき、威力業務妨害罪(刑法234条)の成否を検討する。}

	
		警察は、甲のインターネット上の掲示板への書き込みによって、人員を増やして警戒を強めたパトロールを行うことを強いられている。
		甲の行為が行なわれていたはずの通常の業務は、非権力的公務であるから、本罪にいう「業務」にあたる。
		
		「威力」とは、人の自由意思を制圧するに足る勢力をいう。
		本問では、甲の執拗な書き込みによって、A市警察署の刑事課長Bが、実行の可能性がないとはいえないとの考えを有するに至っており、自由意思を制圧するに足る勢力が用いられたといえる。
		
		現実に、人員を増やしたパトロールが実施されており、その分本来の業務が行えなくなっているから、「妨害した」といえる。
		
		故意も認められるから、甲には威力業務妨害罪(刑法234条)が成立する。
		
\sectionA{甲の書き込みによるA市警察署の警察官に対する害悪の告知につき、脅迫罪(刑法222条1項)の成否を検討する。}

	甲の書き込みは、警ら中の警察官を畏怖させるに足る、生命、身体に対する害悪の告知を含んでおり、脅迫罪の構成要件に該当する。
	故意も認められるから、甲には、脅迫罪(刑法222条1項)が成立する。
	
\sectionA{甲が取調室おいて、Cの面前で携帯電話を床に投げつけて部品を飛び散らせた行為につき、公務執行妨害罪(刑法95条1項)の成否を検討する。}

	\sectionB{}
		Cによる任意の取調べは、強制力を行使する権力的公務ではなく、甲が携帯電話を床に投げつける行為は「威力」にあたりうるから、業務妨害罪の構成要件に該当するように見える。
		そこで、公務執行妨害罪との関係が問題となるが、公務は特に公共性の付与された業務であると解されるから、法条競合により、公務執行妨害罪の成否のみを検討する。
		
	\sectionB{}
		Cは警察官であるから、「公務員」である。
		
		「職務」とは、公務員の行う事務の全てをいう。任意の取調べは、「職務」にあたる。
		
		公務執行妨害罪の実行行為は暴行又は脅迫である。
		本罪は、公務員の身体の安全ではなく、公務員の職務の円滑な執行を保護法益とする罪であるから、暴行又は脅迫は、公務員に対して直接向けられたものに限られず間接的に向けられたものでもよい。
		
		もっとも、刑法95条1項は暴行・強迫が「これに対し」行なわれること、すなわち公務員に対して行われることを要求しているから、
		間接暴行が認められるのは、当該行為が公務員の面前で行われた場合等、公務員の身体に何らかの影響が及ぶ可能性がある場合に限られると解すべきである。		
		
		本問において、甲の行為は、Cから3メートルという近距離で行なわれてものであって、本罪の「暴行」にあたる。
	
		甲には故意も認められるから、以上により、公務執行妨害罪(刑法95条1項)が成立する。
		
\sectionA{罪数}
	以上より、甲には、\UTF{2460}30個の偽計業務妨害罪(刑法233条)、\UTF{2461}威力業務妨害罪(刑法234条)、\UTF{2462}脅迫罪(222条1項)、\UTF{2463}公務執行妨害罪(95条1項)が成立する。
	\UTF{2460}のそれぞれの罪は併合罪(刑法45条前段)となる。\UTF{2461}と\UTF{2462}は観念的競合(刑法54条)となり、これらと、\UTF{2460}、\UTF{2463}は併合罪(刑法45条前段)となる。
		
		

\begin{flushright}
	以上
\end{flushright}
	
\end{document}








