\documentclass[11pt]{jsarticle}

\usepackage[sect]{kian}
\usepackage{otf}
\usepackage{fancybox}
\usepackage{ascmac}
\usepackage[noalphabet]{pxchfon}  
\setminchofont{A-OTF-RyuminPro-Light.otf}
\setgothicfont{A-OTF-FutoGoB101Pr6N-Bold.otf}



\title{\vspace{-30mm}{\textgt{\Large{\fbox{15} 愛人への貢ぎ物}}}}
\date{\vspace{-15mm}}


\begin{document}

\maketitle

\sectionA{}
	有価証券偽造罪(刑法162条)の成否
	
	\sectionB{}
		甲が愛人のDに贈る時計の購入のために、B名義の200万円の小切手を作成し、使用した行為につき、
		有価証券偽造罪(刑法162条1項)、同行使罪(刑法163条1項)の成否を検討する。
	
			\sectionC{}
				小切手は、当然に有価証券に当たる。
				
			\sectionC{}
				偽造とは、「偽造」とは、名義人と作成者との人格の同一性を偽ることをいう。
				名義人とは、文書に表示された意思・観念の帰属主体として文書の形式・内容から認識される者をいい、
				作成者とは文書に表示された意思・観念の帰属主体をいう。
				本問において、小切手の名義人はA会社取締役Bである。では、作成者は誰か。
				
				ここで小切手に接した者は、通常、
				小切手を作成する権限のある者によって作成された小切手であるとの信用を持つ。
				小切手とは、決済の手段として用いられる有価証券であり、
				名義人本人が作成した小切手でない場合であっても、
				小切手を作成する権限を有する者が作成した小切手であれば、
				当該小切手は、名義人本人の、決裁に用いる意思に基づいて作成されたものと見ることができるからである。
				
				そこで、小切手を作成する権限の有無が問題となる。
				本問において、甲は会社の経理部長として、小切手の振り出しを行う一般的な権限を有していたといえる。
				しかし、そのような権限は、「会社の業務運営に必要な限り」で認められていたに過ぎず、
				愛人への時計の購入といった、
				会社の業務と全く関わりのない事柄についてまで小切手を作成する権限が与えられていたとはいえない。
				
				
				したがって、本件小切手に表示された決済意思を名義人に帰属させることはできない。
				よって、本件小切手の作成者は小切手の作成権限のない甲であり、
				名義人との人格の同一性に齟齬があるため、「偽造」に当たる。
			\sectionC{}
				行使の目的と故意も認められる。
		\sectionB{}
			以上より、甲には有価証券偽造罪(刑法162条1項)が成立する。
			甲は偽造した小切手を使用しているから、偽造有価証券行使罪(刑法163条1項)も成立する。

\sectionA{時計の購入に係る業務上横領罪(刑法253条)の成否}
	\sectionB{}
		甲が愛人のDに贈る時計の購入のために、A会社代表取締役B名義で200万円の小切手を振り出した行為は、
		背任罪(刑法247条)および業務上横領罪(刑法253条)の構成要件に該当するように見える。
		
		しかし、両罪の関係は法条競合であり、業務上横領罪が成立する場合には背任罪は成立しないから、
		重い業務上横領罪の成否を先に検討する。
	
		\sectionC{}
			業務上横領罪が成立するには、「業務上自己の占有する他人の物」を「横領」することが必要である。
			預金者であるA会社は、いつでも預金を引き出すことができ、預金の刑法上の所有権はAに帰属する。
			したがって、預金は甲にとって「他人の物」にあたる。では、預金について甲の占有が認められるか。
			
			窃盗罪の客体とならない物についても不法な領得による所有権侵害を処罰するという要請から、
			横領罪における占有は、事実的支配のみならず、法律的支配も含むと解される。。
			
			ここで預金につき、銀行に対する払い戻し権限を有する者は、
			委託された金銭をそのまま占有している者と金銭の保管者である地位であることに変わりないから、
			「預金の占有」を認めることができる。
			
			本問において、甲は、小切手の振り出し権限を有しているから、
			預金について甲の占有が認められる。
		
		\sectionC{}
			もっとも、横領罪の保護法益は所有権及び委託関係であり、業務上横領罪はその加重類型であるから、
			業務上横領罪が成立するには、「占有」が「業務上」の委託に基づいて行なわれる必要がある。
			
			業務上横領罪における業務とは、社会生活上の地位に基づいて、反復継続して行なわれる事務のうち、
			物を管理することを内容とする事務をいう。
			
			甲はA会社の経理部長であるから、A会社の預金を業務上の委託に基づいて占有していたといえる。
			
		\sectionC{}
			次に、甲は「横領した」といえるか。
			「横領」とは、不法領得の意思を発現する一切の行為をいう。
			
			横領罪における不法領得の意思とは、委託の趣旨に反して、権限がないのに、
			その物の効用に基づいて、所有者でなければできないような処分をする意思をいう。
			判例における不法領得の意思の定義では、利用処分意思が要求されていないが、
			横領罪の利欲犯としての性格から器物損壊罪よりも重く処罰されているのだから、利用処分意思を要求すべきである。
			
			愛人に贈る時計の代金支払のために小切手を振り出す行為は、経理部長としての委託の趣旨に反する無権限の行為であり、
			かつ、Aの預金を使用して、その効用に基づきDに時計を取得させる意思が認められるから、不法領得の意思が認められる。
			
			したがって、甲の行為は「横領」にあたる。
			なお、領得目的で小切手を交付した場合、その時点でA会社の預金に対する具体的な危険が生じているから、
			実際にCの口座に代金が振り込まれていなくても、小切手の交付があった時点で「横領した」といえる。
			
		\sectionC{}
			甲には故意および、上述のように不法領得の意思も認められる。
			
	\sectionB{}
		以上より、甲には業務上横領罪(刑法253条)が成立する。
			
\sectionA{借金の返済に係る業務上横領罪(刑法253条)の成否}
	前述の通り、甲はAの預金を業務上占有する者である。また、自身の借金返済のため、300万円の小切手を呈示提出する行為は、
	委託の趣旨に反する無権限の行為であり、かつ、自身の借金返済のためという不法領得の意思も認められるから、「横領した」といえる。
	加えて故意も認められるから、業務上横領罪(刑法253条)が成立する。
	
\sectionA{罪数}
	以上より、甲には、\UTF{2460}有価証券偽造罪(刑法162条1項)、\UTF{2461}偽造有価証券行使罪(刑法163条1項)、
	\UTF{2462}時計の購入に係る業務上横領罪(刑法253条)、\UTF{2463}借金の返済に係る業務上横領罪(刑法253条)が成立する。
	
	\UTF{2460}と\UTF{2462}は手段と目的の関係にあるので、牽連犯(刑法54条1項後段)となり、これと\UTF{2462}、\UTF{2463}は併合罪(刑法45条前段)となる。





\begin{flushright}
	以上
\end{flushright}
	
\end{document}








