\documentclass[11pt]{jsarticle}

\usepackage[sect]{kian}
\usepackage{okumacro}
\usepackage{fancybox}
\usepackage[noalphabet]{pxchfon}  
\setminchofont{A-OTF-RyuminPro-Light.otf}
\setgothicfont{A-OTF-FutoGoB101Pr6N-Bold.otf}



\title{\vspace{-30mm}{\textgt{\Large{\fbox{7} 男の恨みは夜の闇より深く}}}}
\date{\vspace{-15mm}}


\begin{document}

\maketitle

\sectionA{Aに対する暴行について}
	甲、乙間では、Aに対して暴行を加える旨の共謀が成立している。また、共謀に基づいて甲がAの顔面を強打する行為、乙がAを取り押さえる行為は、Aの身体に対する有形力の行使として暴行罪(刑法60条、刑法208条)の共同正犯の構成要件に該当する。

	ここで、Aには加療4週間程度を要する傷害結果が発生しているから、甲、乙に暴行罪の結果的加重犯としての傷害罪の共同正犯が成立するか。
	
	結果的加重犯であっても、基本犯につき共同正犯が成立し、加重結果を共同して惹起したと評価できる以上、結果に対する因果性は肯定できるから、共同正犯の成立を認めてよい。
	
	したがって、甲、丙には、傷害罪の共同正犯(刑法60条、刑法204条)が成立する。
\sectionA{Aのハンドバッグを持ち去った行為について}
	\sectionB{}
		Aは死亡していないから、Aのハンドバッグに対する占有が認められる。甲、乙は当該ハンドバッグを持ち去っているから、窃盗罪(刑法235条)の客観的構成要件に該当する。
	\sectionB{}
		では、窃盗罪の故意は認められるか。甲、乙は、Aが死亡していると誤信しており、誤信した内容を前提とすると、死者には占有の事実も占有の意思も観念できない以上、占有が認められないことから問題となる。
		
		占有が認められなければ、当該財物は窃取の客体となり得ないから、窃盗罪は成立しない。しかし、犯人が自ら被害者を殺害し占有を失わせることにより、被害者の占有する財物を占有離脱物に変えたのに、そのことにより後行行為の罪責が軽くなり、財物の占有が保護されなくなってしまうのは不当である。そこで殺害行為(占有離脱行為)と取得行為の一体性が認められる場合には、当該行為者との関係においては、なお生前の占有が存続しているものとして罪責評価を行うべきである。
		
		本問において、甲、乙の認識を前提とすると、暴行(占有離脱行為)とハンドバッグの持ち去りは時間的・場所的に近接して行なわれている。また、暴行によりAが転倒し動かなくなった状態を利用する意思でハンドバッグ持ち去っており、主観的な連続性も認められる。したがって、両行為には一体性が認められるから、Aの占有が存続しているものとして扱われる。以上より、ハンドバッグの持ち去りは「窃取」と評価され、甲、丙はこれらの行為を認識・認容しているから、窃盗罪の故意が認められる。
		
		\sectionB{}次に、不法領得の意思は認められるか。甲はハンドバッグを焼却する目的、乙は後に売却する目的で持ち去っており、その意思内容が異なることから問題となる。不法領得の意思とは、権利者を排除して他人の物を自己の所有物としてその経済的用法に従い利用、処分する意思をいう。(権利者排除意思は、軽微な一時使用と窃盗罪を区別しながら、犯罪の成否を占有取得時に確定させ、既遂時期の明確性を担保するために要求される。利用処分意思は、窃盗罪等の領得罪を毀棄罪から区別するために要求される。もっとも、厳密な経済的用法に従う必要はなく、財物それ自体から直接的に何らかの効用を享受する意思があれば足りる。)
		
		不法領得の意思は主観的な要件であるから、甲、乙についてそれぞれ判断する必要がある。
		
			\sectionC{甲について}
				足がつかないように焼却して処分する目的には、利用処分意思がないため、甲に不法領得の意思は認められず、窃盗罪は成立しない。
			\sectionC{乙について}
				ハンドバッグを持ち去ることで、Aの利用可能性を排除する意思が認められるから、権利者排除意思が認められる。また、ハンドバッグを売却することで直接、金銭的な利益を得ることができるから、利用処分意思もあり、不法領得の意思が認められる。
				
				以上より、乙には窃盗罪が成立する。
\sectionA{Bに対する暴行について}
	\sectionB{甲の罪責}
		前述の通り、甲は主観的構成要件を満たさず窃盗罪が成立しないから、事後強盗罪の故意は認められない。そこで、傷害罪(刑法204条)の成否を検討する。
		
		甲は乙と共にBに激しい暴行を加えており、これは、暴行罪(刑法208条)の構成要件に該当する。また、少なくとも暴行の故意が認められ、Bには加療3週間程度の打撲傷が生じているから、甲には暴行罪の結果的加重犯としての傷害罪(刑法204条)が成立する。
	\sectionB{乙の罪責}
		乙は窃盗の後、Bから逃げようとして暴行を加えて傷害を負わせているから、事後強盗罪(刑法238条)、結果的加重犯としての強盗致傷罪(刑法240条)を検討する。
		\sectionC{}
			第3(2)の通り、乙には窃盗罪が成立するから、乙は「窃盗」にあたる。
		\sectionC{}
			乙は、甲と共同してBに対して暴行を加えているが、事後強盗罪は「強盗として論」じられるから、事後強盗罪における暴行は、強盗罪におけるそれと同じ程度、すなわち、反抗を抑圧する程度の暴行であることを要する。
			
			本問において、暴行の態様は枢要部である顔面や胸部に対して激しく行なわれており、また、Bが1人であるのに対して、甲と乙は共同して暴行を加えている。さらに、時間帯は夜間であり第三者に助けを求めることが容易な状況ではなかった。
			
			これらのことから、Bに加えられた暴行は反抗を抑圧する程度の暴行と評価できる。
		\sectionC{}
			次に、乙は捕まえようと追跡してきたBから逃れるために暴行を加えているから、「逮捕を免れ」る目的が認められる。
		\sectionC{}
			乙はこれらの事実を認識・認容しているから、故意も認められる。
				
			したがって、乙には事後強盗罪(刑法238条)が成立する。
		\sectionC{}
			そして、Bは暴行によって傷害を負っているから、乙には強盗致傷罪(刑法240条)が成立する。
	\sectionB{共同正犯の成否}
		Bに暴行を加えた時点で、Bから逃れるために共同して暴行を加える旨の共謀が成立していたと解される。そして、甲、乙は共謀に基づいてBに対して暴行を加えている。
		
		もっとも、前述の検討の通り、甲には傷害罪の故意しかないにも関わらず、乙には強盗致傷罪が成立することから、共同正犯の成立する範囲が問題となる。
		共同正犯の本質は、特定の犯罪を共同して実行することに求められるから、異なる構成要件間での共同正犯は成立しない。もっとも構成要件が同質的で重なり合う場合にはその範囲で共同正犯の成立を認めることができ、また、重い故意を有するものには別に単独正犯が成立する。
		
		強盗致傷罪と傷害罪は軽い傷害罪の限度で重なり合いが認められるため、その範囲で共同正犯の構成要件該当性が認められる。
		
		以上より、甲には傷害罪の共同正犯(刑法60条、刑法204条)、乙には強盗致傷罪、傷害罪の共同正犯(刑法60条(ただし傷害罪の限度で)、240条)が成立する。
\sectionA{罪数}
	\sectionB{}
		以上より、甲には\MARU{1}Aに対する傷害罪の共同正犯(刑法60条、刑法240条)、\MARU{2}Bに対する傷害罪の共同正犯が成立する。これらは併合罪(刑法45条前段)となる。
	\sectionB{}
		乙には、\MARU{3}Aに対する傷害罪の共同正犯、\MARU{4}窃盗罪(刑法235条)、\MARU{5}強盗致傷罪、傷害罪の共同正犯(刑法60条(ただし傷害罪の限度で)、240条)が成立する。\MARU{4}は\MARU{5}に吸収される。\MARU{3}と\MARU{5}は併合罪(刑法45条前段)となる。
	
\begin{flushright}
	以上
\end{flushright}
	
\end{document}








