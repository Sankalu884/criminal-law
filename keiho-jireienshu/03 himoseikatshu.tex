\documentclass[11pt]{jsarticle}

\usepackage[sect]{kian}
\usepackage{okumacro}
\usepackage{fancybox}
\usepackage[noalphabet]{pxchfon}  
\setminchofont{A-OTF-RyuminPro-Light.otf}
\setgothicfont{A-OTF-FutoGoB101Pr6N-Bold.otf}



\title{\vspace{-30mm}{\textgt{\Large{\fbox{3} ヒモ生活の果てに}}}}
\date{\vspace{-15mm}}


\begin{document}

\maketitle

\sectionA{甲の罪責}

	\sectionB{}
	甲がBを殴打して死亡させた行為について、傷害致死罪(刑法205条)の成否を検討する。
	
		\sectionC{}
		甲は、Bを殴打することで「傷害」を発生させ、Bは「死亡」している。
		
		\sectionC{}
		では、Bはこの傷害に「よって」死亡したといえるか。
		本問では、行為と結果との間に事実的なつながりは存在するが、同時に必要な救命措置をとらなかったという介在事情も存在するため、法的因果関係を認めてよいかが問題となる。
		
		法的因果関係は、偶然的結果を排除し適正な帰責範囲を画するために要求される。
		これは、客観的に存在する全ての事情を判断資料とし、実行行為の結果が危険へと現実化したか否かによって判断される。
		
		甲の暴行は、Bの頬を平手で殴打した上、人体の枢要部である頭部を5回にわたって殴打するという激しい態様のものであり、
		かつ、死亡の原因となった硬膜下出血、くも膜下出血等の傷害を発生させており、それ自体として死亡の結果を惹起させるような危険性を有している。
		確かに、その後必要な救命措置をとらないという不作為が介在しているものの、不作為は新たな危険を創出するものではなく、
		結果に至る因果の流れを阻止しないに過ぎないから、結果に対する寄与度は小さい。
		したがって、Bの死亡は、甲の暴行による危険が直接実現した結果といえるから、甲の暴行とBの死亡の間の法的因果関係が認められる。
		
		\sectionC{}
		また、甲には殺人の故意はないが、少なくとも暴行の故意が認められるので、甲には傷害致死罪が成立する。
	
	\sectionB{}
	甲がBに必要な救命措置をとらず、Bを死亡させた行為について殺人罪(刑法199条)の成否を検討する。
	
		\sectionC{}
		刑法199条は「人を殺」す行為について規定しているが、不作為義務に反して作為をすることと、
		作為義務に反して作為をしないことは同価値であるから、
		そうした不作為によって構成要件的結果を惹起したのであれば、当該不作為は構成要件で予定される態様として考えることができるので、
		不作為を処罰することも可能である。
		
		作為義務の有無は、作為との同価値性の観点から判断される。
		作為による法益侵害は、結果に至る因果の流れを設定し、これを実質的に支配することに特徴がある。
		したがって、作為義務を認めるためには、不作為者による結果原因の支配が認められることが必要である。
		加えて、偶然的に支配を獲得した場合を除くため、保護の引受け、先行行為や客体との社会的関係性等を考慮することが必要である。
		
		本問において、Bはまだ2歳と幼く、乙宅での出来事に第三者が介入することは容易でないから、
		Bの法益は実質的に甲に委ねられているといえる。
		また、甲の暴行によってBの法益に対する危険が創出されていること、甲はBの同居の親権者であることから、Bの法益保護につき作為義務を課すことを正当化できる。
		
		したがって、甲には、Bに必要な救命措置をとる作為義務が認められる。
		
		また、救急車を要請するなど、作為義務を履行することは容易であるから、作為可能性も認められる。
		
		にも関わらず、Bを放置し、救命措置をとらなかった行為は、作為義務に違反する行為である。
		
		\sectionC{}
		では、作為義務違反と結果との間の因果関係は認められるか。
		不作為の条件関係は、作為義務を履行すればほぼ確実に結果を回避できたといえるときに認められる。
		
		\sectionC{}
		本問において、「Bが傷害を受けた時点ですぐに病院に連れて行って治療を受けさせていれば、
		Bの救命は確実であった」というのであるから、条件関係が認められる。
		そして、Bの死亡結果は、甲の作為義務違反行為の危険が直接現実化したものといえるから、法的因果関係も認められる。
		
		\sectionC{}
		甲は、Bの死亡結果を認識しつつ、乙との関係が上手くいくと思い、これを認容しているから殺人罪の故意が認められる。
		
		以上より、甲には、殺人罪(刑法199条)が成立する。
		なお、後述の通り、乙と保護責任者遺棄致死罪(刑法219条)の限度で共同正犯となる。
\sectionA{乙の罪責}
	\sectionB{乙が甲の暴行を阻止しなかった行為について}
		\sectionC{}
		乙は、Bが泣くと甲がBに暴行を加えることを認識しており、現に甲がBの頬を叩いているのを横目で見ていたから、
		甲の暴行によって傷害結果が生じる可能性を認識・認容していたといえる。
		そこで、傷害致死罪(刑法205条)の成否を検討する。
		
		もっとも、Bに対する暴行につき、甲乙間で意思の連絡はないから、乙に作為犯は成立しないため不作為犯の成否を検討する。
		
		\sectionC{}
		ここで、Bの死亡結果を直接惹起したのは、甲の作為(暴行)であり、乙が甲を阻止しなかったという不作為は結果に対して2次的な影響しか持たないから、
		不作為による傷害致死罪の幇助(刑法62条)の成否が問題となる。
		
		では、乙に作為義務は認められるか。
		
		甲の暴行は、乙宅におけるものであり、かつ、乙を気づかって行なわれたものであって、
		乙以外に甲を制止できる者はおらず、結果原因の支配が認められる。
		また、乙はBの親ではないが、Bと同居しており、まだ幼く脆弱なBの法益につき作為義務を課すことを正当化できる。

		したがって、乙には、甲の暴行を阻止する作為義務が認められる。
		
		\sectionC{}
		甲の暴行は乙を気づかって行なわれているものであるから、甲を阻止することは容易であり、作為可能性が認められる。
		
		にも関わらず、甲を阻止しなかった行為は、作為義務に違反する行為である。
		
		\sectionC{}
		甲を阻止していれば、甲の行為を阻止することができたといえるから、因果関係も認められる。
		
		以上より、乙には傷害致死罪の幇助が成立する。
	
	\sectionB{乙がBに必要な救命措置を取らなかった行為について}
	乙はBについて、乙は「死ぬほどの状態ではないだろう」と思っており、死亡結果の認識・認容に欠けるから、殺人罪は成立しない。
	そこで、保護責任者遺棄致死罪の共同正犯(刑法60条、刑法219条)の成否を検討する。
	
	共同正犯の成立要件は、\MARU{1}共謀と\MARU{2}共謀に基づく実行である。
	
		\sectionC{}

		甲は、乙がBを病院に連れて行くことを拒んだ後、これに同意しているから、
		Bの「生存に必要な保護をし」ないことにつき合意を形成しており、共謀が認められる。
		
		\sectionC{}
		次に、共謀に基づいて行った行為が、保護責任者遺棄罪の実行といえるかが問題となる。
		
			\sectionD{}
			Bは3歳の「幼年者」であり、かつ、甲の暴行により傷害を負っているから「病者」にもあたる。
			
			\sectionD{}
			では、乙は保護責任者といえるか。
			乙はBと同居しており、Bを病院に連れて行くことを提案して保護を引き受けている。
			また、Bを病院に連れて行く、救急車を要請するなど、Bの保護することは容易にできるといえるから、乙は保護責任者であるといえる。
			
			そして、傷害を負って気絶しているBを放置することは、死亡の危険がある行為であり、「生存に必要な保護をしなかった」といえる。
			
			したがって、共謀に基づく実行も認められる。
	
		\sectionC{}
		Bの死亡は、共謀に基づく実行の危険が直接結果に実現したものといえるから、因果関係も認められる。
		
		\sectionC{}
		もっとも、甲には殺意があり、上述の通り殺人罪が成立するから、どの範囲で共同正犯となるかが問題となる。
		
		共同正犯の本質は、特定の犯罪を共同して実行することに求められるから、異なる構成要件間での共同正犯は成立しない。
		もっとも構成要件が同質的で重なり合う場合にはその範囲で共同正犯の成立を認めることができる(部分的犯罪共同説)。
		
		殺人罪と保護責任者遺棄致死罪は保護責任者遺棄致死罪の限度で重なり合いが認められるから、その範囲で構成要件該当性が認められる。
		
	
		なお、保護責任者遺棄致死罪は保護責任者遺棄罪の結果的加重犯であり、結果的加重犯であっても共同正犯が成立するかは問題となりうるが、
		基本犯について共同正犯が成立し、加重結果を共同で惹起したと評価できる以上、結果に対する因果性は肯定できるから、
		結果的加重犯についても共同正犯の成立を認めてもよいと解される。
		なお、構成要件の重なり合いが認められず、共同正犯に解消できない部分については、単独犯が成立する。
		
		以上より、乙には保護責任者遺棄致死罪の共同正犯が成立し(刑法60条、刑法219条)、
		甲には殺人罪、保護責任者遺棄致死罪の限度で共同正犯(刑法60条(ただし保護責任者遺棄致死罪の限度で)、199条)が成立する。

\sectionA{罪数}
以上より、甲には、\MARU{1}傷害致死罪、\MARU{2}殺人罪、保護責任者遺棄致死罪の限度で共同正犯が成立するが、
死の結果を二重に評価することは妥当でないので、\MARU{1}の致死部分は\MARU{2}に吸収されると解した上で、両者は併合罪となる。

乙には、\MARU{3}傷害致死罪、\MARU{4}保護責任者遺棄致死罪の共同正犯が成立するが、甲と同様、\MARU{3}の致死部分は\MARU{4}に吸収されて、両者は併合罪となる。	
	






\begin{flushright}
	以上
\end{flushright}
	
\end{document}








