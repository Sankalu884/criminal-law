\documentclass[11pt]{jsarticle}

\usepackage[sect]{kian}
\usepackage{otf}
\usepackage{fancybox}
\usepackage{ascmac}
\usepackage[noalphabet]{pxchfon}  
\setminchofont{A-OTF-RyuminPro-Light.otf}
\setgothicfont{A-OTF-FutoGoB101Pr6N-Bold.otf}



\title{\vspace{-30mm}{\textgt{\Large{\fbox{2} D子は見ていた}}}}
\date{\vspace{-15mm}}


\begin{document}

\maketitle

\sectionA{}
	甲がAの財布を持ち去った行為について窃盗罪(刑法235条)の成否を検討する。
	
	\sectionB{}
		窃盗罪が成立するためには、故意と不法領得の意思の下、「他人の財物」を「窃取」することが必要である。
		
		\sectionC{}
			まず、「他人の財物」とは、他人の所有物を意味する。本件財布はCの所有物であるから、これに当たる。
			
		\sectionC{}
			次に、「窃取」とは占有者の意思に反する占有の移転をいうから、その前提として、窃取の対象となる「他人の財物」が他人の占有下にあることが必要である。
			
			占有とは、財物に対する事実上の支配を意味し、その有無は、占有の事実と占有の意思を総合して、社会通念に従って判断する。
			
				\sectionD{}
					では、本問において、Aの占有は認められるか。
					
					甲が財布を取得したのは、Aがベンチに置きわすれてから、約2分後のことであり、時間的近接性は認められる。	
					
					しかし、財布は比較的小さく、第三者による取得が容易である。また、財布が置かれた場所はスーパーマーケット内のベンチ上である。スーパーマーケットは不特定多数の者が出入りするから、排他性は弱い。さらに、甲が財布を取得した時点で、Aは6階におり、置き忘れに気付いてから即時の回復は困難であり、見通し状況も悪い。
					
					以上より、Aは財布の現実的支配を喪失してから、短時間のうちに回復できる客観的状況にないから、占有の事実が認められない。したがって、本件財布に対するAの占有は認められない。
					
				\sectionD{}
					もっとも、財布はスーパーマーケットB内のベンチ上に置かれているから、Bの占有が及ぶのではないか。
					
					旅館の客室内など、第三者の立ち入りが容易でなく、閉鎖性排他性の高い場所に置きわすれた場合、財物の所有者からその場所の管理者に占有が移ることは考えられる。しかし、前述の通りスーパーマーケットは不特定多数の者が出入りすることが予定されており、閉鎖性排他性が弱いから、Bの占有は及ばない。

				\sectionD{}
					では、Dに占有は認められるか。Dは財布を注視していたことから問題となる。
				
					確かに、DはAの置き忘れた財布を注視していたが、それ以上に第三者による取得を排除しようとする意思はなく、また、現実に甲の持ち去りを見ていただけであり、第三者による取得を妨害する措置も講じていないから、占有の事実も認められない。したがって、Dに占有は認められない。

		\sectionC{}
			以上より、財布にはいずれの占有も及んでいないから、「窃取」の客体となり得ず、窃盗罪は成立しない。
	\sectionB{}
		そこで、遺失物等横領罪(刑法254条)の成否を検討する。
		\sectionC{}
			誰の占有も及ばないAの財布を奪取する行為は、同罪の客観的構成要件に該当する。
		\sectionC{}
			もっとも甲は、本件財布をCの所有物だと思って財布を奪取している。現実にCの所有物であった場合、甲が財布を奪取した時点、すなわち、Cが財布から約3メートル離れた自販機近くにいる時点では、本件財布に対するCの占有があるといえる。甲はこれらの事実を認識しているから、窃盗罪の故意が認められる。
			
			問題は、客観的に実現した遺失物等横領罪に対応する同罪の主観的要件が充足するかどうかである。
			\sectionD{}
				構成要件に軽い罪の限度で実質的な重なり合いが認められれば、その限度で故意を認めることができる。構成要件の重なり合いの有無は、構成要件の実体が法益侵害行為の類型であることから、行為態様と保護法益の共通性によって判断する。
				
				遺失物横領罪の保護法益は財物に対する「所有権」であり、窃盗罪の保護法益は、財物に対する「占有及び所有権」であるから、両者は「所有権」の限度で重なり合いが認められる。また、両罪の実行行為も、他人の財物を不正に取得する点で共通する。したがって、両罪の構成要件は軽い遺失物横領罪の限度で実質的に重なり合っているといえるから、甲には、遺失物等横領罪の故意が認められる。
				
			\sectionD{}
				また、財布を奪取していることから、当然に不法領得の意思も認められる。
		\sectionC{}	
			以上より、遺失物等横領罪の主観的要件も充足するから、甲には、同罪が成立する。
			
\sectionA{}
	甲がクレジットカードを不正に使用して、商品を購入した行為について詐欺罪(刑法246条1項)の成否を検討する。
	\sectionB{}
		詐欺罪が成立するためには、「人を欺いて」錯誤を生じさせ、その錯誤に基づいて「財物」を交付させることが必要である。
		\sectionC{}
			甲がAのカードを提示した行為は、「人を欺」く行為、すなわち、欺罔行為といえるか。欺罔行為とは、前述の通り、財物の交付に向けて錯誤を惹起する行為であり、その内容は、交付の判断の基礎となる重要な事項について偽ることをいう。甲の行為は、財物の交付に向けられたものであるといえる。
			\sectionD{}
				では、「甲がクレジットカードの名義人であるAと同一人物であること」は交付の判断の基礎となる事項といえるか。
				クレジットカードの加盟店は、カードの会員と利用者の同一性を確認する義務を負っており、カードの会員でない者の決済に応じることはない。したがって、甲とAが同一人物ではないことが分かれば、商品を販売しなかったといえる。したがって、「甲がクレジットカードの名義人であるAと同一人物であること」は交付の判断の基礎となる事項といえる。
			\sectionD{}
				次に、そのような事項は、「重要な事項」といえるか。「重要な事項」とは、詐欺罪は財産犯であるから、詐欺罪の法益侵害性を基礎づける事項、すなわち、財産的損害に係る事項をいう。
				
				加盟店は、過失によって同一性の確認を怠った場合、立替え払いを受けることができない危険を負担する。したがって、錯誤によってこのような危険を負担した状態で商品を交付したことに、実質的な損害を認めることができるから、「甲がクレジットカードの名義人であるAと同一人物であること」は「重要な事項」といえる。
		\sectionC{}
			以上より、甲の行為は欺罔行為に当たる。
	\sectionB{}
		そして、Fは、甲がA本人であると誤信して、商品を交付し、甲はそれを受領している。故意と不法領得の意思も認められるから、甲には詐欺罪(刑法246条1項)が成立する。
		
\sectionA{}
	甲が売上伝票にAと署名した行為について私文書偽造罪(刑法159条1項)及び、同行使罪(刑法161条1項)の成否を検討する。
	\sectionB{}
		私文書偽造罪について
		\sectionC{}
			売上票は、それに基づいてクレジット決済に係る債権債務関係が発生するから、「権利、義務に関する文書」である。
		\sectionC{}
		また、「偽造」とは、文書の名義人と作成者の人格と同一性を偽ることをいう。名義人とは、文書から作成者と認識される者をいい、作成者とは、文書に表示された意思観念の帰属主体を言う。本問において、名義人は署名のあるA本人であり、作成者は甲であるから、人格の同一性に齟齬が生じており、「偽造」にあたる。
		\sectionC{}
			また、甲はAと署名しており、「他人の署名を使用し」たといえる。甲には、売上伝票を提出して決済を行う意図があるから、「行使の目的」が認められる。故意も認められる。
したがって、甲には私文書偽造罪(刑法159条1項)が成立する。
	\sectionB{}
		偽造私文書行使罪について
	
		甲は	、売上票をF交付して、その内容を認識可能な状態においているから、偽造文書を「行使」したといえる。
		
		したがって、甲には偽造私文書行使罪(刑法161条1項)が成立する。

\sectionA{}
	甲には、\UTF{2460}遺失物等横領罪(刑法235条、刑法254条)、\UTF{2461}1項詐欺罪(刑法246条1項)、\UTF{2462}私文書偽造罪(刑法159条1項)、\UTF{2463}偽造私文書行使罪(刑法161条1項)が成立する。
	
	ここで、\UTF{2461}と\UTF{2462}、\UTF{2462}と\UTF{2463}は目的手段の関係にあるから、牽連犯(刑法54条1項後段)となり、これと\UTF{2460}は併合罪(刑法45条)となる。
		




\begin{flushright}
	以上
\end{flushright}
	
\end{document}








