\documentclass[11pt]{jsarticle}

\usepackage[sect]{kian}
\usepackage{otf}
\usepackage{fancybox}
\usepackage{ascmac}
\usepackage[noalphabet]{pxchfon}  
\setminchofont{A-OTF-RyuminPro-Light.otf}
\setgothicfont{A-OTF-FutoGoB101Pr6N-Bold.otf}



\title{\vspace{-30mm}{\textgt{\Large{\fbox{20} クリスマスイブの事件}}}}
\date{\vspace{-15mm}}


\begin{document}

\maketitle

\sectionA{甲がA殺害のため、睡眠薬を飲ませて眠らせた行為につき、殺人未遂罪(刑法203条、199条)の成否を検討する。}

	\sectionB{}
		甲の計画では、Aに睡眠薬を飲ませて眠らせた後、放火して殺害する予定であったが、実際には放火を中止しているため、Aに睡眠薬を飲ませた時点で殺人罪の実行の着手が認められるかが問題となる。
		
		実行の着手は、刑法43条本文の「実行に着手して」という文言から、構成要件該当行為又はこれと密接に関連する行為が開始され、
		かつ、未遂犯の処罰根拠は、結果発生の具体的危険の惹起に求められるから、結果発生の具体的危険性が発生することが必要である。
		結果発生の危険性は、行為者の意思によって影響を受けるから、危険性の判断に当たっては、行為者の犯行計画も考慮する。
		
		本問において、放火してAを殺害するという計画の下では、Aに睡眠薬を飲ませて眠らせることは必要不可欠な行為であり、
		また、Aを眠らせてしまえば、放火して殺害するという計画遂行上、傷害となる特段の事情は存在しない。
		さらに、Aを眠らせる行為と放火行為との間には時間的場所的近接性が認められる。
		
		これらのことからすると、Aに睡眠薬を飲ませて放火するという計画をも考慮すれば、Aに睡眠薬を飲ませて眠らせる行為は、
		殺人罪の実行行為(放火行為)に密接な行為であり、放火行為を経てAの死亡結果を惹起させる具体的な危険性が認められる。
		したがって、殺人罪の実行の着手が認められる。
	
	\sectionB{}
		そして、このように睡眠薬を飲ませてから放火するという計画上の一連の行為は、一連の殺人の実行行為といえるから、睡眠薬を飲ませた時点で、既に実行行為を行っている認識が認められる。保険金詐欺目的であるから、認容も認められ、故意が肯定できる。
	
	したがって、甲には殺人未遂罪(刑法203条、199条)が成立する。

\sectionA{中止犯の成否}
	もっとも甲は、急にAがかわいそうになり、放火を中止しているから、中止犯(刑法43条ただし書き)の成否が問題となる。
	\sectionB{}
		中止犯は、「自己の意思により」(任意性)、「犯罪を中止した」(中止行為)ときに成立する。
		\sectionC{}			
			中止行為といえるためには、当該行為が、発生した具体的危険を消滅させるに足る行為でなければならない。
			A死亡の具体的危険は、甲が放火を取りやめれば消滅するから、上記一連の実行行為を継続しないという不作為で足りる。
			また、因果関係も認められる。
		\sectionC{}
			では、任意性は認められるか。任意性は責任要素であるから、行為者の主観を評価して判断されるが、その判断は、行為者の認識した事情が経験上一般に犯行の障害となるものか否かによってなされる。
			
			本問において、睡眠薬を飲まされたAが昏睡状態に陥り、呼吸が浅くなることはむしろ計画通りであり、甲の認識した事情は経験一般に犯行の障害となるものとはいえない。したがって、任意性が認められる。
			
			以上により、甲には殺人罪(刑法199条)につき中止犯(43条ただし書き)が成立し、刑の必要的減免を受ける。
			
\sectionA{}
	甲が灯油をまいて火をつけようとした行為につき、現住建造物等放火未遂罪(刑法112条、108条)の成否を検討する。
	
	\sectionB{}
		甲が火をつけようとした自宅は、甲とAが現に生活している建造物であるから、現住建造物にあたる。
		では、甲の行為に放火罪の実行の着手が認められるか。
		
		灯油はガソリン等とは異なり、揮発性の低い燃料であって、灯油をまく行為はそれ自体として焼損の具体的危険を発生させるものとは言い難い。焼損の具体的危険が発生するのは、灯油に着火するため媒介物に火をつけようとする直前の時点であるから、本問のように、灯油をまいた上、ライターと媒介物である新聞紙を取り出した時点では、焼損の具体的危険が発生しているとはいえず、実行の着手は認められない。
		
		以上より、甲の行為は、予備罪(刑法113条)としての罪責を負うにとどまる。
		
		\sectionB{}
			ここで、甲は放火を中止しているから、予備罪について、中止未遂の規定(刑法43条ただし書き)の準用があるかが問題となるが、予備罪の法定刑には刑の免除を認めるものとそうでないものとが区別されており、中止犯に対する対応は予備罪の規定の内部に既に織り込み済みだと解される。したがって、準用を認める必要はない。

\sectionA{}
	甲がAを殺害して、保険金をだまし取ろうとした行為につき、詐欺未遂罪(刑法250条、246条)の成否を検討する。
	
	問題は、詐欺罪の実行の着手が認められるかであるが、相手を欺罔して財物を交付させ、これを受領するという詐欺罪の結果の具体的危険が発生するのは、欺罔行為に着手した時点であると考えられる。
	
	本問において、甲はAに睡眠薬を飲ませて眠らせたにとどまり、いまだ保険金を請求する(欺罔行為)に至っていないから、詐欺罪の実行の着手は認められない。したがって、詐欺未遂罪(刑法250条、246条)は成立しない。なお、詐欺罪に予備の規定はない。
	
	\sectionA{罪数}
		以上より、甲には、\UTF{2460}殺人罪の中止未遂(刑法203条、199条、43条ただし書き)、\UTF{2461}現住建造物放火予備罪(刑法113条、108条)が成立する。\UTF{2460}と\UTF{2461}は併合罪(刑法45条前段)となる。
\begin{flushright}
	以上
\end{flushright}
	
\end{document}








