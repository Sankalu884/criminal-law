\documentclass[fontsize=11pt,
jlreq_notes
]{jlreq}

% 1. フォント・日本語処理
\usepackage{pxchfon} 
\usepackage{otf}

% 2. フォント設定
\setminchofont{A-OTF-RyuminPro-Light.otf}
\setgothicfont{A-OTF-FutoGoB101Pr6N-Bold.otf}

% 3. 汎用パッケージ
\usepackage{fancybox}

% 4. 自作パッケージ
\usepackage[sect]{kian}


\title{\vspace{-30mm}{\textgt{\Large{\fbox{28} 元風俗嬢の憤激}}}}
\date{\vspace{-15mm}}


\begin{document}

\maketitle

\sectionA{甲の罪責}
	\sectionB{甲がAの右腰部を包丁で突き刺した行為の罪責}
		甲の行為は傷害罪(刑法204条)の構成要件に該当するが、Aの暴行を契機としてなされているため、正当防衛(刑法36条1項)として違法性が阻却されないかが問題となる。
		\sectionC{}
			Aの甲に対する暴行は、不正の侵害であり、当該暴行はその後も継続することが予想される状況であったから、急迫不正の侵害であったといえる。
		\sectionC{}
			では、甲の行為は、「防衛するため」の行為といえるか。
			
			防衛行為性を認めるためには、「防衛するため」という文言から防衛の意思が必要である。
			
			防衛の意思とは、急迫不正の侵害を認識し、これを避けようとする単純な心理状態をいう。憤激や攻撃の意思が併存していても、ただちに防衛の意思が否定されることはないが、専ら攻撃の意思で反撃を加える行為には、防衛の意思が認められず、正当防衛を否定すべきである。
			
			本問において、甲がAから、菜箸をつかんで目の前に突きつけられて「あごをぶち抜いて,目ん玉ぶち抜いてやる」「ぶっ殺してやる」などと言われたのに対し、恐怖を感じて反撃に転じていることからすれば、甲は主として防衛のためにこれを行ったと見るのが相当であり、防衛の意思に基づく、「防衛するため」の行為と認められる。
		\sectionC{}
			甲の行為は「やむを得ずにした行為」といえるか。
			
			やむを得ずにした行為とは、反撃行為が防衛手段として必要最小限度のものであること、すなわち、反撃行為が防衛手段として相当性を有するものであることをいう。
			必要最小限度か否かは、可能な防衛手段の選択肢及び態様を考慮した上で具体的に判断する。
			
			甲が一般的に女性よりも力の勝る男性であり、実際に、甲はAをなだめることができず、ほぼ一方的に激しい態様の暴行を受け、手をねじ上げられて菜箸を突きつけられる状況にあった。
			甲はこのようなAの暴行から逃れるために包丁で突き刺したものであり、力を込めずに1回軽く突き刺したという程度にとどまること、他に侵害性の低い代替手段があったとはいえないことからみて、
			甲の反撃行為は、防衛行為としての相当性の範囲内にあり、「やむを得ずにした行為」といえる。
			
			以上より、甲の行為は、正当防衛(刑法36条1項)として違法性が阻却され、傷害罪(刑法204条)は成立しない。
	\sectionB{甲がAの腹部を3回突き刺して重傷を与えた行為の罪責}
		甲の行為に、殺人未遂罪(刑法199条、203条)は成立するか。
		\sectionC{}
			殺人の罪の実行行為は認められるか。甲は、人体の枢要部である腹部を、刃渡り15.5センチメートルという殺傷能力の高い包丁で力任せに3回突き刺している。甲の行為は、Aに重傷を負わせる現実的危険性のある行為であり、これにより出血多量等で、Aが死亡する危険性は極めて高いといえるから、甲の行為は、Aの死の結果を発生させる現実的危険性のある行為といえ、殺人の実行行為性が肯定される。実際に、甲はAの胃、十二指腸、胆嚢および肝臓を貫通する刺創を与えており、通常の外科手術によっても救命が困難と考えられるほど致命的な重傷を負わせている。すなわち、甲は、A死亡の具体的危険を発生させたといえるから、実行の着手(刑法43条)が認められ、甲の行為は殺人未遂罪の構成要件に該当する。
		\sectionC{}
			甲に殺人の故意は認められるか。故意とは、犯罪事実の認識及び認容をいう。包丁で腹部を数回力任せに突き刺せば、人が死亡する危険性は極めて高いといえるから、甲には殺人の認識が認められる。それにも関わらず、憤激に任せて行為に及ぶことによってAの死亡を認容していたといえる。したがって、甲には殺人罪の故意が認められる。
		\sectionC{}
			本問において、Aは搬送先の病院で死亡している。甲の行為と、Aの死亡との間に条件関係は存在するものの、乙が基本的検査を誤って省略するという介在事情が存在することから、法的因果関係の存否が問題となる。
			
			法的因果関係は、偶然的な帰責を防ぐために要求される。これは、客観的に存在する全ての事情を判断資料とし、実行行為の危険が結果へと現実化したか否かによって判断される。
			
			本問において、Aの直接の死因は、不適合輸血による重篤な溶血であり、発生した具体的な結果に対する乙のミスの影響力は大きい。甲の行為は、Aに重傷を負わせ、搬送先の病院で医療行為を介しても死亡させる危険性を有しているといいうるが、乙が誤って省略した検査は、必ず行わなければならない基本的検査であったため、介在事情の異常性は大きい。したがって、甲の行為に、病院で必ず行われるべき基本的検査を誤って省略して死亡する危険があったとはいえない。
			
			以上より、甲の行為と、Aの死亡結果との間に法的因果関係は認められないため、殺人既遂罪は成立せず、殺人未遂罪が成立しうるにとどまる。
			
		\sectionC{}
			もっとも、甲はAからの暴行を避けるために上記行為を行っているから、正当防衛(刑法36条1項)が成立するか。
				\sectionD{}
					Aは、甲が包丁で右腰部を軽く突き刺して反撃した後もなお暴行を続けており当該暴行はその後も継続することが予想される状況であったから、急迫不正の侵害が認められる。
				\sectionD{}
					甲は、憤激の域に達して包丁を奪い取った上、Aの腹部を力任せに複数回突き刺しており、もはや急迫不正の侵害を避けようとする防衛の意思ではなく、もっぱらAに攻撃を加える意思で行為に及んだと評価できる。したがって、甲の行為は防衛するための行為とはいえず、正当防衛は成立しない。
					
					以上により、甲には、殺人未遂罪(刑法199条、203条)が成立する。

\sectionA{乙の罪責}
	乙が、必ず行わなければならない基本的検査を誤って省略してAを死亡させた行為につき、業務上過失致死罪(刑法211条)の成否を検討する。
	
	基本的検査を誤って省略すれば、不適合輸血によって患者を死亡させる現実的危険性が認められるから、乙の行為は、同罪の実行行為に当たる。
	
	前述の通り、Aは死亡しているが、Aは乙の行為に「よって死傷」したといえるか。乙が必要な検査を行っていれば、少なくとも、輸血後直ちに死亡することはなかったといえるから、結果回避可能性が認められる。本問において、Aは重篤な溶血を直接の原因として死亡しており、乙の行為の結果に対する寄与度は極めて大きい。したがって、乙が必要な検査を省略する行為の危険が、直接結果に実現したといえるから、法的因果関係も認められる。
	
	乙は、医師として、必要な検査を行わなければ患者の生命に関わることについて十分な知見があると考えられるから、検査の省略によって死亡結果が発生することは十分予見可能であった。したがって、乙にはA死亡の結果を帰責できるだけの業務上の過失(責任)が認められる。
	
	以上より、乙には、業務上過失致死罪(刑法211条)が成立する。
	


\raggedleft{以上}
	
\end{document}








