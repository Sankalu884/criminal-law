\documentclass[11pt]{jsarticle}

\usepackage[sect]{kian}
\usepackage{okumacro}
\usepackage{fancybox}
\usepackage{ascmac}
\usepackage[noalphabet]{pxchfon}  
\setminchofont{A-OTF-RyuminPro-Light.otf}
\setgothicfont{A-OTF-FutoGoB101Pr6N-Bold.otf}



\title{\vspace{-30mm}{\textgt{\Large{\fbox{14} 燃え移った炎}}}}
\date{\vspace{-15mm}}


\begin{document}

\maketitle

\sectionA{}
	甲乙丙がトーチランプの炎が確実に消火しているか否かにつき何ら確認をすることなく、
	工事現場を離れ、通信ケーブル、洞道壁面を焼損させた行為につき、業務上失火罪の共同正犯(刑法60条、刑法117条の2)の成否を検討する。
	
	\sectionB{業務上失火罪の成否}
		\sectionC{}
			炎を消火せず、または消火の確認をせずに、現場を離れる行為は、炎が防護シートに着火し、火災を発生させる実質的危険性のある行為であるから、業務上失火罪の実行行為に当たる。
		\sectionC{}
			そして、通信ケーブルおよび洞道壁面が「焼損」している。
		\sectionC{}
			通信ケーブルおよび洞道壁面は刑法108条、刑法109条の客体ではないから、業務上失火罪が成立するためには、
			「公共の危険」が発生することが必要である(刑法117条の2、刑法116条2項)。
			
			放火罪は、炎による延焼の危険の発生を特徴とし、その保護法益は不特定多数の人の生命・身体・財産である。
			したがって、公共の危険とは、延焼によって不特定多数の人の生命・身体・財産が害される危険をいう。
			なお、延焼の対象を刑法108条および109条1項の物件に限る見解もあるが、
			建造物以外への延焼からもこれらの法益が害されるおそれがあるため、対象を限定する必要はない。
			
			本問において、周囲の建物に延焼する危険はなく、延焼による不特定多数の人の生命・身体・財産が害される危険は発生していないから、
			「公共の危険」は認められない。
			
			or
			
			延焼の危険に限らず、火力による危険性の発生で足りると解して、「公共の危険」を認める。
			
\dotfill
		\sectionC{(以下「公共の危険」を認めた場合)}トーチランプの消火を確認していれば火災は発生せず、結果を回避することが十分可能であった。
		また、焼損の結果は、消火を確認せず現場を離れたという行為に内在する火災発生の危険が現実化したものであり、行為と結果との間には因果関係が認められる。

		
		\sectionC{}
			そして、トーチランプの炎が防護シートに接して着火し、火災が発生する危険を十分に予見することができたというのであり、
			かつ、火を用いる作業員には、火気の安全に配慮すべき社会生活上の地位が認められるから、
			焼損の結果を帰責できるだけの業務上の過失が認められる。
			
			
			したがって、業務上失火罪が成立する。
			
	\sectionB{共同正犯の成否}
		では、甲らに共同正犯が成立しないか。
		
		過失の共同正犯が成立するためには、結果発生の実質的危険を有する共同行為を行うにあたり、
		各人に結果回避の共通の義務が認められる場合において、その義務に共同して違反したこと(共同義務の共同違反)が必要である。
		
		本問において、甲らはトーチランプを用いて作業を行い、意思を通じて共同して火災の危険を発生させており、
		これを回避するために、自身が使用していたランプを消火するとともに、共同者のランプの消火を確認する共同義務が認められる。
		
		にもかかわらず、甲らはいずれもこれらの義務に反して、現場を立ち去ったことから、共同義務の共同違反を肯定できる。
		
		したがって、共同正犯が成立する。
		
		以上より、甲乙丙には業務上失火罪の共同正犯(刑法60条、刑法117条の2)が成立する。
		





\begin{flushright}
	以上
\end{flushright}
	
\end{document}








