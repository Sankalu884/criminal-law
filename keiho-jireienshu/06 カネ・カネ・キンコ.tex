\documentclass[11pt]{jsarticle}

\usepackage[sect]{kian}
\usepackage{okumacro}
\usepackage{fancybox}
\usepackage[noalphabet]{pxchfon}  
\setminchofont{A-OTF-RyuminPro-Light.otf}
\setgothicfont{A-OTF-FutoGoB101Pr6N-Bold.otf}



\title{\vspace{-30mm}{\textgt{\Large{\fbox{6} カネ・カネ・キンコ}}}}
\date{\vspace{-15mm}}


\begin{document}

\maketitle

\sectionA{乙の罪責}
	\sectionB{}
		乙が、Dにエアーガンを突きつけ現金を奪い、
		恐怖心により放心状態であるDと性交に及んだ行為につき、
		強盗・不同意性交罪(刑法241条1項)の成否を検討する。
		
		
		同罪は、強盗罪(刑法236条1項)と、不同意性交罪(刑法177条1項)の結合犯であるから、それぞれの罪が成立するかを順に検討する。
		\sectionC{}
			まず、乙に強盗罪(刑法236条1項)は成立するか。
			
			強盗罪が成立するためには、
			\MARU{1}暴行・脅迫を用いて、
			\MARU{2}他人の財物を強取することが必要である。
				\sectionD{}
					強盗罪は相手方の意思に基づかず、財物を奪取する犯罪であって暴行・脅迫はその手段であるから、強盗罪における暴行・脅迫とは、相手方の意思を抑圧する程度のものをいう。
					
					
					拳銃は極めて殺傷能力の高い武器であり、拳銃とそっくりのエアーガンを突きつける行為は、相手方に生命侵害の危険性を感じさせるものである。
					加えて、犯行時刻は夜10時のDしかいない時間帯であり、シャッターを下ろして閉店したかのように見せかけており、第三者に助けを求めることも困難であった。
					
					以上のことから、乙の行為は、相手方の犯行を抑圧する程度の脅迫といえる。
				\sectionD{}
					そして乙は、この脅迫により現にDに抵抗の意思をなくさせ、現金35万円を奪取しているから、財物を強取したといえる。
				\sectionD{}
					故意と不法領得の意思も認められる。
					したがって、乙には強盗罪が成立する。
		\sectionC{}
			次に乙は、Dが、脅迫により抵抗をすることが困難な状態にあることに乗じて性交を行なっているから、不同意性交罪(刑法177条1項)が成立する。
			
		\sectionC{}
			以上より、乙には強盗罪及び不同意性交罪のいずれも成立するから、強盗・不同意性交罪(刑法241条1項)が成立する。
			
			なお、後述の通り、甲との間で強盗致傷罪の限度で共同正犯となる
	\sectionB{}
		乙が、追跡してきたEに向かってエアーガンを発射し、全治3週間の打撲傷を負わせた行為につき、強盗致傷罪(刑法240条)の成否を検討する。
		
		ここで、乙には強盗・不同意性交罪が成立しているため、別に強盗致傷罪が成立しうるかが問題となる。
		強盗・不同意性交罪は、強盗罪及び強制性交等罪の加重類型であって、強盗致傷罪よりも法定刑が重い。したがって、強盗及び強制性交に通常随伴する程度の傷害は、強盗・不同意性交等罪によって評価されているといえる。
		
		そして、強盗・不同意性交の後に、追跡者に傷害結果が生じることは、十分あり得るといえるから、乙に別途強盗致傷罪は成立しない。

	\sectionB{}
		乙が、Eに「文句はないな」と申し向け、財布を奪った行為につき、強盗罪(刑法236条1項)の成否を検討する。
		\sectionC{}
			乙は、エアーガンをEの身体の中心部に3発射し、Eは仰向けに倒れているから、乙の暴行は、犯行を抑圧する程度の暴行であるといえる。問題は、乙がEに暴行を加えたのは、「捕まってなるものか」という逮捕免脱目的に基づくものであって、財布を領得する意思が生じたのは、暴行後であったことである。
			\sectionD{}
				強盗罪は暴行・脅迫を手段として財物を奪取する犯罪であり、また、刑法176条1項や刑法177条1項のように暴行・脅迫を受けた状態に乗じた場合の規定が強盗罪については存在しない。したがって、強盗罪における暴行・脅迫は、財物奪取の手段として行われる必要があるから、反抗抑圧状態を利用して財物を奪取しただけでは強盗罪は成立せず、財物奪取に向けた新たな暴行・脅迫が必要である。
				
				必要とされる新たな暴行・脅迫は、すでに反抗が抑圧されている者に対する暴行・脅迫であるから、通常の場合と比べて程度の低いものでよく、また反抗抑圧状態を維持・継続させるもので足りると解される。
				
				本問において、乙は倒れて苦悶の表情を浮かべているEに対して「文句はないな」と申し向けているが、このような状況下では、乙の意に反する行動をとれば再度暴行を加えられると考えるのが通常であり、乙の行為はEの反抗抑圧状態を維持・継続させるものといえるから、財物奪取に向けた新たな脅迫が認められる。
			\sectionD{}
				乙は、脅迫により反抗を抑圧したEから現金の入った財布を奪取しており、財物を強取したといえる。
			\sectionD{}
				また故意と不法領得の意思も認められる。
		\sectionC{}
			したがって、乙には強盗罪(刑法236条1項)が成立する。
\sectionA{甲の罪責}
	\sectionB{}
		甲が、乙に強盗を実行させた行為につき、強盗罪(刑法236条1項)の間接正犯の成否を検討する。
		
		\sectionC{}
			間接正犯と認められるためには、背後者が、\MARU{1}他人の行為を一方的に利用して自己の犯罪を実現しようとする意思の下、\MARU{2}他人の行為を一方的に利用して結果の実現過程を支配したといえることが必要である。
		\sectionC{}
			本問において、乙は、甲に万引きの現場を目撃されており、警察や学校に知らされたくないとの思いから、乙の立場は弱くなっている。また、甲の風貌は暴力団員風のものであって、かつ、自身の犯罪歴を具体的に話すなど、乙に相当程度の恐怖心を与えており、乙が14歳とまだ判断能力が十分でない年齢であることも併せて考えると、強盗実行への働きかけは強固なものであるといえる。
			
			また、犯行に用いる道具を交付し、具体的な計画を伝えるなどしているから、自己の犯罪を実現しようとする意思も認められる。
			
			しかし、乙は、犯行現場であるスナックCにおいて、甲から指示されていないにも関わらず自身の判断でシャッターを閉める、警察への発覚を防ぐために性交に及ぶなど、主体的に行動しており、甲の指示命令通りに一方的に利用されて犯行に及んだとはいえない。
			
			したがって、間接正犯は成立しない。
	\sectionB{}
		では、強盗罪の共同正犯は成立するか。甲は強盗の実行行為を行なっていないにも関わらず、共同正犯が成立しうるかが問題となる。
		
		\sectionC{}
			ここで、刑法60条はその文言上、実行行為の有無によって共同正犯を区別していないし、共同正犯における「一部実行全部責任」という法的効果が導かれるのは、構成要件該当事実を共同して惹起するからであるから、構成要件該当事実惹起に重要な事実的寄与を果たすことにより、構成要件該当事実全体の惹起が認められれば、実行行為の分担は必ずしも必要ではないと考えることができる。
		\sectionC{}
			共同正犯の成立要件は、\MARU{1}共謀と、\MARU{2}それに基づく実行である。
			
			\sectionD{}
				共謀とは、正犯意思の下、犯罪を共同で遂行しようという意思連絡をいう。本問において、甲は犯行計画を立案し、必要な道具を交付するとともに、乙に実行を指示・命令しており、構成要件該当事実惹起に重要な事実的寄与を行なっていること、また、強取された現金の大部分を甲が取得していることから、正犯性が認められる。そして、乙は甲の指示を受け入れているから、強盗罪につき共同遂行の意思連絡があったといえ、共謀が認められる。
			\sectionD{}
				そして、乙は共謀に基づいて強盗を行い、強盗を完遂するために不同意性交を行っているから、甲の乙に対して及ぼした影響の強度を考えると、強盗のみならず不同意性交の結果も共謀の危険が現実化した結果といえる。したがって、共謀に基づく実行が認められる。
		\sectionC{}
			もっとも、甲には強盗の故意しかないにもかかわらず、乙は不同意性交を行っているため、共同正犯が成立する範囲が問題となる。
			
			共同正犯の本質は、特定の犯罪を共同して実行することに求められるから、異なる構成要件間での共同正犯は成立しない。もっとも構成要件が同質的で重なり合う場合にはその範囲で共同正犯の成立を認めることができ、また、重い故意を有する者には別に単独正犯が成立する(部分的犯罪共同説)。
			強盗・不同意性交罪と強盗罪は強盗罪の限度で重なり合いが認められるため、その範囲で共同正犯の構成要件該当性が認められる。
		\sectionC{}
			以上より、甲には強盗罪の共同正犯が成立(刑法60条、刑法236条1項)する。
	\sectionB{}
		Eに対する強盗罪につき、甲に共同正犯が成立するか。
		
		甲と乙の間で成立した共謀は、スナックCにおける強盗についてであるが、乙のEに対する強盗は、乙がEの反抗を抑圧した後に犯意をを生じて実行したものであって、Eに対する強盗につき共謀は成立していない。したがって、共同正犯は成立しない。
\sectionA{罪数}
	\sectionB{}
		以上より、乙には\MARU{1}強盗・不同意性交罪、強盗罪の限度で共同正犯(刑法60条(ただし強盗罪の限度で)、241条1項)、\MARU{2}Eに対する強盗罪(刑法236条1項)が成立する。\MARU{1}と\MARU{2}は併合罪(刑法45条前段)となる。
	\sectionB{}
		甲には、\MARU{3}強盗罪の共同正犯が成立する。
\begin{flushright}
	以上
\end{flushright}
	
\end{document}
































































