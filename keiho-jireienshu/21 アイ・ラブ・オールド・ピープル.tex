\documentclass[11pt]{jsarticle}

\usepackage[sect]{kian}
\usepackage{otf}
\usepackage{fancybox}
\usepackage{ascmac}
\usepackage[noalphabet]{pxchfon}  
\setminchofont{A-OTF-RyuminPro-Light.otf}
\setgothicfont{A-OTF-FutoGoB101Pr6N-Bold.otf}



\title{\vspace{-30mm}{\textgt{\Large{\fbox{21} アイ・ラブ・オールド・ピープル}}}}
\date{\vspace{-15mm}}


\begin{document}

\maketitle

\sectionA{}
	甲が、理事会を開いていないにも関わらず議事録を作成した行為につき、無印私文書偽造罪(刑法159条3項)の成否を検討する。
	
	\sectionB{}
	議事録は、議事の内容や結果といった「事実証明に関する文書」である。
	
	\sectionB{}
		では、「偽造」といえるか。「偽造」とは、作成者と名義人との人格の同一性を偽る行為をいう。
		作成者とは文書に表示された意思・観念の帰属主体をいい、名義人とは、文書に表示された意思・観念の帰属主体として文書の形式・内容から認識される者をいう。
		
		本問の議事録の作成者は甲野であるが、名義人は誰か。
		理事会の議事録の機能は、記載された議事が理事会で実際に行なわれたことを証明する点にある。
		したがって、文書偽造罪の保護法益である文書に対する公共の信用は、その記載内容が理事会の議事内容・承認事項であるというところに向けられている。
		よって、議事録の名義人は理事会である。
		
		甲野と理事会は別人格であるから、人格の同一性が偽られている。したがって、「偽造」が認められる。

\sectionB{}
		甲野は、「理事会議事録署名人甲野一郎」と署名している。
		有印私文書偽造罪における有印性を認めるためには、「他人の印章若しくは署名を使用」する必要がある。
		
		本問において甲野が行った署名は、甲野自身の名前に基づくものであり、「他人」である理事会の名前を使用したのではないから、有印性は認められない。
		
	\sectionB{}
		甲野は、膠着状態に業を煮やして理事を選任することを意図していたのであるから、行使の目的が認められる。
		また、これらの犯罪事実を認識・認容していたといえるから、故意も認められる。
		
		以上により、甲には無印私文書偽造罪(刑法159条3項)が成立する。なお、後述する通り乙と共同正犯となる。

\sectionA{
	甲が乙の養子となり、乙山一郎として契約書を作成し、キャッシングカードの交付を受けた行為の罪責}
	
		\sectionB{}有印私文書偽造罪(刑法159条1項)、同行使罪(刑法161条1項)の成否を検討する。
		
		契約書は、「権利、義務に関する文書」といえる。
		\sectionC{}
			では「偽造」は認められるか。
			作成者は、甲こと乙山一郎であり、名義人も乙山一郎であるといいうるから、人格の同一性に誤りはなく、偽造は認められないとも思える。
			
			しかし、キャッシングカードの申込書は、そこに記載された者が融資適格者であることを証明し、それに基づいて、その者が、キャッシングカードの交付を申請する文書であるから、申込書に接する者は、申込書が融資適格者であるか否かを証明できる文書であると信用する。したがって、こうした文書に接した者が名義人として把握するのは、\UTF{2460}「適法に養子縁組をし、融資適格者である乙山一郎」である。それに対して、作成者は、\UTF{2461}「適法な養子縁組をしていない、融資不適格者である、乙山一郎を名乗っている甲野」であるから、文書に対する信用が裏切られたといえる。
			
			したがって、人格の同一性が偽られているから、「偽造」といえる。
			
			\sectionC{}
				甲(\UTF{2461})は、「他人」である融資適格者の乙山一郎(\UTF{2460})の署名を使用しているから、有印性が認められる。
				
			\sectionC{}
				甲には行使の目的、故意が認められ、また実際に真正な文書として使用しているから、有印私文書偽造罪(刑法159条1項)、同行使罪(刑法161条1項)が成立する。

	\sectionB{}
		詐欺罪(刑法246条1項)の成否を検討する。キャッシングカードは、それにより現金を得ることができるから、「財物」であるといえる。
		
			\sectionC{}
				では、「人を欺」く行為(欺罔行為)は認められるか。
				欺罔行為とは、財物の交付に向けて、交付の判断の基礎となる重要な事項について錯誤を惹起する行為をいう。
				
				詐欺罪において財産は交換手段・目的達成手段として保護されているから、重要な事項とは、交付によって達成される目的にとって重要な事項をいう。
				
				本問において、融資不適格者であることを秘して申込を行うことは、交付に向けられた行為であり、また、融資適格者にキャッシングカードを交付することがI社の取引目的といえるから、甲の申込行為は、欺罔行為に当たる。
				
			\sectionC{}
				甲は欺罔行為によって惹起された錯誤に基づいてI社にキャッシングカードを交付させ、これを受領している。
				故意と不法領得の意思も認められるから、甲には、詐欺罪(刑法246条1項)が成立する。

\sectionA{乙の罪責}
	\sectionB{理事会議事録の偽造について共同正犯(刑法60条)の成否}
		共同正犯の成立要件は、\UTF{2460}共謀と、\UTF{2461}共謀に基づく実行である。本問において、甲と乙は、虚偽の議事録を作成することにつき共謀し、
		実際に作成しているから、無印私文書偽造罪(刑法159条3項)の共同正犯(刑法60条)が成立する。
		
	\sectionB{契約書の作成と、キャッシングカードの交付に係る犯罪について}
		甲は、乙の相談に応じただけであって、自己の犯罪として行う意思に欠け、共謀は成立していない。
		したがって、共同正犯は成立しない。
		
		もっとも、乙の発言により甲の犯罪は容易になったといえるから、幇助犯(刑法62条1項)が成立する。

\sectionA{罪数}
	以上より、甲には、\UTF{2460}無印私文書偽造罪(159条3項)の共同正犯(刑法60条)、\UTF{2461}有印私文書偽造罪(刑法159条1項)、\UTF{2462}同行使罪(刑法161条1項)、\UTF{2463}詐欺罪(刑法246条1項)が成立する。\UTF{2461}と\UTF{2462}は目的・手段の関係にあるから、牽連犯(刑法54条1項後段)となり、これと\UTF{2463}も牽連犯となり、全体として科刑上一罪となる。これと\UTF{2460}は併合罪(刑法45条前段)となる。
	
	乙には、\UTF{2460}無印私文書偽造罪(159条3項)の共同正犯(刑法60条)、\UTF{2461}甲に成立する科刑上一罪の関係にある犯罪に対する幇助犯(刑法62条1項)が成立し、これらは併合罪(刑法45条前段)となる。
	
\begin{flushright}
	以上
\end{flushright}
	
\end{document}








