\documentclass[fontsize=11pt,
jlreq_notes,
]{jlreq}

% 1. フォント・日本語処理
\usepackage{pxchfon} 
\usepackage{otf}

% 2. フォント設定
\setminchofont{A-OTF-RyuminPro-Light.otf}
\setgothicfont{A-OTF-FutoGoB101Pr6N-Bold.otf}

% 3. 汎用パッケージ
\usepackage{fancybox}

% 4. 自作パッケージ
\usepackage[sect]{kian}


\title{\vspace{-30mm}{\textgt{\Large{\fbox{26} 某球団ファンの暴走・その1}}}}
\date{\vspace{-15mm}}


\begin{document}

\maketitle

\sectionA{乙、丙がリサイクルショップAに侵入し、逃走中にBを負傷させた行為の罪責}
	\sectionB{建造物侵入罪および窃盗罪の成否}
		甲と乙は、共同してAに侵入して金目のものを盗むという共謀を行い、これに基づいて、Aの管理権者の意思に反して「侵入し」、貴金属売り場に行ってショーケースをたたき割っている。もっとも、ショーケースの中の財物の占有を移転するには至っていない。
		建造物侵入罪の故意、窃盗罪の故意と不法領得の意思も認められるから、甲と乙には、建造物侵入罪(刑法130条前段)の共同正犯(刑法60条)、窃盗未遂罪(刑法243条、235条)の共同正犯が成立する。
	\sectionB{}
		乙は窃盗(未遂)犯であり、逃走する際に携帯していたスパナでBの右脇腹を力いっぱい殴打している。
		これは、犯行を抑圧するに至る程度の暴行であり、事後強盗罪(刑法238条)の実行行為に当たる。逮捕を免れる目的と故意も認められるから、乙には、事後強盗罪が成立する。
		また、Bはこの暴行により加療2週間を要する傷害を負っているから、乙には、強盗致傷罪(刑法240条前段)が成立する。

		では、甲との間に強盗致傷罪の共同正犯は成立するか。甲は暴行を加えることを認識していないため、問題となる。
		
		甲と乙の共謀において、逃走する際に暴行を加えることまで想定していたとはいえず、甲に強盗致傷罪の故意は認められないから、甲に強盗致傷の結果を帰責することはできない。

		したがって、強盗致傷罪の共同正犯は成立しない。
\sectionA{甲の罪責}
	\sectionB{}
	甲が、乙らが犯行に及ぶことに気がついていたにもかかわらず、これを止めなかった行為について、甲は、専ら乙の機嫌を損ねることを回避することに関心があり、窃取される貴金属などについて一切関心がないから、自己の犯罪として関与しているとはいえない。そこで狭義の共犯として[建造物侵入罪および窃盗罪]の幇助犯の成否を検討する。
		\sectionC{}
			甲が「むちゃせんといてや」等と申し向けたことにより、乙の犯意が強化され、犯行を促進する効果があったとは評価できず、作為の幇助犯は成立しない。
		\sectionC{}
			そこで、不作為の幇助犯の成立を検討する。不作為犯の成立を認めるためには、作為義務が必要である。具体的には、法益保護義務ないし、犯罪阻止義務が認められる必要がある。

			甲は、無関係のリサイクルショップAの法益を保護する義務を負っているとは考えられないし、また、甲が乙と同居しているとしても、乙は自律的意思決定に基づいて犯行に及んでおり、その意思決定に介入して犯行を阻止する義務を負っているわけではない。
	\sectionB{}
		したがって、甲に作為義務は認められないから、不作為の幇助犯は成立しない。

\sectionA{丁の罪責}
	\sectionB{}
		丁は、モニターを見て犯行に気付きながら、警察に通報せず、現場にも急行しなかった。これらの行為によって新たに財産に対する危険が創出されているわけではなく、既に発生している危険が拡大しているに過ぎないから、丁の行為は不作為である。また、丁の行為について乙らは認識していないから、窃盗未遂罪に対する不作為の片面的共犯の成否が問題となる。
		\sectionC{}
			共同正犯の成立には、共謀が必要である。共謀とは、2人以上のものが犯罪行為の共同遂行について合意することを意味し、共謀が認められるためには相互の意思連絡が必要であるから、片面的共同正犯は成立しない。
		\sectionC{}
			そこで、片面的幇助犯の成立を検討する。不作為犯が成立するためには作為義務が認められることが必要である。
			
			丁は夜間のAの警備を担当しており、また、乙らが侵入した際にはBが休憩中であって、丁しか犯行を阻止することができなかった。警察に通報する、現場に急行するといった行動を取ることは容易であり、作為可能性も認められるから、作為義務が肯定できる。
			
			幇助犯における因果関係は、幇助行為によって正犯行為を物理的・心理的に容易にする関係があれば足りる。丁が警察に通報すれば、乙らの犯行は困難になったといえるから、丁の不作為には、乙らの犯行を容易にする関係があるといえる。
			
			丁はこれらの犯罪事実を認識・認容していたといえるから、故意も認められる。
	\sectionB{}
		以上より、丁には窃盗未遂罪(刑法243条、235条)の幇助犯(62条1項)が成立する。なお、事後強盗罪、強盗致傷罪の幇助犯について、丁に作為義務が認められるとしても、事後強盗に発展する認識に欠けるため、故意が認められず、幇助犯は成立しない。

\sectionA{罪数}
	以上により、甲にはなんらの犯罪も成立しない。
	
	乙には、\UTF{2460}建造物侵入罪(刑法130条前段)の共同正犯(刑法60条)、\UTF{2461}窃盗未遂罪(刑法243条、235条)の共同正犯が成立する。これらは通例、手段と目的の関係に立つから、牽連犯(刑法54条1項後段)となる。
	
	丙には、\UTF{2462}建造物侵入罪の共同正犯、\UTF{2463}強盗致傷罪、窃盗未遂罪の限度で共同正犯(刑法60条(ただし窃盗未遂罪の限度で)、刑法240条前段)が成立する。これらは牽連犯(刑法54条1項後段)となる。
	
	丁には、窃盗未遂罪(刑法243条、235条)の幇助犯(62条1項)が成立する。



\raggedleft{以上}
	
\end{document}








