\documentclass[11pt]{jlreq}

\usepackage[sect]{kian}
\usepackage{otf}
\usepackage{fancybox}
\usepackage{ascmac}
\usepackage{pxchfon}  
\setminchofont{A-OTF-RyuminPro-Light.otf}
\setgothicfont{A-OTF-FutoGoB101Pr6N-Bold.otf}



\title{\vspace{-30mm}{\textgt{\Large{\fbox{22} 即断3連発}}}}
\date{\vspace{-15mm}}


\begin{document}

\maketitle

\sectionA{甲が背後から飛びかかって羽交い締めにした上、Bの胸を触り、首筋の数ヶ所にキスをした行為の罪責}
	\sectionB{}
		甲の行為は、不同意わいせつ罪(刑法176条1項1号)の構成要件に該当するように見えるが、甲は性交する目的を持っているから、甲に不同意性交未遂罪(刑法180条、刑法177条1項)が成立する場合、両者は法条競合の関係に立ち、前者の構成要件該当性は否定されるから、先に不同意性交未遂罪の成否を検討する。
		\sectionC{}
			甲は性交を意図して上記行為に及んでいるが、これは、同意しない意思を全うすることが困難な状態で行なわれたものであって、不同意性交罪の実行の着手が認められうる行為である。しかし、Bは男性であって目的を達成することは不可能であったため、不能犯とならないかが問題となる。
			
			未遂の処罰根拠は、結果発生の具体的危険に求められるから、未遂犯と不能犯の区別は、当該危険性の有無によって判断される。具体的には、結果発生に必要な事実(仮定的事実)の存在可能性の有無を、一般人の事後的な危険感を実質的な基準として判断する。
		\sectionC{}	
			本問において、BがAであれば、甲は目的を遂げることができたから、これが仮定的事実である。
			Aは残業後、林道を通って帰宅していたから、甲が待ち伏せしていた時間にAが当該林道を通行することはあり得たと考えられる。
			
			したがって、結果発生の具体的危険性が認められるから、不能犯の成立は否定され、実行の着手が認められる。
			
		\sectionC{}
			甲には同罪の故意も認められる。
			
		\sectionB{}
			以上より、甲には不同意性交未遂罪(刑法180条、刑法177条1項)が成立する。
		
\sectionA{甲がBから逃げるために押し倒し、転倒したBに重傷を負わせた行為について強制性交致傷罪(刑法181条2項)の成否を検討する。}

\sectionB{}
		Bは頭部に傷害を負っているから「死傷させた」にあたる。問題は、本罪が成立するためには、死傷結果がどのような行為から生じればよいかである。
		
		181条は「よって人を死傷させた」と規定しており、死傷結果は必ずしも性交等から生じた場合、手段である暴行等から生じた場合に限られない。もっとも、犯行後の行為であっても広く本罪が成立すると処罰範囲が不当に広がってしまうから、少なくとも、犯行との接着性が要求されるべきである。

		\sectionB{}
			本問において、不同意性交未遂が成立する甲の行為が行なわれた後、Bは逃げる甲にすぐに追いついており、そのままもみ合いになっているから、時間的場所的接着性が認められる。

		\sectionB{}
			以上より、甲には強制性交致傷罪(刑法181条2項)が成立する。
			
\sectionA{甲がBは死亡したと誤信して、Bの身体を茂みに隠した行為の罪責}
		
		\sectionB{}
			甲は、Bが死亡したと誤信して行為に及んでいるから、死体遺棄罪(刑法190条)の故意が認められる。
			しかし、客観的にはBは死亡していないから、甲の行為は単純遺棄罪(刑法217条)ないし保護責任者遺棄罪(刑法218条)の構成要件に該当しうる行為である。
			
			そこで、客観的に発生した事実を、死体遺棄罪の故意に対応する構成要件該当事実と評価できるか、いわゆる抽象的事実の錯誤が問題となる。
			
		\sectionB{}
			構成要件に軽い罪の限度で実質的な重なり合いが認められれば、その限度で構成要件該当性を認めることができる。構成要件の重なり合いの有無は、構成要件の実体が法益侵害行為の類型であることから、行為態様と保護法益の共通性によって判断する。
			
			遺棄罪の保護法益は、人の生命であり、死体遺棄罪の保護法益は、国民の宗教的感情及び死者に対する敬虔・尊崇の感情であって保護法益が異なるから、構成要件の実質的な重なり合いは認められない。
			
			\sectionB{}
				以上より、甲の行為に犯罪は成立しない。
				
\sectionA{甲がBの財布から現金5万円を持ち去った行為について窃盗罪(刑法235条)の成否を検討する。}
		\sectionB{}
			Bの財布及びその中の現金は、Bの占有に係るBの所有物であり、甲がこれを持ち去った行為は「窃取」にあたる。
			
			問題は、甲が、Bは死亡していると誤信していることから、甲の認識を前提とすると、Bに財布の占有は認められず、甲の故意は占有離脱物横領(刑法254条)にとどまるのではないかである。
		
		\sectionB{}
			故意とは犯罪事実の認識及び認容をいうから、甲が認識していた事実、すなわち既にBが死亡していたという事実を前提にして窃盗罪が成立するのであれば、窃盗罪の故意を認めることができる。
			
			確かに、死者には占有の意思も占有の事実も認められない以上、死者に占有を認めることはできない。しかし、犯人が自ら被害者を殺害し占有を失わせることにより、被害者の占有する財物を占有離脱物に変えたのに、そのことにより後行行為の罪責が軽くなり、財物の占有が保護されなくなってしまうのは不当である。
			
			そこで殺害行為(占有離脱行為)と取得行為の一体性が認められる場合には、当該行為者との関係においては、なお生前の占有が存続しているものとして罪責評価を行うべきである。
		
		\sectionB{}
		本問において、殺害行為(Bに重傷をおわせた行為)と現金の占有を取得した行為には時間的場所的接着性が認められる。また、Bの死亡によって占有を失わせたことを認識し、これを利用して占有を取得するという1つの意思に基づいて行為に及んでいるから、行為の一体性が認められる。なお、殺害行為と占有取得行為の間には、死体遺棄行為が介在しているが、死体遺棄は殺害行為に通常随伴する行為といえるから、一体性は否定されない。
		
		したがって、甲の認識を前提としても、甲との関係ではBの生前の占有が保護されるから、窃盗罪が成立する。
		
		\sectionB{}
			以上より、甲には窃盗罪の故意が認められる。不法領得の意思も認められるから、甲には窃盗罪(刑法235条)が成立する。
			
\sectionA{罪数}
	以上により、甲には\UTF{2460}不同意性交未遂罪(刑法180条、刑法177条1項)、\UTF{2461}強制性交致傷罪(刑法181条2項)、\UTF{2462}窃盗罪(刑法235条)が成立する。\UTF{2460}は\UTF{2461}に吸収され、これと\UTF{2462}は併合罪(刑法45条前段)となる。






\raggedleft{以上}
	
\end{document}








