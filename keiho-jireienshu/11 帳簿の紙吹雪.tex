\documentclass[11pt]{jsarticle}

\usepackage[sect]{kian}
\usepackage{okumacro}
\usepackage{fancybox}
\usepackage[noalphabet]{pxchfon}  
\setminchofont{A-OTF-RyuminPro-Light.otf}
\setgothicfont{A-OTF-FutoGoB101Pr6N-Bold.otf}



\title{\vspace{-30mm}{\textgt{\Large{\fbox{11} 帳簿の紙吹雪}}}}
\date{\vspace{-15mm}}


\begin{document}

\maketitle

\sectionB{}
	甲がBから帳簿をつかみ取り、Bの制止を振り払った上、帳簿を投棄した行為につき公務執行妨害罪(刑法95条1項)の成否を検討する。
	
	\sectionC{}
		Bは国税調査官であるから「公務員」である。
		
	\sectionC{}
		職務とは、公務員の行う事務の全てをいう。
		
		ここで、公務執行妨害罪の保護法益は公務員によって執行される職務であって、
		違法な職務は保護に値しないと考えられるから、
		刑法95条1項にいう「職務」とは、適法な職務をいうと解される。
		
		職務の適法性は、\MARU{1}当該職務が当該職務の抽象的権限の範囲に属すること、
		\MARU{2}当該公務員が当該職務を行う具体的権限を有していること、
		\MARU{3}法律上の重要な方式を履践していること、という要件の下、行為時を基準に客観的に判断される。






\begin{flushright}
	以上
\end{flushright}
	
\end{document}








