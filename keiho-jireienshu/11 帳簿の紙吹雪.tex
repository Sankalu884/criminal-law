\documentclass[11pt]{jsarticle}

\usepackage[sect]{kian}
\usepackage{okumacro}
\usepackage{fancybox}
\usepackage[noalphabet]{pxchfon}  
\setminchofont{A-OTF-RyuminPro-Light.otf}
\setgothicfont{A-OTF-FutoGoB101Pr6N-Bold.otf}



\title{\vspace{-30mm}{\textgt{\Large{\fbox{11} 帳簿の紙吹雪}}}}
\date{\vspace{-15mm}}


\begin{document}

\maketitle

\sectionB{}
	甲がBから帳簿をつかみ取り、Bの制止を振り払った上、帳簿を投棄した行為につき公務執行妨害罪(刑法95条1項)の成否を検討する。
	
	\sectionC{}
		Bは国税調査官であるから「公務員」である。
		
	\sectionC{}
		「職務」とは、公務員の行う事務の全てをいう。
		
		そして公務執行妨害罪の保護法益は公務員による職務の円滑な執行であって、
		違法な職務は保護に値しないと考えられるから、
		刑法95条1項にいう「職務」とは、適法な職務をいうと解される。
		
		職務の適法性は、\MARU{1}当該職務が当該職務の抽象的権限の範囲に属すること、
		\MARU{2}当該公務員が当該職務を行う具体的権限を有していること、
		\MARU{3}法律上の重要な方式を履践していること、という要件の下、行為時を基準に客観的に判断される。
		
		Bは、A会社の税務調査を行う国税調査官であるから、帳簿を点検する抽象的・具体的権限を有している。
		問題は、Bが法定の身分証明書を所持していなかったため、甲が証明書の提示を求めたにも関わらず、これを提示しなかったことである。
		
		確かに、Bの税務調査は公務としての外形および実質をそなえており、重大な違法はないとも思える。
		しかし、証明書の携行・提示は、強行規定であり、
		証明書の不携帯・不提示は相手方の拒否を正当化するような重要な方式である。
		なお、証明書の不携行は相手方からの提示要求がなければ顕在化することのない
		手続上の瑕疵であるから、提示要求がなければ事後的に不携行が発覚したとしても、
		なお適法な職務執行と評価する余地があるが、
		本問において甲はBに対して証明書の提示を求めてい。
		したがって、Bの職務執行には重大な違法があるといえる。
		
		以上より、Bの税務調査は刑法95条1項にいう「職務」に当たらないから、甲の行為は構成要件に該当しない。
		したがって、甲は不可罰である。
		
\dotfill		
		
	\sectionC{}
		(以下「職務」該当性を認めた場合)
		公務執行妨害罪の実行行為は暴行又は脅迫である。
		本罪は、公務員の身体の安全ではなく、公務員の職務の円滑な執行を保護法益とする罪であるから、
		暴行又は脅迫は、公務員に対して直接向けられたものに限られず間接的に向けられたものでもよい。
		
		%西田・刑法各論〔7版〕451頁
		もっとも、刑法95条1項は暴行・強迫が「これに対し」行なわれること、
		すなわち公務員に対して行われることを要求しているから、
		間接暴行が認められるのは、当該行為が公務員の面前で行われた場合に限られると解すべきである。
		
		本問において、甲はBの手を振り払うという直接的な暴行、
		Bの面前で帳簿を破って破棄するという間接的な暴行を加えており、
		これらは公務執行妨害罪にいう暴行に当たる。
		
	\sectionC{}
		もっとも、甲はBの職務は違法だと判断して、帳簿の投棄に及んでいるから、
		犯罪事実の認識がないとして故意が阻却されるのではないか。
		
		甲は、Bが国税調査官であること、A社の税務調査を行う権限を有していたことを認識しているから、
		違法性を基礎づける事実の認識が認められる。
		その上で、Bの職務執行の違法性が公務執行妨害罪の不成立をもたらす程度の違法性だと誤信したのは、
		違法性の錯誤に過ぎないから、故意は阻却されない。
\sectionB{}
	以上より、甲には公務執行妨害罪(刑法95条1項)が成立する。






\begin{flushright}
	以上
\end{flushright}
	
\end{document}








