\documentclass[11pt]{jsarticle}

\usepackage[sect]{kian}
\usepackage{otf}
\usepackage{fancybox}
\usepackage{ascmac}
\usepackage[noalphabet]{pxchfon}  
\setminchofont{A-OTF-RyuminPro-Light.otf}
\setgothicfont{A-OTF-FutoGoB101Pr6N-Bold.otf}



\title{\vspace{-30mm}{\textgt{\Large{\fbox{16} 哀しき親子}}}}
\date{\vspace{-15mm}}


\begin{document}

\maketitle
	\sectionB{}
		甲と乙がAを押さえつけて死亡させた行為につき、傷害致死罪(刑法205条)の共同正犯(刑法60条)の成否を検討する。
		
		共同正犯の成立要件は、\UTF{2460}共謀と、\UTF{2461}共謀に基づく実行である。
		甲と乙がAを押さえつける行為は、Aの身体に対する有形力の行使として、暴行に当たる。
		Aが酒によって粗暴な行動に出た場合、甲と乙はAの体を押さえつけることがしばしばあったというのであり、本件においても、
		Aが甲の部屋に入って同人に殴りかかろうとした時点で、Aがおとなしくなるまでの間、体を押さえつけるという暴行を共同で行う旨の共謀が成立していたといえる。
		その後、乙がAの頸部を強く押さえたことにより、Aは死亡しているが、これはAをおとなしくさせるための暴行といえるから、共謀に基づく実行といえる。
		したがって、甲と乙は、暴行罪の結果的加重犯としての傷害致死罪の構成要件に該当する行為を共同して行ったといえる。
		
		甲と乙は、これによってAは死亡しているから、甲と乙の行為は傷害致死罪の構成要件に該当する。
		
	\sectionB{}
		もっとも、甲と乙の暴行は、Aが暴れたことを契機としてなされているから、正当防衛(36条1項)として違法性が阻却されないか。
		\sectionC{}
			Aは酒に酔って、甲に殴りかかろうとしているが、この暴行は不正の侵害であり、これを止めてAを押さえ続けなければ、その後も暴れることが予想される情況であったから、急迫不正の侵害があったといえる。
		\sectionC{}
			「防衛するため」の行為であるといえるためには、「防衛するため」と言う文言から、防衛の意思が必要である。
			防衛の意思は主観的な要件であるから、行為者ごとに判断する。
			
			防衛の意思は攻撃の意思が併存していることを以て否定されるものではないが、専ら攻撃の意思で反撃を加える行為は、防衛の意思が認められず、正当防衛を否定すべきである。
			乙は、自身や甲の身を守る意思でAを押さえつけていると同時に、憤激の念が併存しているため、防衛の意思の存否が問題となるが、
			Aの行為が殴りかかるという攻撃性の強いものであるのに対して、乙は体を押さえるという攻撃性の低い反撃行為しか行っておらず、専ら攻撃の意思に基づいて当該行為を行ったとはいえない。
			
			甲は、Aの暴行を阻止しようとして、体を押さえつけているから、防衛の意思が当然に認められる。
			
			したがって、甲と乙の行為は、防衛するための行為であるといえる。
		\sectionC{}
			では、「やむを得ずにした行為」といえるか。
			やむを得ずにした行為とは、反撃行為が防衛手段として必要最小限度のものであること、すなわち、反撃行為が防衛手段として相当性を有するものであることをいう。
			
			甲と乙は、A一人に対して、二人がかりで暴行を加えているが、甲は52歳の女性であり単身でAを止めることはできず、また、乙もAを止めようとしたところ反撃を受けて転倒するなどしており、単身でAを止めるのに十分であったとはいえない。
			現に過去には、二人がかりで押さえつけていても、途中でAが暴れ出すこともあったというのであり、Aの攻撃に対して、防御的に体を押さえつけるという態様である限り、二人がかりであっても侵害性低い他の手段があったとはいえない。
			
			しかし、本件において、乙はAの頸部を強く押さえ続けるという侵害性の高い態様の暴行を加えて、Aを死亡させている。
			Aの暴行を防ぐためには、頸部を強く押さえ続ける必要は必ずしもなく、肩や胸の辺りを押さえることでも防衛の目的は十分に達成できたといえる。
			
			したがって、本件における反撃行為は、反撃行為として相当性を逸脱したものであり、「やむを得ずにした行為」であるとはいえない。
			
			以上により、甲と乙の行為は、正当防衛に該当せず、違法性が阻却されない。
	\sectionB{}
		では、(責任)故意は認められるか。故意とは、犯罪事実の認識及び認容をいうが、過剰性を基礎づける事実を認識していなかった場合、
		その者の認識を基礎とすると正当防衛が成立しており、犯罪事実の認識・認容があるとはいえないから、故意が阻却される。
		
		乙は、Aの頸部を強く押さえつけるという、過剰性を基礎づける事実を認識していたから、故意が認められる。
		
		一方、甲は、乙がAの体のどの部分を押さえつけていたのかをよく見ておらず、過剰性を基礎づける事実を認識していたとはいえず、故意が認められない。
		したがって、乙には傷害致死罪は成立しない。


	\sectionB{}
		以上により、乙には傷害致死罪の共同正犯(刑法60条、刑法205条)が成立し、過剰防衛として刑の任意的減免を受ける(刑法35条2項)。
		甲には、乙の行為の過剰性の認識につき過失があるとき、過失致死罪(刑法210条)が成立し、過剰防衛として刑の任意的減免を受ける(刑法35条2項)。過失が認められない場合、不可罰である。




\begin{flushright}
	以上
\end{flushright}
	
\end{document}








