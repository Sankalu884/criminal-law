\documentclass[fontsize=11pt,
jlreq_notes,
sidenote_length=12zw
]{jlreq}

% 1. フォント・日本語処理
\usepackage{pxchfon} 
\usepackage{otf}

% 2. フォント設定
\setminchofont{A-OTF-RyuminPro-Light.otf}
\setgothicfont{A-OTF-FutoGoB101Pr6N-Bold.otf}

% 3. 汎用パッケージ
\usepackage{fancybox}

% 4. 自作パッケージ
\usepackage[sect]{kian}


\title{\vspace{-30mm}{\textgt{\Large{\fbox{29} 改ざんされた試験結果・その1}}}}
\date{\vspace{-15mm}}


\begin{document}

\maketitle
\sectionA{甲の罪責}
	甲が、改ざんした結果一覧表をBに提出して、虚偽の公文書を作成させた行為について、有印虚偽文書作成罪(刑法156条)、同行使罪(刑法158条1項)の間接正犯は成立するか。甲が本問の公文書を作成する権限を有しないことから問題となる。
		\sectionB{}
			ここで、作成権限がなくとも、作成権限を有する者を介して虚偽公文書を作成することは可能であるから、「職務に関し」といいうる限度で、公務員が間接的に虚偽公文書を作成する場合には、間接正犯の成立を肯定することができる。
	
		本問において、甲は公務員であり、かつ、文書の原案を作成する業務を担当していたのであるから、甲が結果を改ざんして、内容虚偽の原案を作成すれば、職務に関して、作成権限者であるBを道具のように利用して虚偽文書を作成したといえる。
		
		そして、現に、Bは甲が持ってきた原案の、採用試験結果一覧表の作成者「B」 と印字された横に「B」と刻した印鑑を押印して、内容虚偽の有印公文書を作成している。
	
		甲には、行使の目的も故意も認められるから、虚偽公文書作成罪(刑法158条1項)の間接正犯が成立する。
		
		\sectionB{}
			また、当該公文書が人事課の職員であれば誰でもその業務のために使用することができるファイルボックスに置かれた時点で、行使されたと評価できるから、虚偽公文書行使罪(158条1項)も成立する。
		\sectionB{罪数}
			甲に成立する有印虚偽文書作成罪(刑法156条)と虚偽公文書行使罪(158条1項)は通例、目的手段の関係にあるから、牽連犯(刑法54条1項後段)となる。
			
\sectionA{乙の罪責}
	乙が、甲の原案を改ざんし、Bに虚偽の公文書を作成させた行為につき、有印公文書作成罪の間接正犯は成立するか。
	
	ここで、同罪は、公務員を主体とする身分犯である以上、私人である乙が正犯として処罰されることはない。したがって、間接正犯は成立しない\sidenote{狭義の共犯の成立可能性(非故意行為に対する教唆犯?)}。







\raggedleft{以上}
	
\end{document}








