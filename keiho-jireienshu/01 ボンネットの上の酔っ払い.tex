\documentclass[11pt]{jsarticle}

\usepackage[sect]{kian}
\usepackage{otf}
\usepackage{fancybox}
\usepackage{ascmac}
\usepackage[noalphabet]{pxchfon}  
\setminchofont{A-OTF-RyuminPro-Light.otf}
\setgothicfont{A-OTF-FutoGoB101Pr6N-Bold.otf}



\title{\vspace{-30mm}{\textgt{\Large{\fbox{1} ボンネットの上の酔っ払い}}}}
\date{\vspace{-15mm}}


\begin{document}

\maketitle

\sectionA{}
	甲がAの顔面を手拳で殴打した行為について、暴行罪(刑法208条)の成否を検討する。
	
	\sectionB{}
		甲はAを殴打しており、暴行罪(刑法208条)の客観的構成要件に該当する。また、これらの犯罪事実を認識認容しているから、故意も認められる。
		
	\sectionB{}
		もっとも、甲がAを殴打したのは、Aが甲の車の窓から手を入れてきて、甲の胸ぐらをつかもうとしたことに端を発するから、甲には正当防衛(刑法36条1項)が成立し、違法性が阻却されるのではないかが問題となる。
		
		正当防衛が成立するためには、甲の行為が、\UTF{2460}急迫不正の侵害に対し、\UTF{2461}自己又は他人の権利を防衛するため、\UTF{2462}やむを得ずにした行為であると評価できる必要がある。
		
		\sectionC{}
			まず、急迫不正の侵害とは、違法な法益侵害が現に存在するか、又は間近に押し迫っていることをいう。Aは甲の胸ぐらをつかもうとしているから、甲の身体の安全に対する違法な法益侵害が間近に押し迫っているといえる。
			
		\sectionC{}
			次に、36条の「防衛するため」という文言から、正当防衛の成立には、防衛の意思が必要である。防衛の意思とは、急迫不正の侵害を認識しつつ、これを避けようとする単純な心理状態をいう。甲の行為は、専らAを攻撃する意思に基づくものとは認められず、自身の身体の安全を守る意思で行った行為であり、防衛の意思が認められる。
			
			\sectionC{}
				やむを得ずにした行為とは、反撃行為が、防衛手段として必要最小限度のものであること、すなわち反撃行為が防衛手段として相当性を有するものであることをいう。必要最小限度かどうかは、可能な防衛手段の選択肢及び態様を考慮した上で具体的に判断する。
				
胸ぐらをつかむ行為と比べて、手拳で顔面を殴打する行為の危険性が特段高いとは言えず、その場の状況から見ても、これに代わる危険性のより小さい手段があったとも認められない。したがって、甲の行為は防衛手段として必要最小限度のものであり、相当性が認められる。

	\sectionB{}
		以上より、甲の行為は正当防衛として違法性が阻却され、暴行罪は成立しない。
		
\sectionA{}
	甲がBに向けて車を発進させた行為について、暴行罪の結果的加重犯としての傷害罪(刑法204条)の成否を検討する。
	
	\sectionB{}
		暴行とは、人に対する物理力の行使をいうが、その物理力は必ずしも身体に接触する必要はない。したがって、甲の行為は「暴行」にあたる。Bに向けて車を発進することを認識認容しているから、暴行の故意が認められる。
		
		ここで、Bには全治一週間の打撲傷という傷害結果が発生している。Bの傷害結果は、甲が車を発進させた行為の危険性が直接実現したものであるから、因果関係も認められる。
		
		傷害罪には、暴行罪の結果的加重犯の場合を含むから、甲の行為は、傷害罪の構成要件に該当する。
		
	\sectionB{}
		では、甲に正当防衛は成立しないか。
		
		まず、甲は車で追いかけられた上で、武器として使用しうる棒切れ様の物を持ったAに近づかれており、Bもこれを制止するでもなく背後から近づいてきており、甲の身体の安全に対する違法な法益侵害が間近に押し迫っているといえる。
		
		次に、甲は暴行から逃れるために車を発進させており防衛の意思が認められる。
		
		また、甲は1人であったのに対し、ABは2人で、Aは棒切れ様の物を所持し、深夜で助けを呼ぶのも困難であった。この状況で、直接接触を避けつつ車を発進させる行為は、予想される暴行に比べて特に危険性が高いとはいえず、また、より安全な手段もなかった。よって、甲の行為は防衛手段として必要最小限度であり、相当性が認められる。
		
	\sectionB{}
		以上より、甲の行為には正当防衛が成立し、違法性が阻却されるから、傷害罪は成立しない。
		
\sectionA{}
	甲が、Aをボンネットに乗せ状態で車を発進させた上、蛇行運転を行い、急ブレーキをかけて転落させた行為につき、殺人未遂罪(刑法203条、刑法199条)の成否を検討する。
	
	\sectionB{}
		まず、甲の行為は、殺人罪の実行行為といえるか。実行行為とは、結果発生の現実的危険性を有する行為をいうから、甲の行為に死亡結果発生の現実的危険性が認められるかを検討する。
		
		人をボンネットに乗せたまま、時速70キロメートルという高速度で蛇行運転急ブレーキををする行為は、車から転落して頭部などを負傷したり、そのまま車に轢過されたりするおそれがあり、いずれも死亡結果を発生させる現実的危険性が認められる。したがって、甲の行為は殺人罪の実行行為であるといえる。
		
		次に、甲は死亡結果を発生させる危険性のある行為であることを認識している。また、危険性を認識しつつ安全な運転方法に切り替えることはなかったから、死亡結果を認容しているといえる。したがって、殺人罪の故意が認められる。
		
		Bに死亡結果は発生していないから、甲の行為は殺人未遂罪の構成要件に該当する。
		
	\sectionB{}
		では、甲に正当防衛は成立しないか。
		
		Aは武器として使用できる棒切れ様の物を手にしながら近づいてきており、車が発進した後も手を離そうとしておらず、急迫不正の侵害が認められる。また、甲はAおよびBから逃げようとしているから、防衛の意思も認められる。
		
		問題は、甲の行為が「やむを得ずにした行為」といえるかである。
		
		甲の反撃行為は、死亡結果を発生させ得る攻撃性の高い行為であった一方、スピードを落として運転することや、Bから離れて侵害者の数的優位が失わせた時点で警察を呼ぶなど、より防御的な手段をとることも可能であった。したがって、甲の反撃行為は防衛手段として必要最小限度とはいえない。なお、Bが、軽傷にとどまったとしても、防衛行為の相当性は結果の軽重ではなく、反撃行為が防衛手段として必要最小限度のものであったかによって判断されるから、現実の結果は相当性の判断に影響しない。
		
	\sectionB{}
		以上より、甲の行為に正当防衛は成立せず、違法性は阻却されないから、殺人未遂罪が成立する。なお、甲の行為は防衛の程度を超えた行為なので過剰防衛として任意的に刑の減免がされる(刑法36条2項)。



\begin{flushright}
	以上
\end{flushright}
	
\end{document}








